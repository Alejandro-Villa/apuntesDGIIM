
\section{El plano proyectivo}

\begin{ejer}
  Encontrar en $\mathbb{P}^2$ la recta que pasa por los puntos A(1,2,3) y B(-1,0,2).
\end{ejer}

\begin{ejer}
	Sean $ABCD$ y $A'B'C'D'$ dos cuadriláteros en el plano protectivo $\mathbb{P}^2$ tales que $A'B'C'D'$ está inscrito en $ABCD$ y sus cuatro rectas diagonales son concurrentes. Entonces  los cuatro puntos diagonales de los cuadrilateros están alineados.
	\begin{nlist}
		\item Enunciar una versión afín del teorema
		\item Demostrar el teorema
\end{nlist}
\end{ejer}

\begin{nlist}
	\item Versión afín del teorema: $ABCD$ y $A'B'C'D'$ dos cuadriláteros en $\mathbb{R}^2$ tales que $A'B'C'D'$ está inscrito en $ABCD$, y sus cuatro rectas diagonales son concurrentes. Entonces, si $ABCD$ es un paralelograma, $A'B'C'D'$ también lo es.
	\item Probamos el teorema en el afín. Sea $\sigma$ la reflexión central de centro $O$, el punto donde se cortan las diagonales. Por ser $ABCD$ un paralelogramo, sabemos que $O$ es el punto medio de los vértices opuestos, luego:
	\[
	\begin{rcases}
		A \leftrightarrow^\sigma C\\
		B \leftrightarrow^\sigma D
\end{rcases} \implies
	\begin{rcases}
		A\vee B \leftrightarrow C \vee D\\
		A \vee D \leftrightarrow B \vee C
\end{rcases}
\]
Ahora, como por hipótesis sabemos que la recta $E \vee G$ pasa por $O$, entonces la reflexión central lleva la recta en ella misma, luego:
$$E = (A \vee B) \cap (E \vee G) \leftrightarrow^\sigma (C \vee D) \cap (E \vee G) = G$$
$$H = (A \vee D) \cap (F \vee H) \leftrightarrow^\sigma (B \vee C) \cap (F \vee H) = H$$
Ahora sabemos que una homotecia aplicada sobre una recta nos da una recta paralela a ella, y por tanto:
\[
\begin{rcases}
	E\vee H \leftrightarrow^\sigma F \vee G \implies E \vee G \parallel F \vee G\\
	E\vee F \leftrightarrow^\sigma E \vee F \implies E \vee F \parallel H \vee G
\end{rcases} \implies EFGH \text{ es un paralelogramo}
\]
\end{nlist}




\end{document}
