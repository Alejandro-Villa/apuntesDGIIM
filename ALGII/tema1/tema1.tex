\subsection{El anillo de enteros módulo n}

Para cada $n \ge 1$ consideramos el conjunto $\mathbb{Z}_n$, obtenido mediante el cociente $\mathbb{Z}\,/\,n\mathbb{Z}$. O lo que es lo mismo, mediante la relación de equivalencia:
\begin{equation*}
	x \sim y \iff x,y \in \mathbb{Z}_n \iff x-y \in n\mathbb{Z} \iff n | (x-y)\,.
\end{equation*}
Considerando en $\mathbb{Z}_n$ la estructura de dominio euclídeo tenemos que por algoritmo de Euclides: $$\forall a,b \in \mathbb{Z} \; b \neq 0 \; \exists! q,r \; a = bq+r \; con \; 0 \leq r < |b| $$ y por tanto existen exactamente $n$ clases de equivalencia en $\mathbb{Z}_n$ y una colección de representantes es $\{0,1,...,n-1\}$.

Denotaremos al resto $r$ como $r = \res(a,b)$ o $a \equiv r \; \mod(b)$ y para poder operar con los representantes adecuados definiremos las operaciones:

$$i+j = \res(i+j,n)$$

y

$$ij = \res(ij,n)$$

\begin{nprop}
$\mathbb{Z}_n$ es un anillo conmutativo cuyo elemento neutro para la suma es la clase cuyo representante es el cero y cuyo elemento neutro para el producto es la clase cuyo representante es el uno. Además:

$\mathbb{Z}_n$ es un cuerpo $\iff n$ es un número primo.
\end{nprop}

\begin{proof}

Demostraremos sólo la última afirmación. 

$\Rightarrow$ Por reducción al absurdo, si $\mathbb{Z}_n$ es un cuerpo y asumimos que n no es primo entonces n puede factorizarse como $n = r \cdot s$ donde $1 < r,s < n$. Luego en $\mathbb{Z}_n$ se verifica que $0 = r \cdot s$ y por tanto a o b son divisores de cero. Esto es una contradicción ya que un cuerpo es siempre un dominio de integridad y en un dominio de integridad no hay divisores de cero distintos de cero.

$\Leftarrow$ Por otro lado, si n es primo y tomamos $1 \leq r \leq n-1$ se verifica que $\mcd(r,n) = 1$ y por el teorema de Bézout $\exists a,b \in \mathbb{Z}_n$ tales que $1 = an + br$ y tomando restos en $\mathbb{Z}_n$ queda que $1 = br$ de donde b es el inverso de r.

\end{proof}

Recordemos que las unidades de un anillo era el conjunto: $$\mathcal U(A) = \{a \in A : \exists a^{-1} \in A \; tal \; que \; aa^{-1}=1=a^{-1}a\}$$ Las unidades de los anillos de enteros $\mathbb{Z}_n$ son conocidas:

\begin{nprop}[Unidades de los anillos de restos módulo n]
$\mathcal U(\mathbb{Z}_n)$ = $\{r \in \mathbb{Z}_n : r \neq 0 \; y \; \mcd(r,n) = 1 \}$
\end{nprop}

\begin{proof}

Llamemos A = $\{r \in \mathbb{Z}_n : r \neq 0 \; y \; \mcd(r,n) = 1 \}$ y probemos la igualdad con $\mathcal U(\mathbb{Z}_n)$ por doble inclusión.

Veamos que $A \subseteq \mathcal U(\mathbb{Z}_n)$. Si $a \in A$ por el teorema de Bézout $\exists r,s \in \mathcal U(\mathbb{Z}_n)$ tales que $1 = nr+as$ y tomando restos módulo n se verificará que $1 = as$. Luego $a \in \mathcal U(\mathbb{Z}_n)$.

Veamos que $\mathcal U(\mathbb{Z}_n) \subseteq A$. Si $u \in \mathcal U(\mathbb{Z}_n)$ tenemos que en $\mathbb{Z}_n$, $\exists u^{-1}$ tal que $uu^{-1} = 1$ y por tanto, en $\mathbb{Z}$ tendremos que $uu^{-1} = 1 + ny$. Sea $d = \mcd(u,n)$ entonces $d$ divide a u y divide a n por lo que debe dividir a 1, y por tanto, debe ser 1.

\end{proof}

\subsection{Función phi de Euler}

\begin{ndef}[Función phi de Euler]
La función phi de Euler está dada por: $$\phi(n) := |\mathcal U(\mathbb{Z}_n)|$$
\end{ndef}

\begin{nprop}[Propiedades de la función phi de Euler]
1. Si $\mcd(m,n) = 1$ entonces $\phi(mn) = \phi(m) \phi(n)$. \\
2. Si $p \ge 1$ es un número primo entonces $\phi(p^{e}) = p^{e-1}(p-1)$.
\end{nprop}

\begin{proof}
Veamos 1. Usamos la propiedad de las unidades del anillo producto que dice que $$\mathcal U(\mathbb{Z}_m \times \mathbb{Z}_n) = \mathcal U(\mathbb{Z}_m) \times \mathcal U(\mathbb{Z}_n)$$ y el teorema chino de los restos que dice que $$\mcd(m,n) = 1 \iff \mathbb{Z}_{mn} \cong \mathbb{Z}_m \times \mathbb{Z}_n$$ De este modo, aplicando la primera propiedad se obtiene que $$|\mathcal U(\mathbb{Z}_m \times \mathbb{Z}_n)| = |\mathcal U(\mathbb{Z}_m)| \times |\mathcal U(\mathbb{Z}_n)| = \phi(m) \phi(n)$$ y aplicando la segunda y el hecho de que los isomorfismos mantienen el número de unidades del anillo tendremos que $$\phi(mn) = |\mathcal U(\mathbb{Z}_{mn})| = |\mathcal U(\mathbb{Z}_m \times \mathbb{Z}_n)|$$

Veamos 2. Sea $p \ge 1$ un número primo. $$\phi(p^e) = |\mathcal U(\mathbb{Z}_{p^e})| = |\{r \in \mathcal U(\mathbb{Z}_{p^e}) : r \neq 0, \mcd(r,p^e) = 1\}| = |\mathbb{Z}_{p^e}|-|\{0,p,2p,...,(p-1)p^{e-1}\}| = p^e - p^{e-1} = p^{e-1}(p-1)$$
\end{proof}
