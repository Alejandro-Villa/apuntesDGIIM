\section{Teoría general de procesos estocásticos}

\subsection{Definición y propiedades generales}

\subsubsection{Teoría de la medida}

Veremos primero nociones básicas sobre la teoría de la medida.

\begin{ndef}[$\sigma$-álgebra]
  Una \emph{$\sigma$-álgebra} $\A$ sobre un conjunto $\Omega$ es una familia de subconjuntos de $\Omega$ ($\A \subset \Part(\Omega)$) que cumple:

  \begin{nlist}
    \item $\forall A \in \A \implies \overline{A} \in \A$
    \item $\forall \{A_n\}_{n \in \N} \subset \A \implies \bigcup \limits^\infty_{n = 1} A_n \in \A$
  \end{nlist}
\end{ndef}

\begin{nota}
  $\Part(\Omega)$ son las partes de $\Omega$, es decir, todos los subconjuntos posibles de $\Omega$.
\end{nota}

La pertenencia del total, vacío y ser cerado para intersecciones se deduce de las condiciones de la definición.

\begin{ndef}[Espacio medible]
  Un \emph{espacio medible} es una tupla $(\Omega, \A)$ donde $\Omega$ es un conjunto y $\A$ es una $\sigma$-álgebra.
\end{ndef}

Para la teoría de la probabilidad tiene especial interés los espacios Borel, en los que toman valores las variables aleatorias.

\begin{ejemplo}
  $(\R, \B)$, $(\R^n, \B^n)$.
\end{ejemplo}

\begin{ndef}[$\sigma$-álgebra Borel]
  La \emph{$\sigma$-álgebra de Borel} es la generada por las semirrectas:

  $$\B^n = \sigma(\mathfrak{I}^n), \; \mathfrak{I}^n = \{ (-\infty, x]: x \in \R^n \}$$
\end{ndef}

\begin{nota}
  $\sigma(D)$ es la $\sigma$-álgebra minimal generada por D.
\end{nota}

\begin{ndef}[$\sigma$-álgebra Borel restringida]
  Sea un subconjunto $E \subset \R$, la \emph{$\sigma$-álgebra restringida a E} es $$\B_E = \{ B \cap E : B \in \B \},$$
\end{ndef}

\begin{ndef}[Espacio Borel restringido]
  Sea un subconjunto $E \subset \R$, el \emph{espacio Borel restringido a E} es $(E, \B_E)$.
\end{ndef}

Ahora veamos que pasa con las aplicaciones y funciones medibles.

\begin{ndef}[Aplicación medible]
  Sean dos espacios medibles $(\Omega_1, \A_1)$, $(\Omega_2, \A_2)$ una aplicación $f : \Omega_1 \to \Omega_2$ se dice \emph{aplicación medible} $\iff \forall A \in \A_2$, $f^{-1}(A)\in \A_2 \iff f^{-1}(\A_2) \subset \A_1$.
\end{ndef}

\begin{nprop}[Caracterización de aplicaciones medibles]
  Una aplicación entre espacios medibles $f : (\Omega_1, \A_1) \to (\Omega_2, \A_2)$ es aplicación medible $\iff f^{-1}(D) \subset \A_1$ siendo $D \subset P(\Omega_2)$ y $\sigma(D) = \A_2$.
\end{nprop}

\begin{nprop}[Composición de aplicaciones medibles]
  Sean dos aplicaciones medibles $f : (\Omega_1, \A_1) \to (\Omega_2, \A_2)$, $g : (\Omega_2, \A_2) \to (\Omega_3, \A_3)$ entre tres espacios medibles. Entonces la composición $g \circ f : (\Omega_1, \A_1) \to (\Omega_3, \A_3)$ es una aplicación medible.
\end{nprop}

\begin{proof}
  $(g \circ f)^{-1}(\A_3) = f(g^{-1}(\A_3)) \in \A_1$ porque f es aplicación medible y $g^{-1}(\A_3) \in \A_2$ (g es aplicación medible).
\end{proof}


\begin{ndef}[Función medible]
  Una \emph{función medible} es una aplicación medible cuyo espacio de llegada es un espacio de Borel.
\end{ndef}

\begin{nota}
  Si el espacio Borel es $\R$ se denomina función medible \emph{finita} (no toman valores en $\pm \infty$).
\end{nota}

\begin{nprop}[Caracterización de funciones medibles (1)]
  Una función entre un espacio medible y espacio Borel $f : (\Omega, \A) \to (\R^n, \B^n)$ es función medible $\iff f^{-1}(\B^n) \subset \A$.
\end{nprop}

\begin{nota}
  Para $n = 1$ se dice función medible (\emph{unidimensional}), para $n>1$ se denomina función medible \emph{multidimensional}.
\end{nota}

\begin{nprop}[Caracterización de funciones medibles (2)]
  Una función entre un espacio medible y espacio Borel $f : (\Omega, \A) \to (\R^n, \B^n)$ es función medible $\iff f^{-1}((-\infty, x]) \in \A, \; \forall x \in \R$.
\end{nprop}

\begin{nota}
  $f^{-1}(B) = \{ \w \in \Omega : f(\w) \in B \} = [f \in B]$.

  $f^{-1}((-\infty, x]) = \{ \w \in \Omega : f(\w) \leq x \} = [f \leq x]$.
\end{nota}

\begin{ndef}[Función medible Borel]
  Una \emph{función medible Borel} es una función medible cuyo espacio inicial también es espacio de Borel.
\end{ndef}

Ejemplos de funciones medibles:

\begin{ejemplo}
  $\forall A \in \A$, función indicadora $1_A(\omega) =
    \begin{cases}
      1, \; \w \in A \\
      0, \; \w \notin A
    \end{cases}$
\end{ejemplo}

\begin{ejemplo}
  Función constante $f(\w) = c \in \R$.
\end{ejemplo}

\begin{ejemplo}
  Sean $A_1, \ldots, A_n \in \A, x_1, \ldots, x_n \in \R$, función simple $f(\w) = \sum \limits^n_{i = 1} x_i 1_{A_i}(\w)$.
\end{ejemplo}

\begin{nota}
  Cualquier función que parte del espacio medible $(\W, P(\W))$ es medible.
\end{nota}

Veamos que la medibilidad de una función depende de la $\sigma$-álgebra del espacio de partida:

\begin{ejemplo}
  $f(\w) = \w$ es medible si $f : (\R, \B) \to (\R, \B)$, pero no lo es si $f : (\R, \A) \to (\R, \B)$ con $\A = \{ A \subset \R : A \text{ o } \overline{A} \text{ es numerable} \}$.
\end{ejemplo}

Finalmente sobre medidas y espacios medibles:

\begin{ndef}[Función $\sigma$-aditiva]
  Una \emph{función de conjunto $\sigma$-aditiva definida sobre $\A$} es una función $\varphi : \A \to \R$ que cumple:

  $$ \forall \{A_n\}_{n \in \N}, \, A_i \cap A_j = \emptyset \, \forall i \neq j, \; \varphi(\bigcup \limits^{\infty}_{n = 1} A_n) = \sum \limits^{\infty}_{n = 1} \varphi(A_n) $$
\end{ndef}

\begin{ndef}[Medida]
  Una \emph{medida} $\mu$ es una función $\sigma$-aditiva \emph{no negativa}, es decir que $\forall A \in \A \; \; \mu(A) \geq 0$.
\end{ndef}

\begin{ndef}[Espacio de medida]
  Un \emph{espacio de medida} es una terna $(\W, \A, \mu)$ formada por un espacio medible y una medida sobre dicho espacio.
\end{ndef}

\begin{ndef}[Integral en espacio de medida]
  Sea una función medible $f: (\W, \A, \mu) \to (\R, \B)$ entonces definimos la \emph{integral de $f$} en este espacio como:

  $$\int f d\mu := \int_{\W} f d\mu = \int_A f \mu = \int f \chi_A d\mu$$
\end{ndef}

\subsubsection{Teoría de la probabilidad}
La teoría de la probabilidad se desarrolla sobre un tipo especial de espacios de medida llamados \emph{espacios de probabilidad}, a los elementos de la $\sigma$-álgebra se les llama \emph{sucesos}; y a la medida \emph{probabilidad}.

\begin{ndef}[Probabilidad]
  Una función $P: \A \to \R$ se dice que es una \emph{probabilidad} si cumple con los tres axiomas de Kolmogorov:
  \begin{nlist}
    \item $\forall A \in \A$, $P(A) \geq 0$ (no negativa).
    \item $\forall \{A_n\}_{n \in \N} \subset \A$ disjuntos $\implies P(\bigcup \limits^\infty_{n = 1} A_n) = \sum \limits^\infty_{n = 1} P(A_n)$ ($\sigma$-aditividad).
    \item $P(\W) = 1$ (normalización).
  \end{nlist}
  Luego una probabilidad es una medida normalizada.
\end{ndef}

\begin{ndef}[Variable aleatoria]
  Una \emph{variable aleatoria}(v.a) $X : \W \to \R$, $\w \mapsto X(\w)$ es una función medible de un espacio de probabilidad en un espacio Borel; es decir:

  $$X : (\W, \A, P) \to (\R, \B), \, \, \forall B \in \B \; \; X^{-1}(B) \in \A$$
\end{ndef}

\begin{nprop}[Caracterización de variables aleatorias]
  $X$ es una variable aleatoria $\iff \forall x \in \R \; \; X^{-1}((-\infty, x]) = [X \leq x] \subset \A$.
\end{nprop}

\begin{ndef}[Vector aleatorio]
  Un \emph{vector aleatorio}(v.a) $X : \W \to \R^n$, $\w \mapsto X(\w)$, es una función medible multidimensional de un espacio de probabilidad en un espacio Borel (multidimensional).
\end{ndef}

La caracterización anterior es válida para vectores aleatorios, pero además veamos otra:

\begin{ndef}[Caracterización de vectores aleatorios]
  $X$ es un vector aleatorio $\iff \forall i = 1, \ldots, n, \; X_i : (\W, \A, P) \to (\R, \B)$ es una variable aleatoria.
\end{ndef}

\begin{proof}
  Se deja propuesto como ejercicio.
\end{proof}


\begin{ndef}[Distribución de probabilidad]
  Se llama \emph{distribución de probabilidad} de una variable aleatoria a una función de probabilidad definida en el espacio de Borel:
  \begin{align*}
    P_X : \B & \to \R \\
    B & \mapsto P_X(B) = P(X^{-1}(B)) = P[X \in B]
  \end{align*}
\end{ndef}

\begin{proof}
  Hecho en \ref{ej:distributionproof}
\end{proof}

El teorema de correspondencia nos dice que la función de conjunto $P_X$ puede ponerse en correspondencia biunívoca con la función de distribución $F_X$.

\begin{ndef}[Función de distribución]
  La \emph{función de distribución} es la función de puntos:
  \begin{align*}
    F: \R & \to \R \\
    x & \mapsto  F_X(x) = P_X((-\infty, x]) = P[X \leq x]
  \end{align*}
\end{ndef}

El teorema de correspondencia nos permite también establecer una correspondencia biunívoca entre la función de distribución y la función característica $\varphi_X$.

\begin{ndef}[Función característica]
  La \emph{función característica} se define como $\varphi_X(t) = E[e^{itX}], \; \; \forall t \in \R$.
\end{ndef}

La correspondencia entre estos tres tipos de funciones se llama \emph{ley de la variable aleatoria} y existe también la versión análoga para vectores aleatorios.

\begin{ndef}[Función masa de probabilidad]
  Sea $X$ una v.a discreta, se define la \emph{función masa de probabilidad} como $f(x) = P[X = x], \; \; \forall x \in \B_X$.
\end{ndef}

Su función característica asociada es $\varphi_X(t) = \sum \limits_{x \in \B_X} e^{itx} P[X = x]$, $\forall t \in \R$.

\begin{ndef}[Función de densidad]
  Sea $X$ una v.a continua, la \emph{función de densidad} es la que cumple $F_X(x) = \int^x_{-\infty} f(y) dy, \; \; \forall x \in \R $
\end{ndef}

Su función característica asociada es $\varphi_X(t) = \int e^{it} f(x) dx$, $\forall t \in \R$.

\subsubsection{Definición de proceso estocástico}

Veamos que es un proceso estocástico y ciertas definiciones sobre él:

\begin{ndef}[Proceso estocástico]
  Un \emph{proceso estocástico}(p.e) es una familia de variables aleatorias ${\{X_t\}}_{t \in T}$, donde $T$ un conjunto ordenado arbitrario y cada v.a está definida sobre un espacio de probabilidad $(\W, \A, P)$.
\end{ndef}

\begin{nota}
  En algunas ocasiones llamaremos al proceso estocástico como proceso.
\end{nota}

\begin{ndef}[Espacio paramétrico]
  Se llama \emph{espacio paramétrico} al conjunto ordenado arbitrario $T$: en el caso discreto se suele tomar $\N \cup \{0\}$, y en el continuo $[0, +\infty)$.
\end{ndef}

\begin{ndef}[Proceso estocástico real]
  Un p.e se dice que es \emph{real} si $\forall t \in T$, $X_t : (\W, \A, P) \to (\R, \B)$.
\end{ndef}

\begin{nota}
  En este curso son los procesos estocásticos que consideraremos, junto tambien a los p.e multidimensionales (espacio Borel de llegada muldimensional).
\end{nota}

\begin{ndef}[Espacio de estados]
  Se llama \emph{espacio de estados} al espacio Borel donde toman valores las v.a. En general, sea $E \subset \R$ con la $\sigma$-álgebra Borel restringida $\B_E$, el espacio de estados es $(E, \B_E)$,
\end{ndef}

\begin{nota}
  Si no se especifica uno diferente, se considera $(\R, \B)$.
\end{nota}

\begin{ndef}[Trayectoria]
  $\forall \w \in \W$ fijo, definimos la \emph{trayectoria} asociada a $\w$ como
  \begin{align*}
    X(w): T & \to \R \\
    t & \mapsto X_t(w).
  \end{align*}
\end{ndef}

\begin{ejemplo}
  Proceso de recuento:
  \begin{itemize}
    \item $X_t \equiv$ número de veces que ocurre un suceso en el intervalo $[0, t)$.
    \item $T = [0, + \infty]$.
    \item $E = \N \cup \{0\}$.
    \item $s \leq t \in T \implies X_s \leq X_t$ trayectorias crecientes a saltos.
  \end{itemize}
\end{ejemplo}

\begin{ejemplo}
  Recorrido aleatorio:
  \begin{itemize}
    \item $X_n \equiv$ posición de la particula en el instante n.
    \item $T = \N \cup \{0\}$.
    \item $E = \mathbb{Z}$.
  \end{itemize}
\end{ejemplo}

\subsubsection{Algunas características de procesos estocásticos}

Sea $\{X_t\}_{t \in T}$ p.e definido sobre el espacio de probabilidad $(\Omega, A, P)$, suponemos que existen las esperanzas de las siguientes definiciones:

\begin{ndef}[Función media]
  La \emph{función media} es una función que asigna a cada t la esperanza de la variable aleatoria asociada a t, es decir:
  \begin{align*}
    \mu: T & \to \R \\
    t & \mapsto \mu(t) = \mu_t = E[X_t]
  \end{align*}
\end{ndef}

\begin{nota}
  Para que la definición sea válida se debe tener $\forall t \in T$, $E[|X_t|] < \infty$.
\end{nota}

\begin{ndef}[Proceso centrado]
  Un p.e se dice \emph{centrado} si se cumple $\mu(t) = 0$, $\forall t \in T$.
\end{ndef}

\begin{ndef}[Momentos no centrados de orden k]
  El \emph{momento no centrado de orden k} se define como:
  \begin{align*}
    \mu_k : T & \to \R \\
    t & \mapsto \mu_k(t) = E[X^k_t]
  \end{align*}
\end{ndef}

\begin{ndef}[Momentos centrados de orden k]
  El \emph{momento centrado de orden k} se define como:
  \begin{align*}
    m_k : T & \to \R \\
    t & \mapsto m_k(t) = E[(X_t - \mu_t)^k]
  \end{align*}
\end{ndef}

\begin{ndef}[Función varianza]
  La \emph{función varianza} es $\sigma^2_t = m_2(t)$
\end{ndef}

\begin{ndef}[Función correlación]
  La \emph{función correlación} se define como:
  \begin{align*}
    R : T \times T & \to \R \\
    (s,t) & \mapsto R(s,t) = E[X_s X_t]
  \end{align*}
\end{ndef}

\begin{ndef}[Función covarianza]
  La \emph{función covarianza} se define como:
  \begin{align*}
    C: T \times T & \to \R \\
    (s,t) & \mapsto C(s,t) = E[(X_s - \mu_s)(X_t - \mu_t)] = R(s,t) - \mu_s \mu_t
  \end{align*}
\end{ndef}

\begin{nota}
  Consideraciones:
  \begin{itemize}
    \item $C(t,t) = \sigma^2_t$.
    \item Si el proceso es centrado entonces la correlación y la covarianza coinciden.
  \end{itemize}
\end{nota}

\begin{ndef}[Procesos de segundo orden]
  Un p.e $\{X_t\}_{t \in T}$ se dice de \emph{segundo orden} si $\forall t \in T, \, E[X^2_t] < \infty$ (las v.a son cuadrado integrables).
\end{ndef}

\subsection{Clasificación de los procesos estocásticos}

\subsubsection{Clasificación atendiendo al $T$ y $E$}
Sea $\{X_t\}_{t \in T}$ p.e definido sobre el espacio de probabilidad $(\W, A, P)$ con espacio de estados $(E, \B_E)$.

\begin{ndef}[Procesos en tiempo discreto]
  Un p.e es en \emph{tiempo discreto} si $T$ es discreto, es decir, $T \subset \mathbb{Z}$.
\end{ndef}

Nota: para TD usaremos $T = \N \cup \{ 0 \}$ y notamos $\{X_n\}_{n \in \N^*}$ (sucesiónes de v.a).

\begin{ndef}[Procesos en tiempo continuo]
  Un p.e es en \emph{tiempo continuo} si $T$ es continuo, es decir, $T \subset \R$.
\end{ndef}

Nota: para TC usaremos $T = [0, +\infty]$ y notamos $\{X_t\}_{t \in \R^+}$.

\begin{ndef}[Procesos discretos]
  Un p.e es \emph{discreto} si $E$ es discreto, es decir, $E$ es conjunto numerable.
\end{ndef}

Nota: también se les denota como Cadenas.

\begin{ndef}[Procesos continuos]
  Un p.e es en \emph{continuo} si $E$ es continuo, es decir $E$ es conjunto no numerable.
\end{ndef}

Veamos unos ejemplos:

\begin{ejemplo}[PDTD]
  Resultado de lanzar un dado en el n-ésimo lanzamiento.

  $T = \N$, $E = \{1,2,3,4,5,6\}$.
\end{ejemplo}

\begin{ejemplo}[PCTD]
  Cantidad de lluvia en el n-ésimo día.

  $T = \N$, $E = \R^+_0$.
\end{ejemplo}

\begin{ejemplo}[PDTC]
  Cantidad de clientes en el instante t.

  $T = [0, +\infty)$, $E = \N \cup \{0\}$.
\end{ejemplo}

\begin{ejemplo}[PCTC]
  Cantidad de lluvia en el instante t.

  $T = [0, +\infty)$, $E = \R^+_0$.
\end{ejemplo}

\subsubsection{Clasificación atendiendo a la relación entre las variables del proceso}

Sea $\{X_t\}_{t \in T}$ p.e definido sobre el espacio de probabilidad $(\Omega, \A, P)$.

\begin{ndef}[Proceso independiente]
  Un p.e es \emph{independiente} si $\forall n > 1, t_1, \ldots, t_n \in T$ $$X_{t_1}, \ldots, X_{t_n} \text{ son v.a independientes.}$$
\end{ndef}

\begin{ndef}[Proceso con incrementos independientes]
  Un p.e tiene \emph{incrementos independientes} si $\forall n > 1, \forall t_1 < \ldots < t_n \in T$ $$X_{t_1}, X_{t_2} - X_{t_1}, \ldots, X_{t_n} - X_{t_{n-1}} \text{son v.a independientes.}$$
\end{ndef}

\begin{ndef}[Proceso con incrementos estacionarios]
  Un p.e tiene \emph{incrementos estacionarios} si $\forall s < t \in T, \forall h$  $$(X_t - X_s) \sim (X_{t+h} - X_{s+h}).$$
\end{ndef}

\begin{nota}
  $X \sim Y$ indica que $X$ sigue la misma distribución que $Y$.
\end{nota}

\begin{ndef}[Proceso estacionario]
  Un p.e es \emph{estrictamente estacionario} si $\forall n, \forall t_1 < \ldots < t_n \in T, \forall h$ $$(X_{t_1}, \ldots, X_{t_n}) \sim (X_{t_1 + h}, \ldots, X_{t_n + h}).$$
\end{ndef}

\begin{ndef}[Proceso débilmente estacionario]
  Un p.e de 2º orden es \emph{débilmente estacionario} si cumple
  \begin{nlist}
    \item Su función media es constante
    \item $C(s,t) = C(s+h, t+h) = C(0, t-s), \; \forall s,t \in T$.
  \end{nlist}
\end{ndef}

\begin{nota}
  Todos procesos de \emph{2º orden} estrictamente estacionarios son débilmente estacionarios.
\end{nota}

\begin{ndef}[Martingala (tiempo discreto)]
  Un p.e es una \emph{martingala} si $\forall n \in \N$ $$E[X_{n+1} | \, X_1, \ldots, X_n] = X_n \; \; c.s$$
\end{ndef}

\begin{ndef}[Procesos de Markov]
  Un p.e es un \emph{Proceso de Markov} si  $\forall s<t \in T, \forall B \in \B$ $$P[X_t \in B | \, X_n, \; n \leq s] = P[X_t \in B | \, X_s] \; \; c.s$$
\end{ndef}

\subsection{Procesos estocásticos en tiempo discreto (PETD): Trayectorias y distribución}

\subsubsection{Definición de PETD}

Consideramos el espacio de probabilidad $(\W, \A, P)$, espacio paramétrico discreto $T \equiv \N$, y espacios de estados $(\R, \B)$.

\begin{ndef}[Proceso estocástico en tiempo discreto]
  Un \emph{p.e en tiempo discreto}(PETD) es una sucesión $\{X_n\}_{n \in \N}$ de v.a definidas en $(\W, \A, P)$ t.q $\forall n \in \N$:

  $$X^{-1}(B) \in \A, \forall B \in \B \iff [X \leq x] \in \A, \forall x \in \R$$
\end{ndef}

\subsubsection{Trayectorias en un PETD}

\begin{ndef}[Trayectoria (PETD)]
  Sea un PETD $\{X_n\}_{n \in \N}$, fijando $\w \in \W$ definimos la \emph{trayectoria} de $\w$ como:
  \begin{align*}
    X(\w): \N & \to \R \\
    n & \mapsto X_n(\w)
  \end{align*}

\end{ndef}

Vemos que $\{X_n(\w)\}_{n\in\N}$ es una sucesión de números reales. Así, definiendo $\R^\N = \{ \{x_n\}_{n \in \N} : x_n \in \R, n \in \N\}$, tenemos que $\{X_n(\w)\}_{n \in \N} \in \R^\N$.

\begin{ndef}[Trayectoria asociada (PETD)]
  Sea un PETD $\{X_n\}_{n \in \N}$, definimos la \emph{trayectoria asociada} como la función que asigna a cada elemento su trayectoria:
  \begin{align*}
    \chi : (\W, \A) & \to (\R^\N, \B^\N) \\
    \w & \mapsto X(\w) = \{X_n(\w)\}_{n \in \N}
  \end{align*}
\end{ndef}

Necesitamos saber como es el espacio Borel $(\R^\N, \B^\N)$, para ello veamos unas definiciones:

\begin{ndef}[Rectángulo]
  Dados $B_1, \ldots, B_n \subset \R$ llamamos un \emph{rectángulo} de lados $B_1, \ldots, B_n$ a $\{ \{a_n\}_{n \in \N} \in \R^\N : \; x_1 \in B_1, \ldots, x_n \in B_n\} = B_1 \times B_2 \times \ldots \times B_n \times \R \times \ldots.$
\end{ndef}

\begin{ndef}[Rectángulo medible]
  Un \emph{rectángulo medible} es un rectángulo donde $B_1, \ldots, B_n \in \B$.
\end{ndef}

\begin{ndef}[Clase de rectángulos medibles]
  Denotamos por $\C^\N$ la \emph{clase de rectángulos medibles}.
\end{ndef}

\begin{ndef}[Semi-álgebra]
  Una \emph{semi-álgebra} sobre $\W$ es un conjunto $\C \in \Part(\W)$ verificando:
  \begin{nlist}
    \item $\W \in \C$
    \item $\forall A, B \in \C \implies A \cap B \in \C$
    \item $\forall A \in \C \implies \overline{A}$ es unión finita disjunta de elementos de $\C$.
  \end{nlist}
\end{ndef}

No hay ningún método constructivo para hallar la $\sigma$-álgebra minimal asociada a un conjunto, pero si tenemos una semi-álgebra solo nos falta comprobar que estén los complementarios, o lo que es lo mismo, que las uniones finitas disjuntas estén.

%TODO mirar esto

\begin{ndef}[$\sigma$-álgebra de rectángulos medibles]
  El \emph{$\sigma$-álgebra de rectángulos medibles} es el $\sigma$-álgebra minimal generada por los rectangulos medibles, es decir, $\sigma(\C^\N)$.
\end{ndef}

\begin{ndef}[$\sigma$-álgebra Borel sobre $\R^\N$]
  La \emph{$\sigma$-álgebra Borel sobre $\R^\N$} es $\B^\N$ que es la $\sigma$-álgebra sobre la clase $\C^\N$, es decir $\B^\N \equiv \sigma(\C^\N)$.
\end{ndef}

\begin{nth}[Teorema de medibilidad (Caracterización de PETD)]
  Sea $(\W, \A, P)$ espacio de probabilidad, $T \equiv \N$, consideramos $\forall n \in \N$, $X_n : (\W, \A) \to (\R, \B)$ y
  \begin{align*}
    \chi : \W & \to \R^\N \\
    \w & \mapsto \{X_n(\w)\}_{n \in \N}
  \end{align*}
  Entonces, $\chi : (\W, \A, P) \to (\R^\N, \B^\N)$ es función medible $\iff \forall n \in \N, X_n$ es v.a $\iff \{X_n\}_{n \in \N}$ es PETD.
\end{nth}

\subsubsection{Distribución de un PETD}

\begin{ndef}[Distribución de un PETD]
  Sea $\{X_n\}_{n \in \N}$ un PETD definido sobre $(\W, \A, P)$, la \emph{distribución del PETD} es una probabilidad definida:
  \begin{align*}
    P_\chi : \B^\N & \to [0, 1] \\
    B & \mapsto P(\chi^{-1}(B)) = P[\chi \in B]
  \end{align*}
\end{ndef}

$P_\chi$ está bien definida porque $\chi^{-1}(B) \in \A$ al ser $\chi$ función medible (T. medibilidad), y es una probabilidad (se puede comprobar facilmente).

Por los Teoremas de extensión podemos conocer la medida en el $\sigma$-álgebra conociendo solo la medida finita en el semi-álgebra, ya que se extiende de forma única a una medida sobre el álgebra minimal sobre la semi-álgebra (sumatoria de los conjuntos); que a su vez se extiende de forma única sobre la $\sigma$-álgebra minimal sobre la álgebra (que desconocemos).

\begin{nth}[Teorema de consistencia de Kolmogorov]
  Sea $\forall n \in \N$, $P_n$ medida en $(\R^n, \B^n)$ verificando la \emph{propiedad de consistencia}: $$ P_n(B_1 \times \ldots \times B_n) = P_{n+1}(B_1 \times \ldots \times B_n \times \R), \; \forall B_1, \ldots, B_n \in \B $$

  Entonces:
  $$\exists! \; \hat{P} : \B^\N \to [0,1], \;\hat{P}(\{(x_n)_{n \in \N} : \; x_1 \in B_1, \ldots, x_n \in B_n \}) = P_n(B_1 \times \ldots \times B_n)$$
\end{nth}

\begin{ncor}
  La distribución de probabilidad $P_\chi$ está determinada por las distribuciones finito dimensionales, $\forall n \in \N$, $dist(X_1, \ldots, X_n) = P_n = P_{(X_1, \ldots, X_n)}(B_1 \times \ldots \times B_n) = P[X_1 \in B_1, \ldots, X_n \in B_n]$.
\end{ncor}

\begin{proof}
  Veamos que $\{P_{(X_1, \ldots, X_n)}\}_{n \in \N}$ es sucesión de distribuciones de probabilidad que verifican la propiedad de consistencia.

  Se verifica ya que $P_{(X_1, \ldots, X_n)}(B_1 \times \ldots \times B_n) = P[X_1 \in B_1, \ldots, X_n \in B_n] = P[X_1 \in B_1, \ldots, X_n \in B_n, X_{n+1} \in \R] = P_{(X_1, \ldots, X_n, X_{n+1})}(B_1 \times \ldots \times B_n \times R)$.

  Ahora veamos que $P_\chi \equiv \hat{P}$.

  Sea $S \in \C^\N$, $P_\chi(S) = P(\chi^{-1}(S)) = P\{w \in \W : \; \chi(\w) \in S\} = P \{\w \in \W : \; \{X_n(w)\}_{n \in \N} \in S\} = P[X_1 \in B, \ldots, X_n \in B_n] = P_{(X_1, \ldots, X_n)}(B_1 \times \ldots \times B_n) = \hat{P}(S)$
\end{proof}

\subsection{Procesos estocásticos en tiempo continuo (PETC): Trayectorias y distribución}

\subsubsection{Definición de PETC}

\begin{ndef}[Proceso estocástico en tiempo continuo]
  Sea espacio de probabilidad $(\W, \A, P)$ y un intervalo $T \subseteq R$, un \emph{proceso estocástico en tiempo continuo}(PETC) $\{X_t\}_{t \in T}$ es una familia de v.a definidas sobre $(\W, \A, P)$, es decir:

  \begin{equation}
    \forall t \in T, \; X_t : (\W, \A, P) \to (\R, \B) \text{ es medible }
    \iff X^{-1}_t(B) \in \A, \; \forall B \in \B \iff X_t^{-1}((-\infty, x]) \in \A, \; \forall x \in \R
  \end{equation}
\end{ndef}

\subsubsection{Trayectorias}

\begin{ndef}[Trayectoria(PETC)]
  Sea $\w \in \W$, definimos la \emph{trayectoria} de $\w$ como
  \begin{align*}
    X(\w) : T & \to \R \\
    t & \mapsto X_t(\w)
  \end{align*}
\end{ndef}

Tenemos que ver ahora si la función que asocia elementos a sus trayectorias es medible:
\begin{align*}
  \chi : \W & \to \R^T \\
  \w & \mapsto \{ X_t(\w) \}_{t \in T}
\end{align*}

\begin{ndef}[$\R^T$]
  Definimos \emph{$\R^T$} como $\R^T = \{ f : T \to \R \}$ (trayectorias).
\end{ndef}

Necesitamos un $\sigma$-álgebra con $\R^T$, veamos como la definimos:

\begin{ndef}[Rectángulo en $\R^T$]
  Definimos el \emph{rectángulo de lados $B_1, \ldots, B_n$} con $B_i \subseteq \R$ como $R = \{ f \in \R^T : f(t_1) \in B_1, \ldots, f(t_n) \in B_n, \; \forall t_1, \ldots, t_n \in T \}$.
\end{ndef}


\begin{ndef}[$\sigma$-álgebra Borel $\B^T$]
  Definimos \emph{la $\sigma$-álgebra Borel $\B^T$} como el $\sigma$-álgebra minimal sobre el álgebra de rectángulo medibles, es decir, $\B^T = \sigma(\text{Álg. de rect. medibles}) = \sigma(\C^T)$.
\end{ndef}

\begin{ndef}[Teorema de medibilidad(PETC)]
  Sea un espacio de probabilidad $(\W, \A, P)$, un intervalo $T \subseteq \R$, $\forall t \in T$, las funciones $X_t : \W \to \R$, y la función
  \begin{align*}
    \chi : \W & \to \R^T \\
    \w & \mapsto \{X_t(\w)\}_{t \in T}
  \end{align*}.
  Entonces $\chi : (\W, \A, P) \to (\R^T, \B^T)$ es medible $\iff \forall t \in T, \; X_t : (\W, \A, P) \to (\R, \B)$ es v.a $\iff \{X_t\}_{t \in T}$ es un PETC.
\end{ndef}

\subsubsection{Distribución de un PETC}

\begin{ndef}[Distribución de un PETD]
  La \emph{distribución de un PETD} $\{X_t\}_{t \in T}$ definido sobre el espacio de probabilidad $(\W, \A, P)$ es una probabilidad definida como:
  \begin{align*}
    P_\chi : \B^T & \to [0, 1] \\
    B & \mapsto P_\chi(B) = P(\chi^{-1}(B)) = P[\chi \in \B]
  \end{align*}
\end{ndef}

Está bien definida por ser $\chi$ medible, y es una probabilidad (comprobable fácilmente).


% TODO: demostrar que $\C^T$ es semiálgebra
% TODO: demostrar Teoremas de medibilidad
