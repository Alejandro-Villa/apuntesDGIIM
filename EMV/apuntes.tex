\section{Fundamentación probabilística de vectores aleatorios}

\begin{ndef}[Variable Aleatoria]
    % TODO Buscar la F concreta
    Sea $(\Omega, F, P)$ un espacio probabilístico, una variable aleatoria es una función medible
    \[
    X:(\Omega, F) \rightarrow (\mathbb{R}, \mathbb{B}) \\
    \omega \rightarrow X(\omega)
    .\]
    La condición de medibilidad es:
    \[
    \forall B \in \mathbb{B}, X^{-1}(B) \in F
    .\]
\end{ndef}

\begin{ndef}[Probabilidad inducida]
    La probabilidad inducida por una variable aleatoria $X$ a partir de $P$ se define como:
    \[
    P_{X}:\mathbb{B} \rightarrow [0,1] \\
    P_{X}[B]:= P[X^{-1}(B)]
    .\]
\end{ndef}

\begin{ndef}[Vector Aleatorio]
    Un vector aleatorio, $X = (X_1, \dots, X_p)$ es una función medible
    \[
    X:(\Omega, F) \rightarrow (\mathbb{R}^p, \mathbb{B}^p) \\
    \omega \rightarrow X(\omega) = (X_1(\omega), \dots, X_p(\omega))
    .\]
    La condición de medibilidad es:
    \[
    \forall B \in \mathbb{B}^p, X^{-1}(B) \in F
    .\]
\end{ndef}

\begin{ndef}[Probabilidad inducida]
    La probabilidad inducida por un vector aleatorio $X = (X_1, \dots, X_p)$ a partir de $P$ se define como:
    \[
    \forall B \in \mathbb{B}^p, P_X[B] := P[X^{-1}(B)]
    .\]
\end{ndef}

\begin{ndef}[Función de distribución]
    Se define la función de distribución asociada a $P_X$ como:
    \[
    F_X:\mathbb{R}^p \rightarrow [0,1] \\
    F_X(x) = P_X[X_1 \leq x_1, \dots, X_p \leq x_p]
    .\]
\end{ndef}

\begin{ndef}[Función de densidad]
    Si existe una función $f_X$, integrable (en el sentido de Lebesgue) tal que
    \[
    F_X(x) = \int^{x_1}_{-\infty} \dots \int^{x_p}_{-\infty} f_X(u_1, \dots,  u_p) du_1 \dots du_p \forall x \in \mathbb{R}^p
    ,\]
    diremos que $f_X$ es la función de densidad asociada a  $F_X$. En el caso en que $f_X$ sea continua, se puede escribir como:
    \[
    % TODO Añadir lo de las derivadas
    f_X(x) = \dots \forall X \in \mathbb{R}^p
    .\]
\end{ndef}

En general, dado un conjunto producto (también llamado conjunto rectangular) $B \in \mathbb{B}^p$ con
\[
    B = B_1 \times \dots \times B_p, B_j \in \mathbb{B} j = 1, \dots, p
.\]
se tiene que
\[
P_X[B] \neq P_{X_1}[B_1]\cdot \dots \cdot P_X_p[B_p]
.\]

\begin{Independencia}
    En las condiciones anteriores, si se da la igualdad para todo conjunto producto en $\mathbb{B}^p$ se dice que $X_1, \dots, X_p$ son independientes. \\

    Equivalentemente, esto ocurre si y solo si
    \[
    F_X(x) = F_X_1 (x_1)\cdot \dots \cdot F_X_p(x_p) \forall x \in \mathbb{R}^p
    .\]

    Por último, en el caso continuo si y solo si
    \[
    f_X(x) = f_X_1(x_1) \cdot \dots \cdot f_X_p(x_p) \forall x \in \mathbb{R}^p
    .\]
\end{Independencia}
