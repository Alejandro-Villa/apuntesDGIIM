  \documentclass[10pt,a4paper,landscape]{article}
\usepackage{multicol}
\usepackage{calc}
\usepackage{ifthen}
\usepackage[landscape]{geometry}
\usepackage[hidelinks]{hyperref}

\usepackage[utf8]{inputenc}

% Fuentes (Fira Sans)
\usepackage[T1]{fontenc}
\usepackage[sfdefault,scaled=.85]{FiraSans}
\usepackage{newtxsf}
\usepackage[scaled=.85]{FiraMono}


% Listings for code

\usepackage{listings}
\lstset{basicstyle=\ttfamily\footnotesize,breaklines=true}

% Multiple cols/rows

\usepackage{multirow}

% Checkmarks

\usepackage{amssymb}


% To make this come out properly in landscape mode, do one of the following
% 1.
% pdflatex latexsheet.tex
% 
% 2.
% latex latexsheet.tex
% dvips -P pdf  -t landscape latexsheet.dvi
% ps2pdf latexsheet.ps


% If you're reading this, be prepared for confusion.  Making this was
% a learning experience for me, and it shows.  Much of the placement
% was hacked in; if you make it better, let me know...


% 2008-04
% Changed page margin code to use the geometry package. Also added code for
% conditional page margins, depending on paper size. Thanks to Uwe Ziegenhagen
% for the suggestions.

% 2006-08
% Made changes based on suggestions from Gene Cooperman. <gene at ccs.neu.edu>


% To Do:
% \listoffigures \listoftables
% \setcounter{secnumdepth}{0}


% This sets page margins to .5 inch if using letter paper, and to 1cm
% if using A4 paper. (This probably isn't strictly necessary.)
% If using another size paper, use default 1cm margins.
\ifthenelse{\lengthtest { \paperwidth = 11in}}
{ \geometry{top=.5in,left=.5in,right=.5in,bottom=.5in} }
{\ifthenelse{ \lengthtest{ \paperwidth = 297mm}}
  {\geometry{top=1cm,left=1cm,right=1cm,bottom=1cm} }
  {\geometry{top=1cm,left=1cm,right=1cm,bottom=1cm} }
}

% Turn off header and footer
\pagestyle{empty}


% Redefine section commands to use less space
\makeatletter
\renewcommand{\section}{\@startsection{section}{1}{0mm}%
  {-1ex plus -.5ex minus -.2ex}%
  {0.5ex plus .2ex}%x
  {\normalfont\large\bfseries}}
\renewcommand{\subsection}{\@startsection{subsection}{2}{0mm}%
  {-1explus -.5ex minus -.2ex}%
  {0.5ex plus .2ex}%
  {\normalfont\normalsize\bfseries}}
\renewcommand{\subsubsection}{\@startsection{subsubsection}{3}{0mm}%
  {-1ex plus -.5ex minus -.2ex}%
  {1ex plus .2ex}%
  {\normalfont\small\bfseries}}
\makeatother

% Define BibTeX command
\def\BibTeX{{\rm B\kern-.05em{\sc i\kern-.025em b}\kern-.08em
    T\kern-.1667em\lower.7ex\hbox{E}\kern-.125emX}}

% Don't print section numbers
\setcounter{secnumdepth}{0}


\setlength{\parindent}{0pt}
\setlength{\parskip}{0pt plus 0.5ex}


% -----------------------------------------------------------------------

\begin{document}

\raggedright
\footnotesize
\begin{multicols}{3}


  % multicol parameters
  % These lengths are set only within the two main columns
  % \setlength{\columnseprule}{0.25pt}
  \setlength{\premulticols}{1pt}
  \setlength{\postmulticols}{1pt}
  \setlength{\multicolsep}{1pt}
  \setlength{\columnsep}{2pt}



  \begin{center}
    \Large{\textbf{MM temas 1 y 2}} \\
  \end{center}

  \section{Ecuaciones en diferencias de primer orden}

  Una ecuación en diferencias de primer orden lineal es de la forma

  $$x_{n+1} = \alpha x_n + \beta \ \ \ \alpha, \beta \in \mathbb C$$

  \subsection{Resolución de ecuaciones en diferencias de primer orden}

  \subsubsection{$\beta = 0$}

  Al ser una progresión geométrica la solución será $x_n = C\alpha^n$.

  \subsubsection{$\beta \neq 0$ y $\alpha = 1$}

  Nos encontramos ante una progresión aritmética, así que la solución es $x_n =
  C+\beta n$

  \subsubsection{$\beta \neq 0$ y $\alpha \neq 1$}

  Buscamos primero lo que llamamos \textit{solución constante}, la solución que
  tendría la ecuación si no dependiese de $n$.

  $$x_* = \frac{\beta}{1-\alpha}$$

  A continuación escribimos la ecuación homogénea asociada,

  $$z_{n+1} = \alpha z_n$$

  cuya solución sería $z_n = C\alpha^n$ como ya hemos visto antes. Finalmente, la
  solución de la ecuación inicial será

  $$x_n = x_* + z_n = \frac{\beta}{1-\alpha} + C\alpha^n $$

  \subsection{Fórmula de De Moivre}

  Si $\alpha$ es un número complejo, $$\alpha^n = r^n(cos(n\theta)+isen(n\theta))$$

  \subsection{Comportamiento asintótico de las soluciones}

  Dadas las soluciones $\{x_n\}_{n \geq 0}$ de una ecuación en diferencias de primer
  orden,

  \begin{itemize}
  \item Si $|\alpha| < 1$, $x_n \rightarrow x_*$.
  \item Si $|\alpha| > 1$, $x_n$ diverge.
  \item Si $|\alpha| = 1$, $x_n$ oscila alrededor de $x_*$.
  \end{itemize}

  \section{Sistemas dinámicos discretos}

  \subsection{Puntos de equilibrio}

  Un número $\alpha$ se dice que es punto de equilibrio del SDD $\{I, f\}$ si
  $\alpha = f(\alpha),\alpha \in I$.

  Para hallar los puntos de equilibrio simplemente resolvemos la ecuación
  obtenida de la igualdad $\alpha = f(\alpha)$ y comprobamos si las soluciones
  pertenecen a $I$. Hay SDD que no tienen puntos de equilibrio, pero todo aquel
  en el que $I$ sea cerrado y acotado tiene alguno.

  \subsection{Estabilidad asintótica}

  Si $\alpha$ es un punto de equilibrio de un SDD $\{I, f\}$ y $f \in C^1(I)$,
  entonces:
  \begin{itemize}
  \item Si $|f^{\prime}(\alpha)| < 1$ entonces $\alpha$ es localmente
    asintóticamente estable.
  \item Si $|f^{\prime}(\alpha)| > 1$ entonces $\alpha$ es inestable.
  \end{itemize}

  Si $f \in C^3(I)$ y $f^{\prime}(\alpha) = 1$ entonces:
  \begin{itemize}
  \item Si $f^{\prime \prime}(\alpha) \neq 0$ entonces $\alpha$ es inestable.
  \item Si $f^{\prime \prime}(\alpha) = f^{\prime \prime \prime}(\alpha) < 0$
    entonces $\alpha$ es localmente asintóticamente estable.
  \item Si $f^{\prime \prime}(\alpha) = 0$ y $f^{\prime \prime \prime}(\alpha) >
    0$ entonces $\alpha$ es inestable.
  \end{itemize}

  \subsection{Ciclos}

  Un ciclo de orden $s$ o una órbita periódica de periodo $s$ o un
  \textit{s-ciclo} del SDD $\{I, f\}$ es un conjunto de puntos $\{\alpha_0,
  \alpha_1, \hdots, \alpha_{s-1}\} \subset I$ distintos entre sí, verificando

  $$\alpha_1 = f(\alpha_0), \alpha_2 = f(\alpha_1), \hdots, \alpha_{s-1} =
  f(\alpha_{s-2}), \alpha_0 = f(\alpha_{s-1})$$

  A \textit{s} se le llama \textit{periodo de la órbita} u \textit{orden
    del ciclo}.

  \subsubsection{Estabilidad de los ciclos}

  Supongamos $f: I \rightarrow I, f \in C^1(I)$ y que $\{\alpha_0, \alpha_1,
  \hdots, \alpha_{s-1}\}$ es un \textit{s-ciclo} para el SDD
  $\{I,f\}$. Entonces:
  \begin{itemize}
  \item Si $|f^{\prime}(\alpha_0)f^{\prime}(\alpha_1) \hdots
    f^{\prime}(\alpha_{s-1})| < 1$ el ciclo es asintóticamente estable.
  \item Si $|f^{\prime}(\alpha_0)f^{\prime}(\alpha_1) \hdots
    f^{\prime}(\alpha_{s-1})| > 1$ el ciclo es inestable.
  \end{itemize}

  \section{Ecuaciones en diferencias lineales de orden superior homogéneas}

  Dada la ecuación en diferencias lineal homogénea de orden k $$x_{n+k} +
  a_{k-1}x_{n+k-1}+ \hdots + a_{1}x_{n+1} + a_{0}x_{n} = 0 \ \ n \geq 0$$

  Llamaremos \textit{polinomio característico al polinomio}: $$p(\lambda) =
  \lambda^k + a_{k-1}\lambda^{k-1}+ \hdots + a_1\lambda + a_0 $$ Sus raíces las
  llamamos \textit{raíces características}.

  \subsection{Solución de las ecuaciones lineales en diferencias de orden superior homogéneas}

  Distinguiremos distintos casos según las raíces del \textit{polinomio
    característico}.
  
  \subsubsection{$k$ raíces distintas}

  La solución general vendrá dada por

  $$x_n = c_1\lambda_1^n+c_2\lambda_2^n + \hdots + c_k\lambda_k^n, \ \ c_1,
  c_2, \hdots c_k \in \mathbb K$$

  \subsubsection{Raíces complejas}

  Si el polinomio $p(\lambda)$ tiene una raíz compleja $\lambda_*$ entonces
  $\overline{\lambda_*}$ también es raíz. Si $r$ es el módulo y $\theta$ es el argumento de
  $\lambda_*$ (y de $\overline{\lambda_*}$)  entonces en la solución general
  escribiremos en su lugar $r^n cos n\theta$ y $r^nsen n\theta$.

  \subsubsection{Raíces múltiples}

  Supongamos que el polinomio $p(\lambda)$ tiene $r$ raíces características
  $\lambda_1, \lambda_2, \hdots, \lambda_r$ de multiplicidades $m_1, m_2,
  \hdots, m_r$ respectivamente, siendo la suma de las multiplicidades el grado
  del polinomio. Entonces la solución general será de la forma

  $$x_n = \sum_{i=1}^{r}\lambda_i^n \left(c_{i0}+a_{i1}n + a_{i2}n^2 + \hdots +
  a_{i,m_i-1}n^{m_i -1}\right) $$

  \subsection{Comportamiento asintótico de las soluciones}

  Sean $\lambda_1,\lambda_2, \hdots, \lambda_s$ las raíces de $p(\lambda)$. Son
  equivalentes:
  \begin{itemize}
  \item Todas las soluciones de la ecuación lineal en diferencias homogénea
    verifican $lim_{n \to \infty} x_n = 0$.
  \item Las raíces verifican $máx_{i=1, \hdots, s} |\lambda_i| < 1$.
  \end{itemize}

  \subsubsection{El caso $k=2$}

  En el caso $k=2$ las raíces $\lambda_1, \lambda_2$ del polinomio $p(\lambda) =
  \lambda^2 + a_1\lambda + a_0$ verifican $|\lambda_i| < 1$ para $i = 1,2$ si y
  solo si:
  $$
  \begin{cases}
    p(1) = 1 + a_1 + a_0 > 0 \\
    p(-1) = 1 - a_1 + a_0 > 0 \\
    p(0) = a_0 < 1 \\
  \end{cases}
  $$

  \section{Ecuaciones en diferencias lineales de orden superior completas}

  Para resolver una ecuación de la forma $$x_{n+k} +
  a_{k-1}x_{n+k-1}+ \hdots + a_{1}x_{n+1} + a_{0}x_{n} = b(n) \ \ n \geq 0$$
  seguimos los siguientes pasos:

  \begin{enumerate}
  \item Buscamos una solución de la ecuación en diferencias \textbf{homogéna}
    asociada.
  \item  Buscamos una solución particular de la ecuación dada. Dividimos el
    término $b(n)$ en sumandos para aplicar el \textit{principio de
      superposición}. En cada caso buscaremos una solución particular del mismo
    carácter que el término independiente, atendiendo a la siguiente tabla
    \footnote{Elaydi, An Introduction to Difference Equations p. 85}:
    \begin{tabular}{@{}ll@{}}
      $b(n)$    & $x_n^p$ \\ \hline
      $a^n$ & $c_1a^n$ \\
      $n^k$ & $c_0+c_1n + \hdots + c_kn^k$ \\
      $n^ka^n$ & $c_0a^n + c_1na^n + \hdots + c_kn^ka^n$ \\
      $sen bn, \ cos bn$ & $c_1 sin bn + c_2 cos bn$ \\
      $a^nsen bn, \ a^n cos bn$ & $(c_1 sen bn + c_2 cos bn)a^n$ \\
      $a^nn^k sen bn, a^nn^k cos bn$ & $(c_0 + c_1n + \hdots + c_kn^k)a^n
                                       sen(bn)$ \\
      \ & \ \ \ \ $+ (d_0 + d_1n + \hdots d_kn^k)a^n cos(bn)$
    \end{tabular}
  \item Por último, la solución final es la suma de la solución de la ecuación
    homogénea más la solución particular de la completa (si teníamos varias, su suma).
  \end{enumerate}
  

\end{multicols}
\end{document}
