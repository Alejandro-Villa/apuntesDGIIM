\documentclass{article}
\usepackage[left=3cm,right=3cm,top=2cm,bottom=2cm]{geometry}

\renewcommand{\familydefault}{\sfdefault}

\usepackage{amsmath,amsthm,mathtools}
\usepackage{amsfonts,amssymb,latexsym}

\usepackage[spanish]{babel} \usepackage[doument]{ragged2e}

\newcommand{\closure}[1]{\mkern
  1.5mu\overline{\mkern-1.5mu#1\mkern-1.5mu}\mkern 1.5mu}
\newcommand{\interior}[1]{{\kern0pt#1}^{\mathrm{o}}}

\selectlanguage{spanish} \usepackage[utf8]{inputenc}
\setlength{\parindent}{10mm}

\theoremstyle{remark,bold} \newtheorem{exercise}{Ejercicio}
\newtheorem*{solution}{Solución}

\usepackage{upgreek}

\begin{document}

\title{Exámen parcial de Topología I \\ Curso 16/17}
\date{}
%\author{David Cabezas Berrido}
\maketitle

% Ejercicio 1
\begin{exercise} (4 puntos) Un espacio topológico $(X, \mathcal{T})$
  se llama Hausdorff si para cualquier par de puntos $x, y \in X$ con $x
  \neq y$, existen entornos $V \in \mathcal{U}^x$ y $W \in
  \mathcal{U}^y$ cumpliendo $V \cap W = \emptyset$.

  \begin{enumerate}
    
  \item Probar que si $(X, \mathcal{T})$ es un espacio topológico
    Hausdorff, entonces cualquier punto suyo es un subconjunto cerrado.
    
  \item Probar que cualquier subespacio topológico de un espacio
    Hausdorff es Hausdorff.
    
  \item Probar que el producto topológico de dos espacios Hausdorff es
    Hausdorff.
    
  \item Probar que un espacio topológico $(X, \mathcal{T})$ es
    Hausdorff si y sólo si la diagonal
    \[\Delta = \{(x,x) \in X \times X \text{ }| \text{ } x \in X\}\]
    es un subconjunto cerrado de $(X \times X, \mathcal{T} \times
    \mathcal{T})$.
    
  \item Sean $f,g : (X,\mathcal{T}) \longrightarrow (X',
    \mathcal{T}')$ funciones continuas y supongamos que $(X',
    \mathcal{T}')$ es un espacio Hausdorff. Probar que
    \[\{x \in X \text{ } | \text { } f(x) = g (x)\}\] es un
    subconjunto cerrado de X.
    
  \item Si $(X, d)$ es un espacio métrico, probar que $(X,
    \mathcal{T}(d))$ es un espacio Hausdorff.
    
  \end{enumerate}
  
\end{exercise}

% Solución 1
\begin{solution}
  
  \text{ }
  
  \begin{enumerate}

  \item Elegimos cualquier punto $x \in X$, debemos probar que el
    conjunto $\{x\}$ es cerrado en $X$, para ello elegimos un punto
    cualquiera del complementario, $y \in X \backslash \{x\}$, y probamos
    que existe un entorno de $y$ contenido en el conjunto.

    Como $X$ es Hausdorff, existen $V \in \mathcal{U}^x$ y $W \in
    \mathcal{U}^y$ cumpliendo $V \cap W = \emptyset$. Es claro que $x \in
    V$, luego como la intersección es vacía tiene que darse $x \notin
    W$. Esto implica $W \subseteq X \backslash \{x\}$ y por tanto $X
    \backslash \{x\}$ es abierto y $\{x\}$ cerrado como queríamos.

  \item Tomamos $A \subseteq X$, si $A = \emptyset$ o tiene un sólo
    punto, satisface trivialmente la condición. De lo contrario, tomamos
    $x, y \in A$, con $x \neq y$, tenemos $V \in \mathcal{U}^x$ y $W \in
    \mathcal{U}^y$ con $V \cap W = \emptyset$. Por ser $V$ y $W$ entornos,
    existen $\mathcal{O}_x,\mathcal{O}_y \in \mathcal{T}$ cumpliendo $x
    \in \mathcal{O}_x \subseteq V$, $y \in \mathcal{O}_y \subseteq W$. En
    la topología inducida $\mathcal{T}_A$, $\mathcal{O}_x \cap A$ y
    $\mathcal{O}_y \cap A$ son abiertos y por tanto entornos de $x$ e $y$
    respectivamente. Además $\mathcal{O}_x \cap \mathcal{O}_y = \emptyset$
    y por tanto $(\mathcal{O}_x \cap A) \cap (\mathcal{O}_y \cap A) =
    \emptyset$. Luego el subespacio $(A, \mathcal{T})$ es Hausdorff.

  \item Sean $(X, \mathcal{T})$ e $(Y, \mathcal{T}')$ espacios
    topológicos Hausdorff. Tomamos $p, q \in (X \times Y, \mathcal{T}
    \times \mathcal{T}')$ distintos, donde $p = (x_p,y_p)$ y $q =
    (x_q,y_q)$. Como $p \neq q$, ambas coordenadas no pueden coincidir,
    podemos suponer que $x_p \neq x_q$ (si $x_p = x_q$ se tendrá $y_p \neq
    y_q$ y la demostración es análoga). Como $(X, \mathcal{T})$ es
    Hausdorff, existen $V \in \mathcal{U}^{x_p}$ y $W \in
    \mathcal{U}^{x_q}$ con $V \cap W = \emptyset$, y por ser $V$ y $W$
    entornos, existen $\mathcal{O}_{x_p},\mathcal{O}_{x_q} \in
    \mathcal{T}$ con $x_p \in \mathcal{O}_{x_p} \subseteq V$, $x_q \in
    \mathcal{O}_{x_q} \subseteq W$. Ahora probemos que $V' =
    \mathcal{O}_{x_p} \times Y$ y $W' = \mathcal{O}_{x_q} \times Y$
    satisfacen la condición. Es claro que son abiertos (producto de
    abiertos), por lo que $V' \in \mathcal{U}^p$, $W' \in
    \mathcal{U}^q$. Ahora tomemos $z = (x_z,y_z) \in V' \cap W'$, entonces
    $x_z \in \mathcal{O}_{x_p} \cap \mathcal{O}_{x_q}$, pero esta
    intersección es vacía, luego debe darse $V' \cap W' = \emptyset$. Por
    tanto concluimos que $(X \times Y, \mathcal{T} \times \mathcal{T}')$
    es Hausdorff.

  \item $(\Longrightarrow)$ Supongamos que $(X, \mathcal{T})$ es
    Hausdorff, vamos a probar que
    \[(X \times X) \backslash \Delta = \{(x,y) \in X \times X \text{ }
      | \text{ } x \neq y \}\] es abierto en $(X, \mathcal{T})$. Para ello
    tomamos un punto $(x,y) \in (X \times X) \backslash \Delta$, por lo
    que tendremos $x, y \in X$ con $x \neq y$. Como $(X, \mathcal{T})$ es
    Hausdorff, existen $V \in \mathcal{U}^x$ y $W \in \mathcal{U}^y$ con
    $V \cap W = \emptyset$ y por tanto $\mathcal{O}_x,\mathcal{O}_y \in
    \mathcal{T}$ cumpliendo $x \in \mathcal{O}_x \subseteq V$, $y \in
    \mathcal{O}_y \subseteq W$. $\mathcal{O}_x \times \mathcal{O}_y$ es
    entorno de $(x,y)$ y está contenido en $(X \times X) \backslash
    \Delta$, ya que si existiera un punto en $(\mathcal{O}_x \times
    \mathcal{O}_y) \cap \Delta$ sería de la forma $(z,z)$ y tendríamos $z
    \in \mathcal{O}_x \cap \mathcal{O}_y$, que es imposible porque esta
    intersección es vacía. Por tanto $(X \times X) \backslash \Delta$ es
    abierto y $\Delta$ cerrado.

    $(\Longleftarrow)$ Supongamos que $(X \times X) \backslash \Delta$
    es abierto, debemos probar que $(X, \mathcal{T})$ es Hausdorff.

    Tomamos $x, y \in X$ cumpliendo $x \neq y$, por lo que $(x, y) \in
    (X \times X) \backslash \Delta$. Como es abierto, existe un abierto
    básico $\mathcal{O} \times \mathcal{O}'$ (con $\mathcal{O},
    \mathcal{O}' \in \mathcal{T}$) cumpliendo $(x,y) \in \mathcal{O}
    \times \mathcal{O}' \subseteq (X \times X) \backslash \Delta$.
    $\mathcal{O}$ y $\mathcal{O}'$ son entornos de $x$ e $y$
    respectivamente, probemos que son disjuntos. Si $\exists z \in
    \mathcal{O} \cap \mathcal{O}'$, el punto $(z,z) \in \mathcal{O} \times
    \mathcal{O}' \cap \Delta$, lo cual es imposible porque $\mathcal{O}
    \times \mathcal{O}' \subseteq (X \times X) \backslash \Delta$. Por
    tanto tiene que darse $\mathcal{O} \cap \mathcal{O}' = \emptyset$,
    luego $(X, \mathcal{T})$ es Hausdorff.

  \item Llamemos $A$ al conjunto $\{x \in X \text{ } | \text{ } f(x) =
    g(x) \}$, definamos $h : (X, \mathcal{T}) \longrightarrow (X' \times
    X', \mathcal{T}' \times \mathcal{T}')$, $h(x) = (f(x), g(x))$. Como
    $f$ y $g$ son continuas, $h$ lo es, además sabemos que la diagonal \\
    (llamémosla $\Delta'$) de $(X', \mathcal{T}')$ es cerrada en $(X'
    \times X', \mathcal{T}' \times \mathcal{T}')$ por ser $(X',
    \mathcal{T}')$ Hausdorff. Usamos que la imagen inversa por una función
    continua de un cerrado es cerrada para probar lo que queremos.
    \[h^{-1}(\Delta') = \{x \in X \text{ } | \text{ } h(x) \in
      \Delta'\} = \{x \in X \text{ } | \text{ } (f(x), g(x)) \in \Delta' \}
      = \{x \in X \text{ } | \text{ } f(x) = g(x) \} = A\] Como $\Delta'$ es
    cerrado, $A'$ lo es.

  \item Tomemos $x, y \in X$ con $x \neq y$, sea $r = \frac{d(x,
      y)}{2} > 0$. Tomamos $V = B(x, r)$ y $W = B(y, r)$, es claro que $V
    \in \mathcal{U}^x$ y $W \in \mathcal{U}^y$, falta probar que son
    disjuntos. Para ello suponemos $\exists z \in V \cap W$, entonces se
    tiene que $d(x, z) < r$ y $d(y, z) < r$, pero esto es imposible, ya
    que se daría \[d(x, y) \leqslant d(x, z) + d(z, y) < r + r = d(x, y)\]
    lo cual es absurdo. Por tanto $V \cap W = \emptyset$ y $(X,
    \mathcal{T}(d))$ es Hausdorff.
    
  \end{enumerate}

\end{solution}

% Ejercicio 2
\begin{exercise} (3 puntos) Sea $X$ un conjunto y $A, B$ subconjuntos
  no vacíos de $X$ cumpliendo $A \subseteq B$. Se define
  \[\mathcal{T} = \{\mathcal{O} \subseteq X \text{ } | \text{ }
    \mathcal{O} \subseteq X \backslash A\} \cup \{\mathcal{O} \subseteq X
    \text{ } | \text{ } B \subseteq \mathcal{O}\}.\]
  
  \begin{enumerate}
    
  \item Probar que $\mathcal{T}$ es una topología en $X$.
    
  \item Calcular el interior, la adherencia y la frontera de los
    subconjuntos A y B.
    
  \item Hallar una base de la topología $\mathcal{T}$.

  \item Hallar una base de entornos y el sistema de entornos de un
    punto $x \in A$.
    
  \end{enumerate}
  
\end{exercise}

% Solución 2
\begin{solution}

  Los abiertos de $(X, \mathcal{T})$ son los subconjuntos de $X$ que
  no intersecan con $A$ y los que contienen a $B$.

  \begin{enumerate}
    
  \item El conjunto vacío es abierto en la topología porque está
    incluido en $X \backslash A$, y $X$ es abierto porque continene a $B$.

    Sea $\{\mathcal{O}_\lambda\}_{\lambda \in \Lambda}$ una familia de
    abiertos de la topología. Probemos que $\bigcup\limits_{\lambda \in
      \Lambda}\mathcal{O}_\lambda$ también es abierto, si $\exists \lambda_0
    \in \Lambda$ cumpliendo $B \subseteq \mathcal{O}_{\lambda_0}$,
    entonces
    \[B \subseteq \mathcal{O}_{\lambda_0} \subseteq
      \bigcup\limits_{\lambda \in \Lambda}\mathcal{O}_\lambda\] por lo
    que la unión es abierta. De lo contrario, debe de darse
    $\mathcal{O}_\lambda \subseteq X \backslash A \text{ } \forall
    \lambda \in \Lambda$, y por tanto
    $\bigcup\limits_{\lambda \in \Lambda}\mathcal{O}_\lambda \subseteq
    X \backslash A$, luego la unión es abierta en la topología.

    Ahora consideremos $\{\mathcal{O}_i\}_{i=1}^n$ y probemos que
    $\bigcap\limits_{i=1}^n\mathcal{O}_i$ también es abierto, si $\exists
    k \in \{1, \ldots, n\}$ cumpliendo $\mathcal{O}_k \subseteq X
    \backslash A$, entonces
    \[\bigcap\limits_{i=1}^n\mathcal{O}_i \subseteq \mathcal{O}_k
      \subseteq X \backslash A \vspace{2mm}\] por lo que la intersección es
    abierta. De lo contrario, debe de darse $B \subseteq \mathcal{O}_i
    \text{ } \forall i \in \{1, \ldots, n\}$, y por tanto $B \subseteq
    \bigcap\limits_{i=1}^n\mathcal{O}_i$, luego la intersección pertenece
    a la topología. \vspace{2mm}

  \item Si $A = B$, $B \subseteq A$, entonces $A$ es abierto en la
    topología y se tiene $\interior{A} = A$. \\ Si $A$ está contenido
    propiamente en $B$, entonces $A$ no es abierto, porque obviamente no
    está contenido en $X \backslash A$, y cualquier subconjunto suyo no
    vacío tampoco lo estará, por lo que tenemos $\interior{A} =
    \emptyset$.
    
    $X \backslash A \subseteq X \backslash A$, por lo que $X
    \backslash A$ es abierto, luego $A$ es cerrado y se tiene $\closure{A}
    = A$.
    
    Si $A = B$, $A$ es abierto y cerrado a la vez, luego su frontera
    es vacía.
    $$Fr(A) = \closure{A} \backslash \interior{A} = A \backslash A =
    \emptyset$$

    De lo contrario su frontera es el propio $A$ \vspace{-4mm}

    $$Fr(A) = \closure{A} \backslash \interior{A} = A \backslash \emptyset =
    A$$

    $B \subseteq B$, por lo que $B$ es abierto en la topología y se
    tiene $\interior{B} = B$.
    
    Como $A \subseteq B$, se tiene $X \backslash B \subseteq X
    \backslash A$, por lo que el complementario de $B$ es abierto y se da
    $\closure{B} = B$.

    $B$ es abierto y cerrado a la vez, por lo que su frontera es vacía
    \vspace{-4mm}
    
    $$Fr(B) = \closure{B} \backslash \interior{B} = B \backslash B =
    \emptyset$$

  \item Para hallar una base, podemos unir un conjunto de abiertos
    cuyas uniones generen todos los abiertos $\mathcal{O}$ cumpliendo
    $\mathcal{O} \subseteq X \backslash A$ y otro cuyas uniones generen
    los abiertos $\mathcal{O}$ cumpliendo $B \subseteq
    \mathcal{O}$. Consideremos las siguientes familias de abiertos
    \[\mathcal{B}_1 = \big\{\{x\} \text{ } | \text{ } x \in X
      \backslash A\big\}\]
    \[\mathcal{B}_2 = B \cup \big\{B \cup \{x\} \text{ } | \text{ } x
      \in X \backslash B\big\} \]

    Sea $\mathcal{\mathcal{O}}$ un abierto cualquiera de la topología,
    si $\mathcal{\mathcal{O}} \subseteq X \backslash A$ entonces
    \[\mathcal{\mathcal{O}} = \bigcup\limits_{x \in
        \mathcal{\mathcal{O}}}\{x\} \hspace{2mm} \text{donde} \hspace{2mm}
      \{x\} \in \mathcal{B}_1 \hspace{4mm} \forall x \in
      \mathcal{\mathcal{O}} \vspace{-1mm}\] De lo contrario, se tendrá $B
    \subseteq \mathcal{\mathcal{O}}$ y por tanto $\mathcal{\mathcal{O}} =
    B \in \mathcal{B}_2$ o
    \[\mathcal{\mathcal{O}} = B \hspace{1mm} \cup \bigcup\limits_{x
        \in \mathcal{\mathcal{O}} \backslash B}\{x\} = \bigcup\limits_{x \in
        \mathcal{\mathcal{O}} \backslash B} \big(B \cup \{x\}\big)
      \hspace{2mm} \text{donde} \hspace{2mm} \big(B \cup \{x\}\big) \in
      \mathcal{B}_2 \hspace{4mm} \forall x \in \mathcal{\mathcal{O}}
      \backslash B\]

    Tomamos $\mathcal{B} = \mathcal{B}_1 \cup \mathcal{B}_2$, por lo
    que acabamos de probar, cualquier abierto de la topología se puede
    expresar como unión de elementos de $\mathcal{B}$ y por tanto
    $\mathcal{B}$ es base de la topología $\mathcal{T}$.

  \item $x \in A$, por lo tanto cualquier abierto
    $\mathcal{\mathcal{O}}$ que contenga a $x$ tendrá que cumplir $B
    \subseteq \mathcal{\mathcal{O}}$, luego todo entorno de $x$ debe
    contener a $B$. Como el propio $B$ es abierto y $x \in B$, es entorno
    de $x$. Por tanto $\beta^x = \{B\}$.
    
  \end{enumerate}
  
\end{solution}

\newpage

% Ejercicio 3
\begin{exercise} (3 puntos) Sea $F : \mathbb{R}^{n+1} \backslash \{0\}
  \longrightarrow \mathbb{S}^n \times (0, \infty)$ la aplicación
  definida por \vspace{2mm}
  \[F(x) = \bigg(\frac{x}{\|x\|},\|x\|\bigg) \hspace{4mm} \forall x
    \in \mathbb{R}^{n+1} \backslash \{0\}.  \vspace{2mm}\]

  \begin{enumerate}

  \item Probar que $F$ es un homeomorfismo de $(\mathbb{R}^{n+1}
    \backslash \{0\}, \mathcal{T}_u)$ sobre $(\mathbb{S}^n \times (0,
    \infty), \mathcal{T}_u \times \mathcal{T}_u)$.

  \item Probar que $G : (\mathbb{R}^{n+1} \backslash \{0\},
    \mathcal{T}_u) \longrightarrow (\mathbb{S}^n, \mathcal{T}_u)$,
    definida por \vspace{2mm}
    \[G(x) = \frac{x}{\|x\|} \hspace{6mm} \forall x \in
      \mathbb{R}^{n+1} \backslash \{0\}, \vspace{-2mm}\] es una aplicación
    abierta.

  \end{enumerate}
  
\end{exercise}

% % Solución 3
\begin{solution}

  \text{ }
  
  \begin{enumerate}

  \item $F : \mathbb{R}^{n+1} \backslash \{0\} \longrightarrow
    \mathbb{S}^n \times (0, \infty)$ es continua si $F : \mathbb{R}^{n+1}
    \backslash \{0\} \longrightarrow \mathbb{R}^{n+1} \times \mathbb{R}$
    lo es. Basta comprobar que cada una de sus componentes es continua:
    $F_1 : \mathbb{R}^{n+1} \backslash \{0\} \longrightarrow \mathbb{S}^n$
    y $F_2 : \mathbb{R}^{n+1} \backslash \{0\} \longrightarrow (0,
    \infty)$ dadas por \vspace{2mm}
    \[F_1(x) = \frac{x}{\|x\|}; \text{ } F_2(x) = \|x\| \hspace{6mm}
      \forall x \in \mathbb{R}^{n+1} \backslash \{0\} \vspace{4mm}\] son
    continuas, la función que a cada vector le asigna su norma es
    continua, y $F_1$ es cociente de continuas. $F = F_1 \times F_2$, por
    lo tanto es continua.

    Ahora probemos que $F$ tiene inversa, definimos $F^{-1} :
    \mathbb{S}^n \times (0, \infty) \longrightarrow \mathbb{R}^{n+1}
    \backslash \{0\}$ como
    \[F^{-1}(s,t) = ts \hspace{6mm} \forall s \in \mathbb{S}^n, \text{
      } \forall t \in (0, \infty)\] Necesitamos probar que $F(F^{-1}(s, t))
    = (s, t) \hspace{4mm} \forall s \in \mathbb{S}^n, \hspace{2mm} \forall
    t \in (0, \infty)$ y \\ $F^{-1}(F(x)) = x \hspace{4mm} \forall x \in
    \mathbb{R}^{n+1}$.
    
    $s \in \mathbb{S}^n, \hspace{2mm} t \in (0, \infty)$ \vspace{-2mm}
    
    \[F(F^{-1}(s, t)) = F(ts) = \bigg(\frac{ts}{\|ts\|}, \|ts\|\bigg)
      \\ \vspace{4mm} = \bigg(\frac{ts}{|t|\|s\|}, |t| \|s\|\bigg) = (s,
      t)\]

    Ya que $|t| = t$ y $\|s\| = 1$. \vspace{2mm}
    
    $ x \in \mathbb{R}^{n+1} \backslash \{0\}$ \vspace{-2mm}

    \[F^{-1}(F(x)) = F^{-1}\bigg(\frac{x}{\|x\|},\|x\|\bigg) = \|x\|
      \frac{x}{\|x\|} = x \]

    Con esto hemos probado que $F$ tiene inversa, por lo que es
    biyectiva. Obviamente, $F^{-1}$ es continua, por lo que $F$ es una
    biyección continua con inversa continua, un homeomorfismo de
    $(\mathbb{R}^{n+1} \backslash \{0\}, \mathcal{T}_u)$ sobre
    $(\mathbb{S}^n \times (0, \infty), \mathcal{T}_u \times
    \mathcal{T}_u)$.

  \item Consideramos la proyección $\pi : \mathbb{S}^n \times (0,
    \infty) \longrightarrow \mathbb{S}^n$ definida por $\pi(s,t) = s
    \hspace{2mm} \forall s \in \mathbb{S}^n \hspace{2mm} \forall t \in (0,
    \infty)$, que sabemos que es abierta. Además, como $F$ es
    homomorfismo, en particular es abierta. Podemos expresar $G$ como $\pi
    \circ F$. Por lo que si consideramos cualquier abierto
    $\mathcal{\mathcal{\mathcal{O}}} \in \mathcal{T}_u$ se tiene
    \[G(\mathcal{\mathcal{O}}) = (\pi \circ F)(\mathcal{\mathcal{O}})
      = \pi(F(\mathcal{\mathcal{O}}))\] $F(\mathcal{\mathcal{O}})$ es
    abierto por que $F$ es abierta, y $\pi(F(\mathcal{\mathcal{O}}))$ es
    abierto porque $\pi$ es abierta. Por tanto $G$ es abierta.
    
  \end{enumerate}
  
\end{solution}

\end{document}