\section{Introducción}

En esta sección, definimos el concepto de \emph{variedad diferenciable}. Para
ello, comenzamos definiendo el de \emph{superficie diferenciable} que ha podido
verse en otras asignaturas, para después generalizarlo.
\begin{ndef}[Superficie diferenciable]
  Una \emph{superficie de $\R^{3}$} es un subconjunto $S\subset \R^{3}$ tal que,
  para cada $p\in S$, existen $U\subseteq \R^{2}, V\subset \R^{3}$ abiertos, con
  $p\in V$, y existe $X: U\to \R^{3}$ diferenciable, de manera que:
  \begin{nlist}
  \item
    $X(U) = V\cap S$,
  \item
    $X: U\to V\cap S$ es un homeomorfismo, y
  \item
    para cada $q\in U$, $\d X_{q}: \R^{2} \to \R^{3}$ es inyectiva.
  \end{nlist}
\end{ndef}
\begin{nota}
  Si eliminamos el requisito de que $X$ sea diferenciable, la definición es la
  de \emph{superficie topológica}.
\end{nota}
Cuando trabajamos con la diferencial $\d f$ de una aplicación diferenciable $f:
O\subseteq \R^{n} \to \R^{m}$, solemos representarla matricialmente respecto de
las bases canónicas de los espacios euclídeos, y escribimos:
\[
  \d f_{p} = \left(\frac{\partial f_{i}}{\partial x_{j}}\right)_{ij}(p)
.\]
Esta definición depende de la elección de las bases canónicas. Por ello, no se
puede generalizar directamente al caso en que intervengan espacios que no posean
una.

Para subsanar este defecto, damos la siguiente definición.
\begin{ndef}
  Sea $\alpha: (-\epsilon,\epsilon) \to O\subseteq \R^{n}$ diferenciable, con
  $\alpha(0) = p$, $\alpha'(0) = v$. Entonces, la diferencial de $f$ evaluada en
  $v$ es
  \[
    \d f_{p}(v) := \restr{\frac{\d}{\d t}}{t=0} (f\circ \alpha)(t)
  .\]
\end{ndef}
De la aplicación de la regla de la cadena resulta que la diferencial no depende
de la elección de $\alpha$, y que esta definición es equivalente a la
representación matricial.

Ahora, generalizamos la definición de superficie diferenciable para definir una
subvariedad de dimensión arbitraria $n$, todavía dentro de un espacio euclídeo.
\begin{ndef}
  Una subvariedad $n$-dimensional de $\R^{N}$ ($n<N$) es un subconjunto
  $M\subset \R^{N}$ tal que, para cada $p\in M$, existen
  $U\subseteq \R^{n}, V\subseteq \R^{N}$ abiertos, con $p\in V$, y existe
  $X: U\to \R^{N}$ diferenciable, de manera que:
  \begin{nlist}
  \item
    $X(U) = V\cap M$,
  \item
    $X: U\to V\cap M$ es un homeomorfismo, y
  \item
    para cada $q\in U$, $\d X_{q}: \R^{n} \to \R^{N}$ es inyectiva.
  \end{nlist}
\end{ndef}

Veamos algunos ejemplos de subvariedades de dimensión superior.
\begin{ejemplo}[Grafos]
  Sea $F: O\subseteq\R^{n}\to \R^{N-n}$ ($n<N$) diferenciable. Entonces, el
  grafo de $F$ viene dado por
  \[
    \operatorname{Gr}(F) := \left\{ (x,F(x)) : x\in O \right\} \subseteq \R^{N}
  .\]
$\operatorname{Gr}(F)$ es una subvariedad $n$-dimensional de $\R^{N}$: tomando en
la definición $U=O, V=\R^{N}$ para cada $p\in \operatorname{Gr}(F)$, y $X: U\to
\R^{N}$ dada por $X(x) = (x, F(x))$.

La primera condición se cumple claramente: $X(U) = \operatorname{Gr}(F)$.  Para
la segunda, basta observar que la proyección de las $n$ primeras componentes
restringida $\operatorname{Gr}(F)$ es la inversa de $X$. Para la tercera, la
diferencial de $X$ viene dada por
\[ %% TODO: mejorar esto.
  \d X_{q} = \left(
    \begin{array}{c}
      \begin{array}{c|c}
      I_{n} &  0_{n\times (N-n)}
      \end{array}
\\[2pt]\hline
      \frac{\partial F_{i}}{\partial x_{j}}
    \end{array}
     \right) 
,\]
lo que define un monomorfismo.   

En este ejemplo, obtenemos que $\operatorname{Gr}(F)$ es homeomorfa a un
abierto. Por tanto, una subvariedad compacta jamás aparecerá como un grafo.
\end{ejemplo}

El concepto de \emph{valor regular}, y el siguiente teorema, cuya demostración
diferimos, nos permiten dar algunos ejemplos de subvariedades de espacios
euclídeos.
\begin{ndef}[Valor regular]
  Sea $f: O\subseteq\R^{N} \to \R^{m}$ diferenciable, con $m\le N$. Dado
  $a\in\R^{m}$, diremos que es un \emph{valor regular de $f$} si, para cada
  $p\in f^{-1}(a)$, $\d f_{p}$ es sobreyectiva.
\end{ndef}
\begin{nth} \label{thm:valores-regulares}
  Sean $f: \R^{N}\to \R^{N-n}$, $a\in\R^{N}$ un valor regular de $f$. Si
$f^{-1}(a) \ne \emptyset$, entonces $M=f^{-1}(a)$ es una subvariedad
$n$-dimensional de $\R^{N}$.
\end{nth}
Vamos a construir algunas subvariedades como preimágenes de valores regulares, y
probaremos que lo son apoyándonos en este teorema.

\begin{ejemplo}[Esfera de centro $p$, radio $r>0$] \label{ejemplo:esfera}
  La esfera $n$-dimensional de centro $p$ y radio $r>0$,
  \[
    \mathbb{S}^{n}(r,p) = \left\{ x\in\R^{n+1} : \|x-p\|_{2}^{2} = r^{2} \right\}
  ,\] es una subvariedad de codimensión 1.

    Definimos
      \begin{alignat*}{2}
      f: \R^{n+1} & \to  \R \\
          x & \mapsto  \|x-p\|^{2} = \langle x-p,x-p \rangle.
      \end{alignat*}

    Tomamos $x\in\mathbb{S}^{n}(r,p)$, $v\in\R^{n}$, y $\alpha:
    (-\epsilon,\epsilon)\to\R^{n+1}$, con $\alpha(0)=x$, $\alpha'(0) = v$.
    Entonces,
      \begin{align*}
      \d f_{x}(v) &= \restr{\frac{\d}{\d t}}{t=0} \langle \alpha(t) -
                    p,\alpha(t) - p \rangle \\
                  &= 2\langle \alpha'(0), \alpha(0) - p \rangle \\
        &= 2\langle v, x-p \rangle,
      \end{align*}
  sobreyectiva.
\end{ejemplo}
\begin{ejemplo}[Toros $n$-dimensionales]
  El toro $n$-dimensional,
  \[
    \mathbb{T}^{n} = \left\{ (x_{1},y_{1},\dots,x_{n},y_{n})\in\R^{2n} :
      \begin{array}{lll}
        x_{1}^{2}+y_{1}^{2}& = & r_{1}^{2} \\
                           & \vdots & \\
        x_{n}^{2} + y_{n}^{2} & = & r_{n}^{2}
      \end{array}
\right\} = \prod_{i=1}^{n} \mathbb{S}^{1}(r_{i})
  ,\]
es también una subvariedad $n$-dimensional, de $\R^{2n}$ en este caso.

Definimos
\begin{alignat*}{2}
f: \R^{2n} & \to  \R^{n} \\
    (x_{1},y_{1},\dots,x_{n},y_{n}) & \mapsto (x_{1}^{2} + y_{1}^{2}, \dots, x_{n}^{2} + y_{n}^{2}),
\end{alignat*}
y así $\mathbb{T}^{n} = f^{-1}(r_{1}^{2}, \dots, r_{n}^{2}) \ne \emptyset$.

Ahora, tomamos $x\in f^{-1}(r_{1}^{2}, \dots, r_{n}^{2})$, y consideramos $\d
f_{x}: \R^{2n}\to\R^{n}$. Tomamos $v = (w_{1}, z_{1}, \dots, w_{n},
z_{n})\in\R^{2n}$, $\alpha(t) = (\beta_{1}(t), \gamma_{1}(t), \dots,
\beta_{n}(t), \gamma_{n}(t))$ con $\alpha(0) = x, \alpha'(0) = v$. Tenemos
\begin{align*}
  \restr{\frac{\d}{\d t}}{t = 0} (f\circ \alpha)(t) = & \\
    = & \restr{\frac{\d}{\d t}}{t = 0}
        (\beta_{1}^{2}(t) +
        \gamma_{1}^{2}(t),
        \dots, \beta_{n}^{2}(t)
        + \gamma_{n}^{2}(t))\\
    = & 2(x_{1}w_{1} + y_{1}z_{1}, \dots, x_{n}w_{n}+y_{n}z_{n}),
\end{align*}
trivialmente inyectiva.

\begin{nota}
  A la hora de calcular la diferencial en ejercicios, una buena práctica es
  comprobar que sea, efectivamente, lineal.
\end{nota}
\end{ejemplo}

\begin{proof}[Demostración del teorema \ref{thm:valores-regulares}]
  Sea $x_{0}\in f^{-1}(a)$. Suponemos $\d f_{x_{0}}: \R^{N}\to \R^{N-n}$
sobreyectiva. Viene representada por
  \[
    \d f_{x_{0}} = \left(
      \begin{array}{c|c}
        \displaystyle\frac{\partial f_{i}}{\partial x_{j}} & \displaystyle\frac{\partial f_{i}}{\partial x_{k}}
      \end{array}
\right)_{\substack{i=1,\dots,N-n \\ j=1,\dots,n \\ k=n+1,\dots,N}}(x_{0})
  \]
El teorema de Rouché-Frobenius nos asegura que $\d f_{x_{0}}$ posee un menor de
orden $(N-n)\times (N-n)$ con determinante no nulo. Supongamos, sin pérdida de
generalidad, que este es
\[
  \begin{pmatrix}
    \displaystyle \frac{\partial f_{i}}{\partial x_{k}}
  \end{pmatrix}_{\substack{i=1,\dots,N-n \\ k=n+1,\dots,N}}(x_{0})
.\]
Ahora, definimos
\begin{alignat*}{2}
    G: O \subseteq \R^{N} & \to \R^{N} \\
    x & \mapsto  (x_{1}, \dots, x_{n}, f_{1}(x), \dots, f_{N-n}(x)),
\end{alignat*}
y la diferencial de $G$ viene dada por
\[
  \d G = \left(
    \begin{array}{c|c}
      I_{n} & 0_{n\times N-n} \\ \hline
      \frac{\partial f_{i}}{\partial x_{j}} & \frac{\partial f_{i}}{\partial x_{k}}
    \end{array}
\right)
,\]
luego $\det(\d G(x_{0})) = \det(I_{n})\det\left(\frac{\partial f_{i}}{\partial
x_{k}}\right) \ne 0$. Aplicamos el teorema de la función inversa a $G$, y
obtenemos $V\subseteq O$ abierto, con $x_{0}\in V$, de manera que $G: V \to W :=
G(V)$ es un difeomorfismo.

Ahora,
\[
  G(f^{-1}(a)) = \left\{ (x_{1}, \dots, x_{n}, a) : x\in f^{-1}(a) \right\}
,\]
luego
\[
  G: V\cap f^{-1}(a)\to \hat{W} = W\cap G(f^{-1}(a)) = \left\{ x\in W : (x_{n+1}, \dots, x_{N}) = a \right\}
.\]
$V\cap f^{-1}(a)$ es abierto en $f^{-1}(a)$, luego $\hat{W} = G(V\cap
f^{-1}(a))$ es abierto en $G(f^{-1}(a)) = \left\{ (x,a) : x\in f^{-1}(a)
\right\} = f^{-1}(a)\times \{a\}$.

Definiendo $\pi: G(f^{-1}(a)) \to \R^{n}$ como la proyección sobre las $n$
primeras coordenadas, observamos que $\pi$ es una aplicación abierta, al ser una
proyección de un espacio topológico producto a una de sus componentes. Por
tanto, $U := \pi(\hat{W})$ es abierto, y
\begin{alignat*}{2}
    X: U \subseteq \R^{n} & \to \R^{N} \\
    x & \mapsto G(x,a)
\end{alignat*}
es la parametrización buscada. En efecto,
\[
  \d X_{q} = \underbrace{\d G_{-}}_{\text{isomorfismo}}\circ \underbrace{\d(x\mapsto (x,a))}_{\text{inyectiva}}
.\]
\end{proof}

Continuamos con algunos ejemplos más de subvariedades de espacios euclídeos que
se obtienen mediante este teorema. Los siguientes ejemplos aparecerán como
grupos de Lie de matrices. Para abordarlos, introducimos la notación
  \[
    \gl(n,\R) := \mathscr{M}_{n}(\R)
  \]
  para el espacio vectorial de matrices de orden $n$.
\begin{ejemplo}[Grupo ortogonal] \label{ejemplo:ortogonal}
  Vamos a ver que
  \[
    O(n) := \left\{ A\in\gl(n,\R) : AA^{T} = I \right\}
  \]
  es una subvariedad de $\gl(n,\R)$. Las matrices simétricas
  \[
    \Sim(n,\R) := \left\{ A\in\gl(n,\R) : A = A^{T} \right\}
  \]
  son un subespacio de dimensión $\frac{n^{2}+n}{2}$ de $\gl(n,\R)$.
  Definimos
\begin{alignat*}{2}
    f: \gl(n,\R) & \to \Sim(n,\R) \\
    A & \mapsto AA^{T}
\end{alignat*}
y veamos que $I$ es un valor regular de $f$:
\[
  \d f_{A}(B) = AB^{T} + BA^{T}
,\]
sobreyectiva, tomando $B = \frac{CA}{2}$ para obtener $\d f_{A}(B) = C$. Por
tanto, $O(n)$ es una subvariedad de dimensión $\frac{n(n-1)}{2}$ de $\gl(n,\R)$.
\end{ejemplo}

\begin{ejemplo}[Grupo especial lineal]
  El grupo especial lineal
  \[
    \Sl(n,\R) = \left\{ A\in \gl(n,\R) : \det(A) = 1 \right\}
  \]
  es una hipersuperficie (subvariedad de codimensión 1) de $\gl(n,\R)$. Para
  probarlo, vamos a ver que $1$ es un valor regular del determinante.

  Sea $A\in\det^{-1}(1)$, y consideramos la curva $A(t) = (1+t)A$ --la curva
  \emph{que pasa a velocidad $A$ por $A$ en el cero}--, para obtener
  \begin{align*}
    \d (\det)_{A}(A) &= \restr{\frac{\d}{\d t}}{t = 0} \det((1+t)A)\\
                   &= \restr{\frac{\d}{\d t}}{t = 0}(1+t)^{n}\det(A)\\
    &= n\det(A) \ne 0.
  \end{align*}
\end{ejemplo}

\section{Definición de variedad diferenciable}

En esta sección, vamos a construir la definición de variedad diferenciable a
partir de aproximaciones a ella. Veremos los problemas que presenta cada
aproximación, hasta dar con la definición correcta.

El principal objetivo de esta definición será liberarnos del espacio euclídeo
ambiente, para poder estudiar variedades definidas en espacios más generales.

\subsection{Primera aproximación}

Empezamos con el concepto de variedad topológica, y ya podemos hablar de
diferenciabilidad de aplicaciones que nacen en una subvariedad de $\R^{N}$.
\begin{ndef}[Variedad topológica]
  Sea $M^{n}$ un espacio topológico Haussdorf, tal que para cada $p\in M$, existen
  $U\subseteq \R^{n}, V\subseteq M$ abiertos, y $X: U\to V$ tales que
  \begin{nlist}
  \item
    $X: U \to V$ es un homeomorfismo.
  \end{nlist}
\end{ndef}

Esta definición presenta cierta estructura diferenciable, es decir, podemos
hablar de funciones diferenciables.
\begin{ndef}
  Sea $f: M^{n} \to \R$, con $M^{n}$ una subvariedad $n$-dimensional de
$\R^{N}$. Diremos que $f$ es diferenciable si, para cada $p\in M^{n}$, existe
una parametrización $X: U\to M^{n}$ tal que $f\circ X: U\to \R$ es
diferenciable.
\end{ndef}
Se puede probar que, con esta definición, la diferenciabilidad no depende de la
parametrización escogida.

Esta definición dota a las variedades de estructura topológica, pero no
diferenciable, en el sentido de que no se exige diferenciabilidad a las
parametrizaciones.

Además, solo tenemos garantía de que la diferenciabilidad es invariante ante
cambios de parametrizaciones para subvariedades de $\R^{N}$, como ilustra el
siguiente ejemplo. Esto nos impide extender la definición anterior a variedades
topológicas arbitrarias.
\begin{ejemplo}
  Tomamos la variedad topológica $\R$ con la parametrización
\[
    X(t) =
  \begin{cases}
    t & t < 0,\\
    2t & t \ge 0.
  \end{cases}
\]
La aplicación $X^{-1}$ es diferenciable, en el sentido de la definición
anterior, con la parametrización $X$, pero no con la identidad.
\end{ejemplo}

\subsection{Segunda aproximación}

Para intentar capturar la invarianza de la diferenciabilidad por cambios de
parametrizaciones, podemos añadir la siguiente propiedad a la definición anterior:
\begin{nlist}
\setcounter{enumi}{1}
\item
  para cada par de parametrizaciones $X: U \to M, \tilde{X}: \tilde{U}\to M$,
  con $X(U)\cap \tilde{X}(\tilde{U}) = W \ne \emptyset$, la aplicación
  $\tilde{X}^{-1}\circ X: X^{-1}(W)\to \tilde{X}^{-1}(W)$ es de clase $\mathscr{C}^{\infty}$,
\end{nlist}
y considerar cada variedad con un conjunto de parametrizaciones fijo.

Sin embargo, esto tampoco es suficiente.
\begin{ejemplo}
  Consideramos de nuevo la variedad $\R$, con las parametrizaciones $X =
\mathrm{id}$ y $\hat{X}(t) = t^{3}$. Estas dos parametrizaciones verifican la
propiedad añadida --cada una por separado--, y sin embargo, $f(t) = t^{1/3}$
sería diferenciable con $\hat{X}$ y no con $X$.
\end{ejemplo}

\subsection{Definición}

Pasamos ya a presentar la definición correcta de variedad diferenciable.
\begin{ndef}[Estructura diferenciable]
  Sea $M$ un espacio topológico Haussdorf. Una \emph{estructura diferenciable
    $n$-dimensional sobre $M$} es una familia $\mathscr{D} = \left\{ (V_{i},
    \phi_{i}) \right\}_{i\in I}$ que verifica
  \begin{nlist}
  \item
    $\left\{ V_{i} \right\}_{i\in I}$ es un recubrimiento abierto de $M$,
  \item para cada $i\in I$, $\phi_{i}: V_{i} \to \phi_{i}(V_{i})$,
  $\phi_{i}(V_{i})$ es abierto, y $\phi_{i}$ es un homeomorfismo,
  \item
    Si $V_{i}\cap V_{j}\ne\emptyset$, entonces $\phi_{j}\circ \phi_{i}^{-1}:
    \phi_{i}(V_{i}\cap V_{j})\to \phi_{j}(V_{i}\cap V_{j})$ es un difeomorfismo, y
  \item
    $\mathscr{D}$ es maximal entre las familias que verifican las propiedades anteriores.
  \end{nlist}
\end{ndef}

\begin{ndef}[Variedad diferenciable]
  Una \emph{variedad diferenciable $n$-dimensional} es un par $(M, \mathscr{D})$.
\end{ndef}
\begin{notacion}
  En ocasiones, usaremos la notación $M^{n}$ para una variedad diferenciable
  $n$-dimensional $M$, para indicar su dimensión.
\end{notacion}

Llamamos a cada par $(V,\phi)$ de una estructura diferenciable \emph{carta} o
\emph{entorno coordenado}.

\begin{ndef}[Atlas]
  Llamamos \emph{atlas} a una estructura que verifique las tres primeras
  propiedades de la definición de estructura diferenciable.
\end{ndef}

Algunas propiedades de las cartas de una estructura diferenciable:
\begin{nprop} \label{prop:cartas}
  Sea $(V,\phi)\in\mathscr{D}$. Se verifican
  \begin{nlist}
  \item \label{item:rest-cartas}
    si $W\subseteq V$ abierto, entonces
$(W,\restr{\phi}{W})\in\mathscr{D}$,
  \item 
    si $\psi: \phi(V) \subseteq \R^{n} \to O\subseteq\R^{N}$ es un
    difeomorfismo, entonces $(V, \psi\circ\phi)\in\mathscr{D}$.
  \end{nlist}
\end{nprop}

Y algunas propiedades de los atlas:
\begin{nprop} \label{prop:unicidad-atlas}
  Sea $\mathscr{A}$ un atlas $n$-dimensional en $M$. Se verifican
  \begin{nlist}
  \item
    existe una única estructura diferenciable $\mathscr{D}(\mathscr{A})$ con
    $\mathscr{A}\subseteq \mathscr{D}(\mathscr{A})$,
  \item
    si $\mathscr{A}'$ es otro atlas $n$-dimensional con $\psi\circ\phi^{-1}$
    para cada par $\psi\in \mathscr{A}'$, $\phi\in \mathscr{A}$, entonces
    $\mathscr{D}(\mathscr{A}) = \mathscr{D}(\mathscr{A}')$.
  \end{nlist}
\end{nprop}

\subsection{Ejemplos de variedades diferenciables}

\begin{ejemplo}[$\R^{n}$]
  $\mathscr{A} = \left\{ (\R^{n}, \mathrm{id}) \right\}$ es un atlas de esta
  variedad. La estructura diferenciable asociada contiene todos los
  difeomorfismos entre abiertos de $\R^{n}$.
\end{ejemplo}

\begin{ejemplo}[Abierto de una variedad]
  Sea $O\subseteq M^{n}$ abierto. Entonces, la restricción de la estructura
  diferenciable
  \[
    \restr{\mathscr{D}}{O} = \left\{ (V\cap O, \restr{\phi}{V\cap O}) \right\}
  \]
  es una estructura diferenciable en $O$. El hecho de que $O$ sea abierto es
  necesario para garantizar la tercera condición en la definición de estructura diferenciable.
\end{ejemplo}

\begin{ejemplo}[Subvariedades]
  Cualquier subvariedad de $\R^{N}$ es una variedad diferenciable, considerando en
  ella el atlas
  \[
    \mathscr{A} = \left\{ (V\cap M, \restr{X^{-1}}{V\cap M})\, | \, X: U\subseteq
      \R^{n}\to V\subseteq\R^{N} \text{ parametrización} \right\}
  .\]
\end{ejemplo}

\begin{ejemplo}[Variedad producto]
  Sean $M_{1}^{n_{1}}, M_{2}^{n_{2}}$ variedades diferenciables. En $M_{1}\times
  M_{2}$ consideramos, con su topología producto,
  \[
    \left\{ (V_{1}\times V_{2}, \phi_{1}\times\phi_{2}) : (V_{i}, \phi_{i})\in \mathscr{D}_{i} \right\}
  ,\]
que es un atlas.
\end{ejemplo}

\begin{ejemplo}[Cocientes]
  Los cocientes de variedades \textbf{no} producen, en general,
  variedades. Basta pensar en la relación de equivalencia por la que se obtiene
  un cono a partir de un cilindro.
\end{ejemplo}

\begin{ejemplo}[Espacio proyectivo real] \label{ejemplo:proyectivo}
  El espacio proyectivo real es un cociente
  \[
    \RP^{n} = \frac{\R^{n+1}-\{0\}}{\mathrm{R}}
  ,\]
con $R$ la relación de equivalencia $x\mathrm{R}y \iff x=\lambda y, \lambda\ne
0$. Este es un caso particular de cociente de una variedad que \textbf{sí} es
una variedad. Para verlo, vamos a construir un atlas.

Recordemos que la topología de un cociente $\frac{X}{\mathrm{R}}$ viene dada
de la siguiente forma: $O\subseteq \frac{X}{\mathrm{R}}$ es abierto si
$\pi^{-1}(O)$ es abierto en $X$.

Ahora, sean los abiertos de $\R^{n+1}$
\[
  O_{i} = \left\{ x\in\R^{n+1} : x_{i} \ne 0 \right\}
\]
para $i=1,\dots,n+1$. Estos abiertos forman un recubrimiento de $\R^{n+1}$, y
además tienen la siguiente propiedad: $x\in O_{i}$ implica que $\pi(x)\subseteq
O_{i}$. Esta propiedad implica que $V_{i} := \pi(O_{i})$ son abiertos. Además,
recubren a $\RP^{n}$. Para construir las cartas, vamos a contruir
aplicaciones $\psi_{i}: O_{i}\to \R^{n}$, y vamos a usarlas para inducir
aplicaciones $\phi_{i}$ que hagan conmutativo el diagrama
\[
  \begin{tikzcd}
    O_{i} \arrow{r}{\psi_{i}} \arrow{d}{\pi_{i}} & \R^{n}\\
    V_{i} \arrow[dashed]{ur}{\phi_{i}}
  \end{tikzcd}
\]
Recordemos que $\psi_{i}$ induce una tal aplicación si $x\mathrm{R}y$ implica
$\psi_{i}(x) = \psi_{i}(y)$, y que dicha aplicación será un homeomorfismo si
$\psi_{i}$ es una identificación. Una condición suficiente para que $\psi_{i}$
sea una identificación es que tenga una inversa continua a derecha. Pues bien,
vamos a definir $\psi_{i}$ y ver que cumple estas propiedades.
\begin{alignat*}{2}
    \psi_{i}: O_{i} & \to \R^{n} \\
    (x_{1},\dots,x_{n+1}) & \mapsto \frac{1}{x_{i}}(x_{1},\dots,x_{i-1},x_{i+1},\dots,x_{n+1}).
\end{alignat*}
  Es inmediato que respeta la relación de equivalencia $\mathrm{R}$ y que
  $h_{i}(y_{1},\dots,y_{n}) = (y_{1},\dots,y_{i-1},1,y_{i+1},\dots,y_{n})$ es su
  inversa a derecha. 
\end{ejemplo}

\section{Aplicaciones diferenciables entre variedades}

Consideramos una aplicación $F: M^{m}\to N^{n}$ continua entre dos variedades
diferenciables $M$ y $N$.
\begin{ndef}
  $F$ es \emph{diferenciable} si, para cada $p\in M$, existe $(V,\phi)$ carta de
  $M$ con $p\in V$, y además existe $(W,\psi)$ carta de $N$ con $F(p)\in W$
  tales que
  \[
    \psi\circ F\circ \phi^{-1}: \phi(F^{-1}(W)\cap V)\to \psi(W)
  \]
  es diferenciable.
\end{ndef}
De la definición se desprende el siguiente hecho: si $(\hat{V},\hat{\phi}),
(\hat{W},\hat{\psi})$ son otras cartas, entonces
\[
  \hat{\psi}\circ F\circ\hat{\phi}^{-1} = \underbrace{\hat{\psi}\circ\psi^{-1}}_{\text{dif.}}\circ\underbrace{\psi\circ
  F \circ \phi^{-1}}_{\text{dif.}}\circ \underbrace{\phi \circ \hat{\phi}^{-1}}_{\text{dif.}}
,\]
restringiendo las aplicaciones a un abierto más pequeño si fuera necesario. Es
decir, se verifica la siguiente proposición.
\begin{nprop}
  $\psi\circ F\circ \phi^{-1}$ es diferenciable para cualesquiera cartas
  $(V,\phi), p\in V$, $(W,\psi), F(p)\in W$, si y solo si existe un par de
  cartas para el que sea diferenciable.
\end{nprop}

Podemos trasladar también el concepto de difeomorfismo.
\begin{ndef}[Difeomorfismo]
  $F$ es un difeomorfismo si tiene inversa $F^{-1}$ y ambas son diferenciables.
\end{ndef}
\begin{nota}
  Esto implica $m=n$.
\end{nota}

\begin{ejemplo}
  Consideramos en $\R$ dos atlas: $\mathscr{A} = \left\{ (\R,\mathrm{id})
  \right\}$, y $\mathscr{A}' = \left\{ (\R,t\mapsto t^{1/3}) \right\}$. Además,
  consideramos la aplicación
\begin{alignat*}{2}
    F: (\R,\mathscr{D}(\mathscr{A})) & \to (\R, \mathscr{D}(\mathscr{A}')) \\
    t & \mapsto t^{3}.
\end{alignat*}
  Claramente, $(t^{1/3})\circ F\circ \mathrm{id}^{-1} = \mathrm{id}$ y
  $\mathrm{id}\circ F^{-1}\circ (t^{1/3})^{-1} = \mathrm{id}$ son
  diferenciables, luego $F$ es un difeomorfismo con estas estructuras
  diferenciables, aunque no lo sea con la usual.
\end{ejemplo}

\begin{nprop}[Propiedades de las aplicaciones diferenciables] \label{prop:propiedades-dif}
  \hfill
  \begin{nlist}
  \item
    $F:M^{m}\to N^{n}$ constante es diferenciable,
  \item
    la aplicación identidad en un variedad es diferenciable,
  \item
    la composición de aplicaciones diferenciables es diferenciable,
  \item
    la composición de difeomorfismos es un difeomorfismo,
  \item \label{item:prod-dif}
    si $f,g: M\to\R$ son diferenciables, entonces $fg: M\to\R$ es
    diferenciable,
  \item
    si $O\subseteq M$ es abierto, la inclusión en $M$ es diferenciable. Por
    tanto, las restricciones a abiertos de aplicaciones diferenciables son
    diferenciables,
  \item \label{item:carta-difeo}
    si $(V,\phi)$ es una carta de $M$, y $\phi: V\to \phi(V)$ es un difeomorfismo.
  \end{nlist}
\end{nprop}
\begin{proof}
  \ref{item:prod-dif}: sea $(V,\phi)$ una carta alrededor de $p\in M$. Entonces,
  $f\circ \phi^{-1}$ y $g\circ \phi^{-1}$, de $\R^{n}$ a $\R$, son
  diferenciables. Solo queda observar que
  $(fg)\circ\phi^{-1} = (f\circ\phi^{-1})(g\circ\phi^{-1})$.
  
  \ref{item:carta-difeo}: considerando en $\R^{n}$ la estructura diferenciable
  usual, $(\phi(V),\mathrm{id})$ es una carta suya. $\mathrm{id}\circ
  \phi\circ\phi^{-1}$ es diferenciable, y también lo es su inversa.
\end{proof}

Algunas propiedades que serán útiles cuando tratemos con subvariedades del euclídeo.
\begin{nprop} \label{prop:prop-dif-sub}
  Sean $M^{n}\subset\R^{N}$ una subvariedad, y $N^{n}$ una variedad
  diferenciable. Entonces,
  \begin{nlist}
  \item \label{item:inc}
    la inclusión $i: M\to\R^{N}$ es diferenciable,
  \item \label{item:inc-2}
    $F: N\to M$ es diferenciable si, y solo si, $i\circ F: N\to\R^{N}$ es
    diferenciable,
  \item \label{item:restr-dif}
    si $O\subseteq\R^{N}$ es un abierto, con $M\subset O$, y $F: O\to N$ es
    diferenciable, entonces $\restr{F}{M}$ es diferenciable.
  \end{nlist}
\end{nprop}
\begin{proof}
  \ref{item:inc}: tomando $p\in M$, $(V,\phi)$ una carta de $M$ con $p\in V$, y
  llamando $X=\phi^{-1}$, el siguiente diagrama conmuta:
\[
  \begin{tikzcd}
    V \arrow{r}{i} \arrow[shift left=1ex]{d}{\phi} & \R^{N} \arrow{d}{\mathrm{id}}\\
    U:=\phi(V)\subseteq\R^{N} \arrow{r}{X} \arrow[shift left=1ex]{u}{X} & \R^{N}
  \end{tikzcd}
\]

\ref{item:inc-2}: tomamos $p\in N$, $(W,\psi)$ una carta con $p\in W$; y
$(V,\phi)$ una carta en $M$ con $F(p)\in V$. Llamamos $U := \phi(V)$,
$X:=\phi^{-1}$.
Tenemos el diagrama
\[
  \begin{tikzcd}
    W \arrow{r}{F} \arrow{d}{\psi} & V \arrow[shift left=1ex]{d}{\phi} \arrow{r}{i} & \R^{N}\\
    \R^{m} & U\subseteq\R^{n} \arrow[shift left=1ex]{u}{X}
  \end{tikzcd}
\]
y tenemos que $i\circ F$ es diferenciable. Por tanto, dada una parametrización
$G$ de $\R^{N}$, $G^{-1}\circ i\circ F\circ \psi^{-1}$ será diferenciable, y lo
será $G^{-1}\circ i \circ X \circ \phi \circ F \circ \psi^{-1}$. Si elegimos $G$
de forma que $G^{-1}\circ i \circ X = x\mapsto (x,0)$, obtendremos el resultado,
pues tendremos
\[
  \underbrace{\underbrace{G^{-1}\circ i \circ X}_{x\mapsto (x,0)} \circ \phi
    \circ F \circ \psi^{-1}}_{\text{diferenciable}}
.\]

Definamos
\begin{alignat*}{2}
  G: U\times\R^{N-n} & \to \R^{N} \\
 (x,y) & \mapsto X(x) + (0,y).
\end{alignat*}
Entonces, es claro que $G(x,0) = X(x)$, luego, supuesta invertible,
$G^{-1}(X(x)) = (x,0)$.  Además, la diferencial de $G$ en $\phi(F(p))$ viene
representada matricialmente en la forma
\[
  \d G_{\phi(F(p))} =
  \left(
  \begin{array}{c|c}
    \frac{\partial X_{i}}{\partial x_{j}} & 0\\ \hline
    \frac{\partial X_{k}}{\partial x_{j}} & I
  \end{array}\right)((\phi\circ F)(p))
,\] y por ser $X$ una parametrización de una subvariedad, su determinante será
no nulo, luego $G$ será un difeomorfismo en un entorno
$\hat{U}\times B(0,\epsilon)$ de $\phi(F(p))$. Además, por continuidad, podemos
suponer que $G(\hat{U}\times B(0,\epsilon))\subseteq V$, y tomando $V$ más
pequeño (proposición \ref{prop:cartas}-\ref{item:rest-cartas}), suponemos la
igualdad, y por tanto conmutará
\[
  \begin{tikzcd}
    W \arrow{r}{F} \arrow{d}{\psi} & V \arrow[shift left=1ex]{d}{\phi}
    \arrow{r}{i} & V \arrow{d}{G^{-1}}\\
    \R^{m} & U\subseteq\R^{n} \arrow[shift left=1ex]{u}{X} \arrow{r}{x\mapsto (x,0)} & \hat{U}\times B(0,\epsilon)
  \end{tikzcd}
\]

\ref{item:restr-dif}: $\restr{F}{M} = F\circ i$.
\end{proof}

\begin{ejer}
  Sean $M_{1}^{n_{1}}$ y $M_{2}^{n_{2}}$ variedades diferenciables. Probar:
  \begin{nlist}
  \item
    Las proyecciones $\pi_{1}$ y $\pi_{2}$ de la variedad producto $M_{1}\times
    M_{2}$ son diferenciables,
  \item
    si $(p_{0},q_{0})\in M_{1}\times M_{2}$, entonces $i_{q_{0}}: M_{1}\to
    M_{1}\times M_{2}$ dada por $i(p) = (p,q_{0})$ es diferenciable.
  \end{nlist}
\end{ejer}
\begin{sol}
  Ver en Ejercicios (\ref{ejer:1}).
\end{sol}

La siguiente proposición caracteriza la diferenciabilidad de aplicaciones que
nacen en espacios proyectivos.
\begin{nprop}
  Sea $M^{m}$ una variedad diferenciable, $F: \RP^{n}\to M$, y $\pi:
  \R^{n+1}-\{0\}\to\RP^{n}$ la proyección del proyectivo. Entonces, $F$ es
  diferenciable si, y solo si, $F\circ \pi$ es diferenciable.
\end{nprop}
\begin{proof}
  \textbf{Supuesto que $F$ es diferenciable:} se sigue de que $\pi$ es diferenciable,
  veámoslo. Usamos la notación introducida en la demostración del ejemplo
  \ref{ejemplo:proyectivo}. Sea $x\in\R^{n+1}-\{0\}$, entonces $x\in O_{i}$ para cierto
  $i$. Queda el diagrama
  \[
    \begin{tikzcd}
      O_{i} \arrow{r}{\pi} \arrow{d}{\mathrm{id}} & \pi(O_{i})
      \arrow{d}{\phi_{i}}\\
      O_{i} \arrow{r}{\phi_{i}\circ\pi} & \R^{n}
    \end{tikzcd}
  \]
  con $\phi_{i}\circ\pi(x_{1},\dots,x_{n+1}) =
  \frac{1}{x_{i}}(x_{1},\dots,x_{i-1},x_{i+1},\dots,x_{n+1})$, diferenciable.

  \textbf{Supuesto que $F\circ\pi$ es diferenciable:} tomando de nuevo $x\in
  O_{i}$, y $(W, \psi)$ una carta de $F(p)$, tenemos el siguiente diagrama:
  \[
    \begin{tikzcd}
      O_{i} \arrow{r}{\pi} \arrow{d}{\mathrm{id}} & \pi(O_{i})
      \arrow{d}{\phi_{i}} \arrow{r}{F} & W \arrow{d}{\psi}\\
      O_{i} \arrow{r}{\phi_{i}\circ\pi} & \R^{n} & \psi(W)
    \end{tikzcd}
  \]
  $F\circ\pi$ es diferenciable, luego lo es $\psi\circ F\circ\pi$. Observando
  que $\phi_{1}^{-1}(y_{1},\dots,y_{n}) =
  \pi(y_{1},\dots,y_{i-1},1,y_{i},\dots,y_{n})$, se obtiene que $\psi\circ F
  \circ \phi_{i}^{-1}$ es diferenciable.
\end{proof}

\subsection{Funciones meseta}

En esta sección, probamos la existencia de funciones diferenciables no triviales
que nacen en variedades diferenciables arbitrarias y toman valores reales.

\begin{lema} \label{lema:meseta}
 Sean $r_{1}$, $r_{2}\in\R^{+}$ con $r_{2}> r_{1}$. Entonces, existe $h:
 \R^{n}\to\R$ diferenciable con
 \begin{nlist}
 \item
   $\restr{h}{B(0,r_{1})} = 1$,
 \item
   $\restr{h}{\R^{n}-B(0,r_{2})} = 0$, y
 \item
   $\restr{h}{B(0,r_{2})-B(0,r_{1})}\in[0,1]$.
 \end{nlist}
\end{lema}

\begin{nprop}[Funciones meseta en una variedad] \label{prop:meseta}
  Sean $M$ una variedad diferenciable, $p\in M$ y $O\subset M$ un
  abierto. Entonces, existen $f: M\to \R$ diferenciable y $V_{p}$ un abierto
  relativamente compacto --es decir, con adherencia compacta--, con
  $p\in\bar{V_{p}}\subset O$, tales que:
  \begin{nlist}
  \item
    $\restr{f}{V_{p}} = 1$, y
  \item
    $\sop(f) = \overline{\left\{ x\in M : f(x)\ne 0 \right\}}$ es
      compacto y contenido en $O$.
  \end{nlist}
\end{nprop}
\begin{proof}
  Sea $(W,\phi)$ una carta de $M$ con $p\in W\subset O$ y $\phi(p) = 0$, lo cual
  se puede suponer sin pérdida de generalidad, componiendo con una
  traslación. Sean $r_{1},r_{2}\in\R^{+}$ con $r_{1}<r_{2}$,
  $\bar{B}(0,r_{2})\subset\phi(O)$, y $h$ la función dada por el lema
  \ref{lema:meseta}.

  Definimos ya $f: M \to \R$: $\restr{f}{W} = h\circ \phi$, $\restr{f}{M-W} =
  0$. Además, definimos $V_{p} := \phi^{-1}(B(0,r_{1}))$, y de esta forma:
  \begin{itemize}
  \item $\overline{V_{p}} = \phi^{-1}(\overline{B(0,r_{1})})$ es compacto y está
    contenido en $W\subset O$,
  \item $\sop(f) = \phi^{-1}(\sop(h))$ es compacto y
    está contenido $W\subset O$.
  \end{itemize}

  Por tanto, $\restr{f}{W-\phi^{-1}(\overline{B(0,r_{2})})} = 0$, y $f$ es
  diferenciable.
\end{proof}

\begin{ncor}
  Sean $M$ una variedad diferenciable, $K\subseteq M$ un compacto, y $O\subseteq
  M$ un abierto con $K\subseteq O$. Entonces, existe $f:M\to\R$ diferenciable,
  con $\restr{f}{K} = 1$, $\sop(f)\subseteq O$.
\end{ncor}
\begin{proof}
  Para cada $p\in K$, sean $f_{p}$ y $V_{p}$ los dados por el teorema anterior
  para $p$ y $O$. Entonces, $\left\{ V_{p} \right\}_{p\in K}$ es un
  recubrimiento por abiertos de $K$. Extraigamos un subrecubrimiento finito
  $\left\{ V_{p_{i}} \right\}_{i=1,\dots,r}$. Definiendo
  \[
    f = 1 - \prod_{i=1}^{r}(1-f_{p_{i}})
  \]
  se obtiene la función deseada. El producto de aplicaciones diferenciables es
  diferenciable (proposición \ref{prop:propiedades-dif}). El soporte de $f$ es
  compacto: es la unión de los soportes de las $f_{p_{i}}$.
\end{proof}

\section{Diferencial de una aplicación}

En esta sección, definimos el espacio tangente de una variedad diferenciable $M$
en un punto $p\in M$, y la diferencial de una aplicación diferenciable entre
variedades diferenciables.

Para entender cómo la siguiente definición generaliza a la ya conocida para
funciones diferenciables entre espacios euclídeos, vamos a hacer algunas
observaciones sobre este último concepto. Sea $f:\R^{n}\to\R$ diferenciable, y
sea $p\in\R^{n}$. La diferencial de $f$ en $p$ viene dada por la forma lineal
\begin{alignat*}{2}
  \d f_{p}: \R^{n} & \to \R \\
  v & \mapsto \restr{\frac{\d}{\d t}}{t = 0}f(p+tv).
\end{alignat*}
En esta definición, cada vector $v\in \R^{n}$ está jugando el papel de operador
de derivación. Podemos formalizar esto de la siguiente forma: definimos
\begin{alignat*}{3}
  \lambda = \lambda_{p}: \R^{n} & \to && (\Cinf(\R^{n})\to\R)   \\
               v & \mapsto && \hat{v}: f \mapsto \d f_{p}(v).
\end{alignat*}
y $\hat{v} = \lambda(v)$ sería $v$ visto como operador de derivación. En la
definición de $\lambda$, usamos $(\Cinf(\R^{n})\to\R)$ para notar el conjunto de
funciones de $\Cinf(\R^{n})$ a $\R$.

La linealidad de la derivada y la regla del producto para la derivación nos
permiten afirmar que la imagen de $\lambda$ está contenida en el subconjunto
\[
  \mathscr{W}^{p} := \left\{ \alpha\in(\Cinf(\R^{n})\to\R) :
    \begin{array}{l}
      \alpha\text{ es lineal} \\
      \alpha\text{ verifica la regla del producto en $p$}
    \end{array}
\right\}
,\]
donde esta segunda restricción debe entenderse del siguiente modo:
\[
  \alpha(fg) = \alpha(f)g(p) + f(p)\alpha(g)
\]
para cada $f,g\in\Cinf(\R^{n})$.

Vamos a probar lo siguiente:
\begin{nth}
  $\mathscr{W}^{p}$ es un espacio vectorial de dimensión $n$ y $\lambda_{p}$ es un
  isomorfismo de espacios vectoriales.
\end{nth}
\begin{proof}
  Que $\mathscr{W}^{p}$ es un espacio vectorial con la suma y el producto por
  escalares usuales se puede comprobar rutinariamente. Que $\lambda$ es lineal
  se obtiene de la linealidad de la diferencial en un punto. Asimismo, es
  inmediato comprobar que el núcleo de $\lambda$ es trivial. La dificultad de la
  demostración reside en comprobar que $\lambda$ es sobreyectiva.

  Sean $\alpha\in \mathscr{W}^{p}$, $x\in \R^{n}$ y
  $f\in\Cinf(\R^{n})$. Primero, observamos que
  \[
    \alpha(1) = \alpha(1)\cdot 1 + 1\cdot\alpha(1) = 2\alpha(1)
  .\]
  Por linealidad, $\alpha(c) = 0$ para cualquier función constante $c$.

  Segundo, observamos que es posible desarrollar $f$ de la siguiente forma:
  \[
    f(x)-f(p) = \int_{0}^{1} \restr{\frac{\d}{\d t}}{t = s}f(tx+(1-t)p)\d s =
    \sum_{i=1}^{n}(x_{i}-p_{i})\underbrace{\int_{0}^{1}\frac{\partial}{\partial
        x_{i}}f(sx + (1-s)p)\d s}_{g_{i}(x)}
  .\]

Por tanto, por linealidad, por la regla del producto y por ser cero en funciones
constantes,
\[
  \alpha(f) = \sum_{i=1}^{n} \alpha(x_{i}-p_{i})g_{i}(p) = \sum_{i=1}^{n} \alpha(x_{i})g_{i}(p)
.\]

Tomando $v = (\alpha(x_{1}), \dots, \alpha(x_{n}))$,
\[
  \lambda(v)(f) = \d f_{p}(\alpha(x_{1}),\dots,\alpha(x_{n})) = \sum_{i=1}^{n}
  \alpha(x_{i})\underbrace{\frac{\partial}{\partial x_{i}}f(p)}_{=g_{i}(p)} = \alpha(f)
.\]
\end{proof}

\subsection{Espacio tangente}

Motivados por la exposición anterior, podemos dar las siguientes definiciones,
notando $\Cinf(M) := \left\{ f\in(M\to\R) : f\text{ es diferenciable} \right\}$.

\begin{ndef}[Derivación]
  Se llama \emph{derivación (en $p$)} a un funcional $v: \Cinf(M)\to\R$ lineal y que
  respete la regla del producto en $p$.
\end{ndef}

\begin{ndef}[Espacio tangente]
  Sean $M$ una variedad diferenciable, y $p\in M$. Llamamos \emph{espacio
tangente de $M$ en $p$} al conjunto
  \[
    \T_{p}M := \left\{ v\in(\Cinf(M)\to\R) :
        v\text{ es una derivación en $p$}
\right\}
  .\]
A los elementos de $\T_{p}M$ se les llama \emph{vectores tangentes}.
\end{ndef}

Es inmediato comprobar que $\T_{p}M$ tiene la estructura de espacio vectorial
natural. Ahora, vamos a extender el conjunto de derivaciones que consideramos,
para poder obtener diferenciales de funciones que no estén definidas en toda la
variedad $M$, sino solamente en un abierto de la misma. Para ello, vamos a
servirnos del siguiente lema.
\begin{lema} \label{lema:derivacion-local}
  Se verifican:
  \begin{nlist}
  \item \label{item:dif-local}
    Sean $f,g\in\Cinf(M)$, $p\in M$ y $O$ un entorno abierto suyo, tales que
    $\restr{f}{O} = \restr{g}{O}$. Entonces, para cada $v\in \T_{p}M$, $v(f)=v(g)$.
  \item
    Si $c:M\to\R$ es constante, entonces $v(c) = 0$ para cada $v\in \T_{p}M$.
  \end{nlist}
\end{lema}
\begin{proof}
  \ref{item:dif-local}: sea $h: M\to\R$ la función dada por la proposición
  \ref{prop:meseta} para $p$ y $O$. Entonces, $h(f-g)$ es la función
  constantemente cero, por lo que
  \begin{align*}
    0 &= v(h(f-g)) \\
    & = v(h)\underbrace{(f-g)(p)}_{=0} + h(p)v(f-g) = v(f-g).
  \end{align*}
\end{proof}

Para considerar funciones que sean diferenciables en un entorno de $p$, no
tenemos por qué restringirnos a un entorno concreto. El siguiente conjunto
también goza de estructura de álgebra de manera natural:
\[
  \Cinf(p) := \left\{ f\in(O\to\R) : O\text{ es un entorno de $p$ y } f\text{ es diferenciable} \right\}
.\]
La suma y el producto vienen dados por la suma y el producto usuales de las
funciones restringidas a la intersección de sus dominios. Además, es claro que
$\Cinf(M)\subseteq\Cinf(p)$.

Este espacio nos permite considerar otra variante del espacio tangente:
\[
  \hat{\T}_{p}(M) := \left\{ v\in(\Cinf(p)\to\R) :
        v\text{ es una derivación en $p$}
\right\}
.\]
Pues bien,
\begin{nth}
  $\hat{\T}_{p}M$ y $\T_{p}M$ son isomorfos como espacios vectoriales.
\end{nth}
\begin{proof}
  Basta dar un isomorfismo $\Phi:\hat{\T}_{p}M\to \T_{p}M$. $\Phi(v) =
  \restr{v}{\Cinf(M)}$. En el otro sentido, para definir $\Phi^{-1}$, hemos de
  extender un funcional $\hat{v}: \Cinf(M)\to\R$ a $\Cinf(p)$. Para ello, dada
  una función $f\in\Cinf(p)$, la extendemos a una función $F\in\Cinf(M)$, y
  definimos $\Phi^{-1}(\hat{v})(f) = \hat{v}(F)$. El lema
  \ref{lema:derivacion-local} implica que esta definición no depende de la
  extensión tomada.

  Sea $O$ el dominio de $f$, y sea $h:M\to\R$ la dada por la proposicion
  \ref{prop:meseta} para $p$ y $O$. Definiendo $\restr{F}{O} = hf$ y
  $\restr{F}{M-O} = 0$, se tiene el resultado.
\end{proof}
A partir de este momento, notaremos por $\T_{p}M$ al espacio extendido
$\hat{\T}_{p}M$.

\paragraph{Vectores tangentes a curvas}
Ya estamos listos para generalizar un concepto que nos es familiar del estudio
de superficies diferenciables: el vector tangente a una curva.
\begin{ndef} \label{def:diferencial-curva}
  Sean $\alpha:(a,b)\to M$ diferenciable, y $t_{0}\in(a,b)$. El \emph{vector
    tangente a $\alpha$ en $t_{0}$} es la aplicación
    \begin{alignat*}{2}
      \alpha'(t_{0}): \Cinf(\alpha(t_{0})) & \to \R \\
      f & \mapsto \restr{\frac{\d}{\d t}}{t = t_{0}}(f\circ\alpha)(t).
    \end{alignat*}
\end{ndef}
Es inmediato comprobar que $\alpha'(t_{0})\in \T_{\alpha(t_{0})}M$.

\begin{nprop}[Cambio de parámetros]
  Sean $h:(c,d)\to(a,b)$ un difeomorfismo, y $\alpha:(c,d)\to M$
  diferenciable. Entonces,
  \[
    (\alpha\circ h)'(s_{0}) = h'(s_{0})\alpha'(h(s_{0}))
  .\]
\end{nprop}

Otro resultado esperable, al menos en parte, es el siguiente:
\begin{nth} \label{thm:base-parciales}
  $\T_{p}M$ tiene dimensión $n$ (la de la variedad). Dada una carta $(V,\phi)$
  con $p\in V$, una base es $\left\{ \diffop{x_{i}}{p} : i=1,\dots,n \right\}$, donde
  \begin{alignat*}{2}
    \diffop{x_{i}}{p}: \Cinf(p)  & \to \R \\
    f & \mapsto \frac{\partial}{\partial r_{i}}(f\circ \phi^{-1})(q),
  \end{alignat*}
  $q = \phi(p)$, y $\frac{\partial}{\partial r_{i}}$ son las derivadas parciales usuales.
\end{nth}
\begin{proof}
  Solo es necesario probar la segunda afirmación. Probamos que es un sistema
  linealmente independiente. Llamando $x_{i} = r_{i}\circ\phi$, con $r_{i}$ las
  proyecciones de $\R^{n}$:
  \[
    \left( \sum_{i=1}^{n} \lambda_{i} \diffop{x_{i}}{p} \right)(x_{j}) = \lambda_{j}
  ,\]
  por lo que si los coeficientes son tales que $\sum_{i=1}^{n} \lambda_{i}\diffop{x_{i}}{p}
  = 0$, han de ser nulos todos.

  Ahora, veamos que es sistema de generadores. Sean $v\in \T_{p}M$, y
  $f\in\Cinf(p)$. Desarrollamos
  \[
    (f\circ \phi^{-1})(r) = (f\circ \phi^{-1})(q) + \sum_{i=1}^{n}(r_{i}-q_{i})g_{i}(r)
  \]
  en una bola alrededor de $q$. Llamando $\phi^{-1}(r) = x$, obtenemos el
  desarrollo
  \[
    f(x) = f(p) + \sum_{i=1}^{n} (x_{i}(x) - q_{i})(\underbrace{g_{i}\circ\phi}_{\hat{g}_{i}})(x)
  .\]
  Dado este desarrollo, vemos que $v$ actúa sobre $f$ de la forma
  \begin{align*}
    v(f) & = \sum_{i=1}^{n}\left( v(x_{i}-q)\hat{g}_{i}(p) +
           \underbrace{(x_{i}(p) - q_{i})}_{=0}v(\hat{g}_{i}) \right) \\
    & = \sum_{i=1}^{n} v(x_{i})\hat{g}_{i}(p).
  \end{align*}
  Solo queda ver que $v(x_{i})$ son las coordenadas de $v$ respecto de este
  sistema:
  \begin{align*}
    \left[ \sum_{i=1}^{n} v(x_{i}) \diffop{x_{i}}{p} \right](f) & =
                                                                  \sum_{i=1}^{n}
                                                                  v(x_{i})
                                                                  \diffop{x_{i}}{p}(f)
    \\
    & = \sum_{i=1}^{n} v(x_{i}) \underbrace{\sum_{j=1}^{n} \diffop{x_{i}}{p}(x_{j})
    \hat{g}_{j}(p)}_{\delta_{ij}\hat{g}_{j}(p)} \\
    & = \sum_{i=1}^{n} v(x_{i})\hat{g}_{i}(p).
  \end{align*}
\end{proof}

De la demostración anterior se desprende cuáles son las coordenadas de un vector
tangente respecto de una base de este tipo. En particular, cuáles son las
coordenadas de los elementos de otra base.
\begin{notacion}
  En ocasiones, notaremos a una carta $(V,\phi,x_{i})$ para indicar
  explícitamente sus coordenadas $x_{i} = r_{i}\circ \phi$.
\end{notacion}
\begin{nprop}[Cambio de coordenadas]
  Dadas dos cartas $(V, \phi, x_{i})$, $(W,\psi,y_{i})$, y sus bases dadas por
  el teorema \ref{thm:base-parciales} $B$ y $B'$, la matriz de cambio de base
  entre ellas es
  \[
    M(B,B') = \left( \diffop{y_{i}}{p}(x_{i}) \right)_{ij}
  .\]
\end{nprop}

\begin{nprop}
  Para cada $v\in \T_{p}M$, existe $\alpha: (-\epsilon,\epsilon)\to M$ curva
  diferenciable tal que $\alpha(0)=p$, $\alpha'(0)=v$.
\end{nprop}
\begin{proof}
  Sea $(V,\phi,x_{i})$ una carta en $p$, entonces $v = \sum_{i=1}^{n} v(x_{i})
  \diffop{x_{i}}{p}$. Llamando $w := (v(x_{1}), \dots, v(x_{n}))\in
  \R^{n}$, y definiendo $\alpha(t) = \phi^{-1}(\phi(p)+tw)$,
  \begin{align*}
    \alpha'(0)(f) & = \restr{\frac{\d}{\d t}}{t = 0}(f\circ \alpha)(t) \\
    & = \restr{\frac{\d}{\d t}}{t = 0}(f\circ \phi^{-1})(\phi(p)+tw) \\
    & = \sum_{i=1}^{n} w_{i} \underbrace{\frac{\partial}{\partial r_{i}}(f\circ\phi^{-1})(\phi(p))}_{\diffop{x_{i}}{p}(f)}.
  \end{align*}
\end{proof}

\subsection{Diferencial de una aplicación diferenciable}

Pasamos ya a definir la diferencial de una aplicación.
\begin{ndef}[Diferencial de una aplicación]
  Sea $f:M^{m}\to N^{n}$ diferenciable. Si $p\in M$, entonces la \emph{diferencial de
  $f$ en $p$}, que notamos $\d f_{p}: \T_{p}M\to \T_{f(p)}N$, se define como
  \[
    (\d f_{p}(v))(g) = v(g\circ f) .\]
\end{ndef}

La siguiente proposición nos permite asegurar que la diferencial está bien definida.
\begin{nprop}
  Para cada $p\in M$, $\d f_{p}$ es una derivación.
\end{nprop}
\begin{proof}
  \begin{align*}
    (\d f_{p})(g_{1}g_{2}) &= v((g_{1}g_{2})\circ f) \\
                           & = v(g_{1}\circ f)g_{2}(f(p)) + g_{1}(f(p))v(g_{2}\circ f)\\
    & = \d f_{p}(g_{1})g_{2}(f(p)) + g_{1}(f(p))\d f_{p}(g_{2}).
  \end{align*}
\end{proof}
\begin{nprop}[Propiedades de la diferencial] \label{prop:propiedades-diferencial}
  Se verifican:
  \begin{nlist}
  \item
    La diferencial es lineal,
  \item
    si $f$ es constante, entonces $\d f_{p} = 0$ para cada $p\in M$,
  \item
    $\d (\mathrm{id})_{p} = \mathrm{id}$,
  \item \label{item:cadena}
    regla de la cadena: $\d (G\circ F)_{p} = \d G_{F(p)}\circ \d F_{p}$,
  \item
    si $F: M^{n}\to N^{n}$ es un difeomorfismo, entonces para cada $p\in M$, $\d F_{p}$
    es un isomorfismo y su inversa es $(\d F_{p})^{-1} = \d F^{-1}_{p}$,
  \item \label{item:generaliza-dif}
    Sea $F: \R^{n}\to \R^{m}$ diferenciable. Notando por $\d F_{p}^{*}$ a la
    diferencial en el sentido clásico del análisis real, el siguiente diagrama es
    conmutativo:
    \[
      \begin{tikzcd}
        \R^{n} \arrow{d}{\lambda_{p}} \arrow{r}{\d F_{p}^{*}} & \R^{m} \arrow{d}{\lambda_{F(p)}} \\
        \T_{p}\R^{n} \arrow{r}{\d F_{p}} & \T_{F(p)}\R^{m}
      \end{tikzcd}
    \]
    donde $\lambda_{p}$ es el isomorfismo definido al comienzo de esta sección.
  \item
    si $f$ es diferenciable, $p\in M$, $v\in \T_{p}M$, y $\alpha:
    (-\epsilon,\epsilon)\to M$ es cualquier curva que verifique $\alpha(0) = p$
    y $\alpha'(0) = v$, entonces
    \[
      \d f_{p}(v) = \restr{\frac{\d}{\d t}}{t = 0}(f\circ\alpha)(t)
      ,\]
    entendiendo la derivada a la derecha de la igualdad en el sentido de la definición \ref{def:diferencial-curva}.
  \end{nlist}
\end{nprop}
\begin{proof}
  \ref{item:cadena}:
  \begin{align*}
    \d (G\circ F)_{p}(v)(g) &= v(g\circ G\circ F) \\
                            &= (\d F_{p})(v)(g\circ G) \\
    &= \d G_{F(p)}(\d F_{p}(v))(g).
  \end{align*}

  \ref{item:generaliza-dif}: 
  Por un lado,
  \begin{align*}
    \lambda_{F(p)}(\d F_{p}^{*}(v)) & = (g \mapsto \d g_{F(p)} (\d F_{p}^{*}(v)))
    \\
    & = (g \mapsto \d (g\circ F)_{p}(v)).
  \end{align*}
  Por otro,
  \begin{align*}
    \d F_{p}(\lambda_{p}(v)) & = (g \mapsto \lambda(v)(g\circ F)) & \text{por
                                                                definición de
                                                                $\d F_{p}$} \\
    & = (g \mapsto \d (g\circ F)_{p}(v)) & \text{por definición de $\lambda_{p}$}.
  \end{align*}
\end{proof}

\subsection{Expresión matricial de la diferencial}

Fijadas dos cartas $(V,\phi,x_{i})$ con $p\in M$, $(W,\psi,y_{j})$ con $F(p)\in
N$, hemos visto que podemos obtener sendas bases de los espacios tangentes: la
base de $\T_{p}M$
\[
  \left\{ \diffop{x_{i}}{p} : i=1,\dots,m \right\}
,\]
y la base de $\T_{F(p)}N$
\[
  \left\{ \diffop{y_{j}}{f(p)} : j=1,\dots,n \right\}
.\]
Esto nos permite expresar la diferencial $\d F_{p}$ matricialmente:
\begin{align*}
\d F_{p}\diffop{x_{i}}{p} &= \sum_{j=1}^{n} \left( \d F_{p}\diffop{x_{i}}{p}
                              \right)(y_{j})\diffop{y_{j}}{F(p)} \\
  & = \sum_{j=1}^{n}\diffop{x_{i}}{p}(y_{j}\circ F)\diffop{y_{j}}{F(p)},
\end{align*}
por lo que la matriz jacobiana asociada es
\[
  \d F_{p} =
  \begin{pmatrix}
    \diffop{x_{i}}{p}(y_{j}\circ F)
  \end{pmatrix}_{ij}
.\]

\subsection{El espacio tangente en subvariedades del euclídeo} \label{sec:tan-subv}

En esta sección vamos a establecer una forma práctica de trabajar con el espacio
tangente de subvariedades del euclídeo. Para ello, lo identificaremos con un
cierto subespacio de $\R^{N}$.

Hemos visto que si $M^{n}\subset \R^{N}$ es una subvariedad, entonces la
inclusión es diferenciable. Además, su diferencial es inyectiva:
\begin{nprop}
  Para cada $p\in M$, $\d i_{p}$ es inyectiva.
\end{nprop}
\begin{proof}
  Sea $(V,\phi,x_{i})$ una carta en $p$, y $X: U\subseteq \R^{n}\to V\subseteq
  \R^{N}$ una parametrización tal que $\restr{X^{-1}}{V\cap M} = \phi$.
  \[
    \d X_{\phi(p)} = \d
      i_{p} \circ \d \phi^{-1}_{\phi(p)},
  \]
  con $\d X_{\phi(p)}$ inyectiva y $\d \phi^{-1}_{\phi(p)}$ un isomorfismo.
\end{proof}

Esto nos da la siguiente cadena de aplicaciones lineales:
\[
  \begin{tikzcd}
    \T_{p}M \arrow{r}{\d i_{p}} & \T_{p}\R^{N} \arrow{r}{\lambda_{p}^{-1}} & \R^{N}
  \end{tikzcd}
\]
Trabajaremos identificando en lo sucesivo $\T_{p}M$ con $V_{p} :=
\lambda^{-1}_{p}(\d i_{p}(T_{p}M))$ mediante $\lambda_{p}^{-1}\circ \d i_{p}$,
que es una aplicación lineal inyectiva, y por tanto ambos espacios tienen la
misma dimensión.

Una observación adicional que será útil:
\begin{nprop} \label{prop:inversa-lambda}
  La inversa de $\lambda_{p}$ viene dada por
  \begin{alignat*}{2}
    \lambda_{p}^{-1}: \T_{p}\R^{N} & \to \R^{N} \\
   v & \mapsto (v(r_{1}),\dots,v(r_{N})),
 \end{alignat*}
 con $r_{i}$ las proyecciones usuales de $\R^{N}$.
\end{nprop}

Esta identificación es natural en el siguiente sentido: sea $v\in \T_{p}M$, y
sea $\alpha:(-\epsilon,\epsilon)\to M$ una curva diferenciable con $\alpha(0)=p$
y $\alpha'(0)=v$. Podríamos haber elegido identificar $v$ con el vector tangente
a la curva vista como una curva en $\R^{N}$, es decir, con $(i\circ\alpha)'(0)$,
en el sentido de la derivada clásica del análisis. Pues bien, esto es
equivalente a la identificación que hemos hecho:
\begin{align*}
  (i\circ\alpha)'(0) & = \restr{\frac{\d}{\d t}}{t = 0}(i\circ\alpha)(t) \\
                     & = (\alpha'(0)(i_{1}),\dots,\alpha'(0)(i_{N})) & \text{en
                                                                       el
                                                                       sentido
                                                                       de \ref{def:diferencial-curva}} \\
                     & = (v(i_{1}),\dots,v(i_{N})) \\
                     & = \lambda_{p}^{-1}(\d i_{p}(v)) & \text{por la
                                                          proposición \ref{prop:inversa-lambda}}.
\end{align*}

En suma, esta identificación nos permite determinar el espacio tangente mediante
el siguiente procedimiento: consideramos curvas diferenciables en $M$ que pasen
por $p$ en $0$, y las derivamos de la manera usual, tratándolas como curvas en
el espacio ambiente. Los vectores que obtengamos de esta forma comformarán el
espacio tangente.

\begin{ejemplo}[Esfera]
Calculemos el espacio tangente de la esfera (ejemplo \ref{ejemplo:esfera}). Sea
$\alpha :(-\epsilon,\epsilon)\to\mathbb{S}^{n}(a,R)$. Entonces, $\alpha$
verifica la ecuación $\|\alpha-a\|^{2} = R^{2}$. Derivando,
\[
  \langle \alpha'(0), \alpha(0)-a \rangle = \langle v, p-a \rangle = 0
,\]
lo que nos da la condición $V_{p} \perp p-a$. Por tanto, $V_{p}\subseteq
(p-a)^{\perp}$, pero $(p-a)^{\perp}$ tiene dimensión $n$, luego se da la igualdad.
\end{ejemplo}

\begin{ejemplo}[Grupo ortogonal]
  Recordamos el grupo ortogonal del ejemplo \ref{ejemplo:ortogonal}. Pues bien,
  sea $A$ una curva diferenciable en $O(n)$, con $A(0) = I$ y $A'(0)=B$. $A$
  verifica
  \[
    AA^{T} = I
  ,\]
luego derivando
\[
  A'(0)A(0)^{T}+A(0)A'(0)^{T} = B+B^{T} = 0
.\]
Razonando como en el ejemplo anterior, el espacio tangente a $O(n)$ en $I$ es el
espacio de las matrices antisimétricas.

Para obtener el espacio tangente en una matriz $A\in O(n)$ cualquiera, vamos a
emplear una técnica que se puede extender a cualquier grupo de Lie. Consideramos
las traslaciones a izquierda y derecha $l_{A},r_{A}: O(n)\to O(n)$ dadas por
$l_{A}(B) = AB$ y $r_{A}(B) = BA$. Es inmediato que son difeomorfismos. Por
tanto, $(\d l_{A})_{I}, (\d r_{A})_{I}: \T_{I}O(n)\to \T_{A}O(n)$ son isomorfismos.
Ahora,
\[
  (\d l_{A})_{I}(B) = AA'(0) = AB
,\]
luego
\[
  \T_{A}O(n) = \left\{ AB : B+B^{T} = 0 \right\}
.\]
\end{ejemplo}

\begin{ejemplo}
  Razonando como en el ejemplo anterior, podemos llegar a
  \[
    \T_{A}\Sl(n,\R) = \left\{ AB : \traza(B) = 0 \right\}
  .\]
\end{ejemplo}

\begin{ejemplo}
  Retomamos la notación del ejemplo \ref{ejemplo:proyectivo}. Vamos a hallar el
  espacio tangente a un punto $p=\pi(x)$, con $x\in O_{i}$, a través de la
  diferencial
  \[
      \d\pi_{x}: \T_{x}\R^{n+1}  \to \T_{p}\RP^{n}.
  \]
  Como nace en un espacio de dimensión $n+1$ y toma valores en uno de dimensión
  $n$, es obvio que tiene núcleo. Calculémoslo.

  Sea $v\in\ker(\d\pi_{x})$, lo cual se verifica si y solo si
  \[
    v(f\circ\pi) = 0
  \]
  para cada $f\in\Cinf(p)$. Consideramos $f = r_{j}\circ \phi_{i}$, con $i\ne
  j$. Entonces
  \begin{align*}
    (f \circ\pi)(x_{1}, \dots, x_{n+1}) & =
                                          (r_{j}\circ\psi_{i})(x_{1},\dots,x_{n+1})
    \\
    & = \frac{x_{j}}{x_{i}}.
  \end{align*}
  Veamos qué vale $v(x_{j}/x_{i})$:
  \begin{align*}
    v \left( \frac{x_{j}}{x_{i}} \right) & = \left( \sum_{k=1}^{n+1} v(x_{k})
                                           \diffop{x_{k}}{x} \right)\left(
                                           \frac{x_{j}}{x_{i}} \right) \\
    & = \sum_{k=1}^{n+1} v(x_{k}) \left( \frac{x_{i}\delta_{jk} -
      x_{j}\delta_{ik}}{x_{i}^{2}} \right) \\
                                         & = \frac{v(x_{j})}{x_{i}} -
                                           \frac{v(x_{i})x_{j}}{x_{i}^{2}} \\
                                           & = 0.
  \end{align*}

  Esto implica que, si notamos $v = (v_{1}, \dots, v_{n+1})$ en coordenadas de
  $\left\{ \diffop{x_{k}}{x} \right\}_{k=1}^{n+1}$, $v =
  \frac{v_{i}}{x_{i}}x$. Por tanto, en dichas coordenadas,
  \[
    \ker(\d\pi_{x}) = \langle x  \rangle 
    .\]

  Ahora, aplicando el primer teorema de isomorfía para espacios vectoriales,
  \[
    \frac{\T_{x}\R^{n+1}}{\ker(\d\pi_{x})} \cong \im(\d\pi_{x}) = \T_{p}\RP^{n}
    .\]
  Por otro lado, mediante el isomorfismo
  \begin{alignat*}{2}
       \frac{\T_{x}\R^{n+1}}{\ker(\d\pi_{x})} &\to \ker(\d\pi_{x})^{\perp} \\
    z + \ker(\d\pi_{x}) & \mapsto P_{\ker(\d\pi_{x})^{\perp}}(z),
  \end{alignat*}
  con $P_{\ker(\d\pi_{x})^{\perp}}$ la proyección ortogonal sobre
  $\ker(\d\pi_{x})^{\perp}$, podemos identificar
  \[
    \T_{p}\RP^{n} \cong \ker(\d\pi_{x})^{\perp} \cong \langle x  \rangle^{\perp}
  .\]
\end{ejemplo}

\subsection{Difeomorfismos locales}

Vamos a extender el concepto de difeomorfismo local al contexto de
diferenciabilidad entre variedades.
\begin{ndef}
  $F:M^{m}\to N^{m}$ es un \emph{difeomorfismo local} si, para cada $p\in M$,
  existen entornos abiertos $O$ y $O'$ de $p$ y $F(p)$ tal que $F: O \to O'$ es
  un difeomorfismo.
\end{ndef}

% TODO: ejemplo: proyección de la esfera al proyectivo

\begin{nprop}
  Si, para cada $p\in M$, $\d F_{p}$ es un isomorfismo, entonces $F$ es un
  difeomorfismo local.
\end{nprop}
\begin{proof}
  La demostración se basa en trasladar el problema a abiertos de $\R^{m}$
  mediante cartas en $p$ y $F(p)$ y aplicar allí el teorema de la función
  inversa. Sean $(V, \phi)$ y $(W, \psi)$ cartas en $p$ y $F(p)$. Entonces, las
  diferenciales hacen conmutativo el diagrama
  \[
  \begin{tikzcd}
    \T_{p}M \arrow{r}{\d F_{p} (\cong)} \arrow{d}{\d\phi_{p} (\cong)} &
    \T_{F(p)}N \arrow{d}{\d\psi_{F(p)} (\cong)} \\
    \R^{m} \arrow{r}{\d(\psi\circ F \circ\phi^{-1})} & \R^{m}
  \end{tikzcd}
\]
luego todas las aplicaciones del diagrama son isomorfismos. Aplicando el teorema
de la función inversa a $\psi\circ F \circ\phi^{-1}$ y notando que $\psi$ y
$\phi$ son difeomorfismos se obtiene el resultado.
\end{proof}

\subsection{Particiones de la unidad}

Vamos a estudiar el concepto de \emph{partición de la unidad}, que nos servirá
para extender funciones diferenciables en variedades.

A partir de ahora, exigiremos a las variedades que consideremos que sean, además
de Haussdorf, segundo axioma de numerabilidad, es decir, que su topología posea
una base numerable. Esta propiedad se hereda en subespacios, productos y
funciones continuas y sobreyectivas.

Pues bien, pasamos a definir el concepto:
\begin{ndef}[Partición de la unidad]
  Una \emph{partición de la unidad en $M$} es una familia $\left\{ \theta_{n}:
    M\to \R \right\}_{n\in\N}$ de funciones diferenciables que verifican:
  \begin{nlist}
  \item
    \emph{Finitud local}: para cada $p\in M$, existe un entorno abierto $O$ de
    $p$ tal que $\sop(\theta_{n})\cap O = \emptyset$ excepto para un subconjunto
    finito de la familia,
  \item
    $0\leq \theta_{n}\leq 1$ para cada $n\in\N$,
  \item
    $\sum_{n=1}^{\infty} \theta_{n} = 1$.
  \end{nlist}
\end{ndef}

Los siguientes lemas nos permitirán probar la existencia de particiones de la unidad
no triviales (del tipo $\left\{ 1/2, 1/2 \right\}$) y justificarán el requisito
de ser ANII.
\begin{lema} \label{lema:base-rel-compactos}
  Sea $M$ una variedad diferenciable. Entonces, la topología de $M$ admite una
  base numerable de abiertos relativamente compactos.
\end{lema}
\begin{proof}
  Sea $\mathscr{B}$ una base numerable de la topología de $M$. Definimos
  \[
    \mathscr{B}' = \left\{ B\in \mathscr{B} : \overline{B} \text{ es compacto} \right\}
  ,\]
  y vamos a ver que es también una base de la topología.

  Sean $p\in M$, $V\subseteq O\subseteq M$ abiertos con $p\in V$, y $(V,\phi)$
  una carta de $M$. Podemos suponer que $V$ es difeomorfo a una bola y que
  $\overline{V}\subset O$, luego $\overline{V}$ es compacto.

  Por ser $\mathscr{B}$ base, existe $B\in\mathscr{B}$ con $p\in B\subseteq V$,
  luego $\overline{B}\subseteq\overline{V}$, compacto, es decir, $B\in\mathscr{B}'$.
\end{proof}
\begin{lema} \label{lema:rel-compactos}
  Sea $M$ una variedad diferenciable. Entonces, existe una familia
  $\left\{ G_{n} \right\}_{n\in\N}$ de subconjuntos de $M$ que verifica:
  \begin{nlist}
  \item
    $\overline{G}_{n}$ es compacto para cada $n\in\N$,
  \item
    $\overline{G}_{n}\subseteq G_{n+1}$ para cada $n\in\N$,
  \item
    $\bigcup_{n=1}^{\infty} G_{n} = M$.
  \end{nlist}
\end{lema}
\begin{proof}
  Sea, por el lema \ref{lema:base-rel-compactos},
  $\mathscr{B} = \left\{ B_{n} \right\}_{n\in\N}$ una base numerable de la
  topología compuesta por abiertos relativamente compactos. Definimos,
  inductivamente:
  \begin{nlist}
  \item
    $G_{1} := B_{1}$,
  \item
    supuesto que $G_{n}$ está definido, observamos que $\mathscr{B}$ recubre a
    $\overline{G_{n}}$, y tomamos un subrecubrimiento finito $\left\{ B_{n_{k}}
    \right\}_{k=1}^{N}$. Entonces, definimos
    \[
      G_{n+1} := \bigcup_{k=1}^{N} B_{n_{k}} \cup B_{n}
    .\]
  \end{nlist}
\end{proof}

El siguiente teorema nos da particiones de la unidad no triviales.
\begin{nth} \label{thm:part-recub}
  Sean $M$ una variedad diferenciable, y
  $U = \left\{ U_{\alpha} \right\}_{\alpha\in A}$ un recubrimiento abierto de
  $M$. Entonces, existe una partición de la unidad
  $\left\{ \theta_{n} \right\}_{n\in\N}$ de forma que se verifica, para cada
  $n\in\N$:
  \begin{nlist}
  \item
    $\theta_{n}$ tiene soporte compacto,
  \item
    existe $\alpha_{n}\in A$ tal que $\sop(\theta_{n})\subseteq U_{\alpha_{n}}$.
  \end{nlist}
\end{nth}
\begin{proof}
  Sea $\left\{ G_{n} \right\}_{n\in\N}$ una familia dada por el lema
  \ref{lema:rel-compactos}. Además, llamemos $G_{0} = \emptyset$. Consideramos
  \[
    A_{n} := \overline{G_{n+1}}-G_{n}
    .\]
  Observamos:
  \begin{nlist}
  \item
    $A_{n}$ es un cerrado en un compacto, luego compacto,
  \item
    $\left\{ A_{n} \right\}_{n\in\N}$ recubre a $M$,
  \item
    $A_{n} = \overline{G_{n+1}}-G_{n} \subseteq G_{n+2}-\overline{G_{n-1}}$, que
    es abierto.
  \end{nlist}
  Ahora, si $p\in A_{n}$, entonces existe $\alpha_{p}$ con $p\in
  U_{\alpha_{p}}$, luego $p\in (G_{n+2}-\overline{G_{n-1}})\cap U_{\alpha_{p}}$.
  Por el lema \ref{prop:meseta}, existen $f_{p}: M\to \R$ una función meseta y
  $V_{p}$ un entorno abierto relativamente compacto de $p$ de manera que:
  \begin{nlist}
  \item
    $\sop(f_{p}) \subseteq (G_{n+2}-\overline{G_{n-1}})\cap U_{\alpha_{p}}$ y es
    compacto,
  \item
    $\restr{f_{p}}{V_{p}} = 1$.
  \end{nlist}

  De este modo, $\left\{ V_{p} : p\in A_{n} \right\}$ es un recubrimiento
  abierto de $A_{n}$. Tomamos un subrecubrimiento finito que recubra a $A_{n}$,
  y las funciones correspondientes. Repitiendo esta construcción para cada
  $n\in\N$, obtenemos --recordamos que la unión numerable de conjuntos
  numerables es numerable-- una familia numerable de funciones diferenciables
  $\left\{ f_{n} \right\}_{n\in\N}$, y cada una verifica las propiedades
  anteriores. Además, sus soportes recubren a $M$. Podemos definir ya
  \[
    \theta_{n} = \frac{f_{n}}{\sum_{m=1}^{\infty} f_{m}}
    .\]
  Es inmediato comprobar que la suma del denominador es finita y no nula, y que
  la familia así definida verifica las propiedades del enunciado.
\end{proof}

\begin{ncor}
  Sean $F\subseteq M$ cerrado y $O\subseteq M$ abierto, con $F\subseteq
  O$. Entonces, existe $f: M\to \R$ diferenciable con $0\leq f \leq 1$,
  $\restr{f}{F} = 1$, y $\sop(f) \subseteq O$.
\end{ncor}
\begin{proof}
  Tomamos el recubrimiento $U = \left\{ M-F, O \right\}$ de $M$, y $\left\{
    \theta_{n} \right\}_{n\in\N}$ la partición de la unidad dada por el teorema
  para $U$. Consideramos $\left\{ \theta_{k} : k\in I \right\}$ la subfamilia de
  las funciones que verifican
  \begin{nlist}
  \item
    $\sop(\theta_{k})\not\subseteq M-F$,
  \item
    $\sop(\theta_{k})\cap F \ne \emptyset$.
  \end{nlist}
  Por tanto, verifican $\sop(\theta_{k})\subseteq O$. Definimos
  \[
    f := \sum_{k\in I} \theta_{k}
    .\]
  Claramente, $0\leq f \leq 1$ y $\sop(f)\subseteq O$. Además,
  \[
    \restr{f}{F} = \sum_{k\in I} \restr{\theta_{k}}{F} = \sum_{n\in\N} \restr{\theta_{n}}{F} = 1
    ,\]
  pues el soporte de las que se añaden en la segunda sumatoria está en $M-F$.
\end{proof}

El siguiente corolario nos dice que, en cierto sentido, no hay funciones
diferenciables exclusivas de las subvariedades.
\begin{lema} \label{lema:extension-subv}
  Sean $p\in M$, y $f:M\to \R$ diferenciable. Existen un entorno abierto $W_{p}\subseteq \R^{N}$ de $p$ y una
  función diferenciable $\tilde{f}_{p}: W_{p}\to \R$ tales que
  $\restr{\tilde{f_{p}}}{W_{p}\cap M} = \restr{f}{W_{p}\cap M}$.
\end{lema}
\begin{proof}
  Vamos a usar una construcción como la que se empleó en la demostración de la
  proposición \ref{prop:prop-dif-sub}-\ref{item:inc-2}. Sea $(V,\phi)$ una carta
  de $M$ con $p\in V$, y obtenemos, como se hizo allí, una carta $(W_{p},\psi)$ de $\R^{N}$ tal
  que $\psi\circ i\circ\phi^{-1} = (x\mapsto (x,0))$.

  Definimos
  \[
    \tilde{f}_{p} = f\circ \phi^{-1} \circ\pi\circ\psi
    ,\]
  con $\pi$ la proyección sobre las primeras $n$ componentes, y así
  $\tilde{f}_{p}$ es diferenciable. Además, si $v\in V$, entonces
  \[
    (\psi\circ i)(v) = (\psi\circ i \circ\phi^{-1})(\phi(v)) = (\phi(v), 0)
    ,\]
  luego
  \[
    (\pi\circ\psi\circ i)(v) = \phi(v)
    ,\]
  y por último
  \[
    (\phi^{-1}\circ\pi\circ\psi\circ i)(v) = v
    .\]
  Es decir, $\restr{\phi^{-1}\circ\pi\circ\psi}{V} = \mathrm{id}$. Por tanto,
  $\restr{\tilde{f}_{p}}{W_{p}\cap M} = \restr{f}{W_{p}\cap M}$.
\end{proof}
\begin{ncor} \label{cor:ext-fun-subv}
  Sean $M\subseteq \R^{N}$ una subvariedad del euclídeo, cerrada en $\R^{N}$, y
  $f:M\to\R$ diferenciable. Entonces, existe $F:\R^{N}\to\R$ diferenciable que
  extiende a $f$.
\end{ncor}
\begin{proof}
  Sean, para $p\in M$, $W_{p}$ el entorno de $p$ y $\tilde{f}_{p}$ la función
  dados por el lema \ref{lema:extension-subv}. Consideramos el recubrimiento de
  $\R^{N}$:
  \[
    U := \left\{ W_{p} \right\}_{p\in M} \cup \left\{ \R^{N}-M \right\}
    .\]
  Aplicamos el teorema \ref{thm:part-recub} y obtenemos una partición de la
  unidad $\left\{ \theta_{n} \right\}_{n\in\N}$ para $U$. De esta familia,
  consideramos la subfamilia $\left\{ \theta_{k} \right\}_{k\in I}$ de las
  $\theta_{k}$ que verifiquen $\sop(\theta_{k})\cap M \ne \emptyset$, lo que
  implica $\sop(\theta_{k}) \subseteq w_{p_{k}}$ para cierto $p_{k}\in M$.

  Definimos, para cada $k\in I$,
  \[
    \hat{f}_{k} :=
    \begin{cases}
      \theta_{k}\tilde{f}_{p_{k}} & \text{ en $W_{p_{k}}$}, \\
      0 & \text{ en $\R^{N}-W_{p_{k}}$},
    \end{cases}
  \]
  y $F := \sum_{k\in I} \hat{f}_{k}$. Es inmediato que es una extensión
  diferenciable de $f$.
\end{proof}

\section{Campos de vectores}

En esta sección vamos a abordar el concepto de \emph{campo de vectores}, que
consistirá en una aplicación que asigne a cada punto de una variedad
diferenciable un vector tangente a la variedad en ese punto.
\begin{ndef}[Fibrado tangente]
  Llamamos \emph{fibrado tangente} de una variedad diferenciable $M$ a:
  \[
    \T M = \bigsqcup_{p\in M} \T_{p}M
  .\]
\end{ndef}

También se puede definir el fibrado tangente con la unión usual --no
disjunta--. En este caso, el cero es común a todos los tangentes, pero salvo
esta excepción, la unión es disjunta.
\begin{lema} \label{lema:union-subesp}
  Sean $V$ un espacio vectorial, y $U$ y $W$ subespacios propios. Entonces,
  $U\cup W\ne V$.
\end{lema}
\begin{proof}
  Razonamos por contradicción, suponiendo que $U\cup W = V$. Observamos que
  $U\cap W \ne U$: si se da la igualdad, $U\subseteq W$, luego $U\cup W = W \ne
  V$. Análogamente para $W$. Tomamos $v\in U -U\cap W$, $v'\in W - U\cap W$.

  Entonces, $v+v'\in V$, luego $v+v'\in U$ o $v+v'\in W$. Sin pérdida de
  generalidad, suponemos que $v+v'\in U$. Entonces, $v+v'-v = v'\in U$,
  contradicción.
\end{proof}
\begin{nprop}
  Si $M$ es una variedad diferenciable, $p,q\in M$, y $p\ne q$, entonces
  $\T_{p}M \cap \T_{q}M = \{0\}$.
\end{nprop}
\begin{proof}
  Sea $v: \Cinf(M)\to\R$ lineal y no nula, y supongamos que $v\in \T_{p}M$,
  $v\in \T_{q}M$. Entonces, para cada $f,g\in\Cinf(M)$,
  \begin{align*}
    v(fg) & = v(f)g(p) + f(p)v(g) \\
    & = v(f)g(q) + f(q)v(g),
  \end{align*}
  luego
  \[
    v(f)(g(p)-g(q)) + v(g)(f(p)-f(q)) = 0
    .\]
  En particular, tomando $f=g$,
  \[
    2v(f)(f(p)-f(q)) = 0
  \]
  para cada $f\in\Cinf(M)$, luego, si $f\in\Cinf(M)$, entonces
  \begin{equation}
    \label{eq:case-split-1}
    \tag{$*$}
    \begin{cases}
      f\in\ker(v) & \text{ o} \\
      f(p) = f(q).
    \end{cases}
  \end{equation}
  Definimos $L:\Cinf(M)\to\R$ por $L(f) = f(p)-f(q)$, lineal. Podemos traducir
  \eqref{eq:case-split-1} como
  \[
    \ker(v)\cup\ker(L) = \Cinf(M)
    .\]
  
  Ahora, por ser dos funcionales lineales no nulos, ninguno de sus núcleos es el
  espacio completo. Por tanto, su unión no puede ser el espacio completo (lema
  \ref{lema:union-subesp}), y hemos llegado a contradicción.
\end{proof}

La siguiente proyección será relevante al definir los campos de vectores.
\begin{ndef}
  La \emph{proyección natural} del fibrado tangente de una variedad es
  \begin{alignat*}{2}
    \Pi: \T M & \to M \\
   (p,v) & \mapsto p.
  \end{alignat*}
\end{ndef}

\begin{ndef}[Campo de vectores]
  Un campo de vectores en $M$ es una aplicación
  \begin{alignat*}{2}
    X: M & \to \T M \\
   p & \mapsto X_{p}
  \end{alignat*}
  que verifica $\Pi\circ X=\mathrm{id}$, es decir, $X_{p}\in \T_{p}M$ para cada
  $p\in M$.
\end{ndef}

Consideraremos, en el conjunto de los campos de vectores, estructuras de espacio
de vectorial y de módulo sobre las funciones $(M\to \R)$, de la siguiente forma:
\begin{nlist}
\item
  el producto escalar y la suma: $(aX+bY)_{p} = aX_{p}+bY_{p}$,
\item
  el producto por funciones $f: M\to\R$: $(fX)_{p} = f(p)X_{p}$.
\end{nlist}

\begin{ejemplo}
  Una carta $(V,\phi,x_{i})$ en $M$ permite dar un campo en $V$ mediante
  \begin{alignat*}{2}
    \Diffop{x_{i}}: V  & \to \T M \\
    p & \mapsto \diffop{x_{i}}{p}.
  \end{alignat*}
\end{ejemplo}
Este campo de vectores nos permite expresar, localmente, cualquier otro campo en
coordenadas respecto de $\left\{ \Diffop{x_{i}} : i=1,\dots,n \right\}$: si $X:
V \to \T M$ es un campo de vectores, entonces
\[
  \sum_{i=1}^{n} a_{i}(p)\diffop{x_{i}}{p}
\]
para ciertos coeficientes $a_{i}(p)$. Así, estos coeficientes definen funciones
$a_{i}: V\to\R$, con lo cual
\[
  X_{p} = \sum_{i=1}^{n}\left( a_{i}\Diffop{x_{i}} \right)_{p} = \left(
    \sum_{i=1}^{n} a_{i} \Diffop{x_{i}} \right)_{p}
,\]
es decir, $X = \sum_{i=1}^{n}a_{i}\Diffop{x_{i}}$.
