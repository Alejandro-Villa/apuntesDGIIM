\section{Aplicaciones diferenciables entre variedades}

\begin{ejer} \label{ejer:1}
  Sean $M_{1}^{n_{1}}$ y $M_{2}^{n_{2}}$ variedades diferenciables. Probar:
  \begin{nlist}
  \item \label{ejer-item:proy}
    Las proyecciones $\pi_{1}$ y $\pi_{2}$ de la variedad producto $M_{1}\times
    M_{2}$ son diferenciables,
  \item \label{ejer-item:incl}
    si $(p_{0},q_{0})\in M_{1}\times M_{2}$, entonces $i_{q_{0}}: M_{1}\to
    M_{1}\times M_{2}$ dada por $i(p) = (p,q_{0})$ es diferenciable.
  \end{nlist}
\end{ejer}
\begin{sol}
  \ref{ejer-item:proy}: Tomando $(x,y)\in M_{1}\times M_{2}$, y $(V_{1}\times V_{2},
  \phi_{1}\times\phi_{2})$ una carta en $(x,y)$, el siguiente diagrama ilustra
  la situación:
  \[
    \begin{tikzcd}
      V_{1}\times V_{2} \arrow{r}{\pi_{1}} \arrow{d}{\phi_{1}\times\phi_{2}} &
      V_{1} \arrow{d}{\phi_{1}}\\
      \R^{n_{1}+n_{2}} & \R^{n_{1}}
    \end{tikzcd}
  \]
  Ahora, es claro que $\phi_{1}\circ\pi_{1}\circ(\phi_{1}\times\phi_{2})^{-1}$
  actúa como la proyección $(\phi_{1}(x),\phi_{2}(y))\mapsto \phi_{1}(x) :
  \phi_{1}(V_{1})\times\phi_{2}(V_{2})\to \phi_{1}(V_{1})$.

  \ref{ejer-item:incl}: en este caso, tendríamos un diagrama parecido:
  \[
    \begin{tikzcd}
      V_{1} \arrow{r}{i_{q_{0}}} \arrow{d}{\phi_{1}} &
      V_{1}\times V_{2} \arrow{d}{\phi_{1}\times\phi_{2}}\\
      \R^{n_{1}} & \R^{n_{1}+n_{2}}
    \end{tikzcd}
  \]
  y la composición $(\phi_{1}\times\phi_{2})\circ i_{q_{0}}\circ \phi_{1}^{-1}$
  llevaría puntos $\phi_{1}(x)$ en $(\phi_{1}(x), \phi_{2}(q_{0}))$.
\end{sol}