%%%%%%%%%%%%%%%%%%%%%%%%%%%%%%%%%%%%%%%%%%%%%%%%%%%%%%%%%%%%%%%%
%
% Apuntes de la asignatura Análisis Matemático I.
% Doble Grado de Informática y Matemáticas.
% Universidad de Granada.
% Curso 2016/17.
% 
% 
% Colaboradores:
% Javier Sáez (@fjsaezm)
% Daniel Pozo (@danipozodg)
% Pedro Bonilla (@pedrobn23)
% Guillermo Galindo
% Antonio Coín (@antcc)
% Sofía Almeida (@SofiaAlmeida)
%
% Agradecimientos:
% Andrés Herrera (@andreshp) y Mario Román (@M42) por
% las plantillas base.
%
% Sitio original:
% https://github.com/libreim/apuntesDGIIM/
%
% Licencia:
% CC BY 4.0 (https://creativecommons.org/licenses/by/4.0/)
%
%%%%%%%%%%%%%%%%%%%%%%%%%%%%%%%%%%%%%%%%%%%%%%%%%%%%%%%%%%%%%%%


%------------------------------------------------------------------------------
%   ACKNOWLEDGMENTS
%------------------------------------------------------------------------------

%%%%%%%%%%%%%%%%%%%%%%%%%%%%%%%%%%%%%%%%%%%%%%%%%%%%%%%%%%%%%%%%%%%%%%%%
% Plantilla básica de Latex en Español.
%
% Autor: Andrés Herrera Poyatos (https://github.com/andreshp) 
%
% Es una plantilla básica para redactar documentos. Utiliza el paquete  fancyhdr para darle un
% estilo moderno pero serio.
%
% La plantilla se encuentra adaptada al español.
%
%%%%%%%%%%%%%%%%%%%%%%%%%%%%%%%%%%%%%%%%%%%%%%%%%%%%%%%%%%%%%%%%%%%%%%%%%

%%%
% Plantilla de Trabajo
% Modificación de una plantilla de Latex de Frits Wenneker para adaptarla 
% al castellano y a las necesidades de escribir informática y matemáticas.
%
% Editada por: Mario Román
%
% License:
% CC BY-NC-SA 3.0 (http://creativecommons.org/licenses/by-nc-sa/3.0/)
%%%

%%%%%%%%%%%%%%%%%%%%%%%%%%%%%%%%%%%%%%%%
% Short Sectioned Assignment
% LaTeX Template
% Version 1.0 (5/5/12)
%
% This template has been downloaded from:
% http://www.LaTeXTemplates.com
%
% Original author:
% Frits Wenneker (http://www.howtotex.com)
%
% License:
% CC BY-NC-SA 3.0 (http://creativecommons.org/licenses/by-nc-sa/3.0/)
%
%%%%%%%%%%%%%%%%%%%%%%%%%%%%%%%%%%%%%%%%%


% Tipo de documento y opciones.
\documentclass[11pt, a4paper, titlepage]{article}


%---------------------------------------------------------------------------
%   PAQUETES
%---------------------------------------------------------------------------

% Idioma y codificación para Español.
\usepackage[utf8]{inputenc}
\usepackage[spanish, es-tabla, es-lcroman, es-noquoting]{babel}
\selectlanguage{spanish} 
%\usepackage[T1]{fontenc}

% Fuente utilizada.
\usepackage{courier}    % Fuente Courier.
\usepackage{microtype}  % Mejora la letra final de cara al lector.

% Diseño de página.
\usepackage{fancyhdr}   % Utilizado para hacer títulos propios.
\usepackage{lastpage}   % Referencia a la última página.
\usepackage{extramarks} % Marcas extras. Utilizado en pie de página y cabecera.
\usepackage[parfill]{parskip}    % Crea una nueva línea entre párrafos.
\usepackage{geometry}            % Geometría de las páginas.

% Símbolos y matemáticas.
\usepackage{amssymb, amsmath, amsthm, amsfonts, amscd}
\usepackage{upgreek}

% Otros.
\usepackage{enumitem}   % Listas mejoradas.
\usepackage[hidelinks]{hyperref}


%---------------------------------------------------------------------------
%   OPCIONES PERSONALIZADAS
%---------------------------------------------------------------------------

% Redefinir letra griega épsilon.
\let\epsilon\upvarepsilon

% Formato de texto.
\linespread{1.1}            % Espaciado entre líneas.
\setlength\parindent{0pt}   % No indentar el texto por defecto.
\setlist{leftmargin=.5in}   % Indentación para las listas.

% Estilo de página.
\pagestyle{fancy}
\fancyhf{}
\geometry{left=3cm,right=3cm,top=3cm,bottom=3cm,headheight=1cm,headsep=0.5cm}   % Márgenes y cabecera.

% Redefinir entorno de demostración (reducir espacio superior)
\makeatletter
\renewenvironment{proof}[1][\proofname] {\vspace{-15pt}\par\pushQED{\qed}\normalfont\topsep6\p@\@plus6\p@\relax\trivlist\item[\hskip\labelsep\it#1\@addpunct{.}]\ignorespaces}{\popQED\endtrivlist\@endpefalse}
\makeatother

% Aumentar el tamaño del interlineado
\linespread{1.3}
%---------------------------------------------------------------------------
%   COMANDOS PERSONALIZADOS
%---------------------------------------------------------------------------

% Valor absoluto: \abs{}
\providecommand{\abs}[1]{\lvert#1\rvert}    

% Fracción grande: \ddfrac{}{}
\newcommand\ddfrac[2]{\frac{\displaystyle #1}{\displaystyle #2}}

% Texto en negrita en modo matemática: \bm{}
\newcommand{\bm}[1]{\boldsymbol{#1}}

% Línea horizontal.
\newcommand{\horrule}[1]{\rule{\linewidth}{#1}}


%---------------------------------------------------------------------------
%   CABECERA Y PIE DE PÁGINA
%---------------------------------------------------------------------------

% Cabecera del documento.
\renewcommand\headrule{
	\begin{minipage}{1\textwidth}
		\hrule width \hsize 
	\end{minipage}
}

% Texto de la cabecera.
\lhead{\subject}  % Izquierda.
\chead{}            % Centro.
\rhead{\docauthor}    % Derecha.

% Pie de página del documento.
\renewcommand\footrule{                                 
	\begin{minipage}{1\textwidth}
		\hrule width \hsize   
	\end{minipage}\par
}

% Texto del pie de página.
\lfoot{}                                                 % Izquierda
\cfoot{}                                                 % Centro.
\rfoot{Página\ \thepage\ de\ \protect\pageref{LastPage}} % Derecha.


%---------------------------------------------------------------------------
%   ENTORNOS PARA MATEMÁTICAS
%---------------------------------------------------------------------------

% Nuevo estilo para definiciones.
\newtheoremstyle{definition-style} % Nombre del estilo.
{10pt}               % Espacio por encima.
{10pt}               % Espacio por debajo.
{}                   % Fuente del cuerpo.
{}                   % Identación.
{\bf}                % Fuente para la cabecera.
{.}                  % Puntuación tras la cabecera.
{.5em}               % Espacio tras la cabecera.
{\thmname{#1}\thmnumber{ #2}\thmnote{ (#3)}}     % Especificación de la cabecera (actual: nombre en negrita).

% Nuevo estilo para notas.
\newtheoremstyle{remark-style} 
{10pt}                
{10pt}                
{}                   
{}                   
{\itshape}          
{.}                  
{.5em}               
{}                  

% Nuevo estilo para teoremas y proposiciones.
\newtheoremstyle{theorem-style}
{10pt}                
{10pt}                
{\itshape}           
{}                  
{\bf}             
{.}                
{.5em}               
{\thmname{#1}\thmnumber{ #2}\thmnote{ (#3)}}                   

% Nuevo estilo para ejemplos.
\newtheoremstyle{example-style}
{10pt}                
{10pt}                
{}                  
{}                   
{\scshape}              
{:}                 
{.5em}               
{}                   

% Teoremas, proposiciones y corolarios.
\theoremstyle{theorem-style}
\newtheorem*{nth}{Teorema}
\newtheorem*{nprop}{Proposición}
\newtheorem{ncor}{Corolario}

% Definiciones.
\theoremstyle{definition-style}
\newtheorem*{ndef}{Definición}

% Notas.
\theoremstyle{remark-style}
\newtheorem*{nota}{Nota}

% Ejemplos.
\theoremstyle{example-style}
\newtheorem*{ejemplo}{Ejemplo}

% Listas ordenadas con números romanos (i), (ii), etc.
\newenvironment{nlist}
{\begin{enumerate}
\renewcommand\labelenumi{(\emph{\roman{enumi})}}}
{\end{enumerate}}

% División por casos con llave a la derecha.
\newenvironment{rcases}
  {\left.\begin{aligned}}
  {\end{aligned}\right\rbrace}

%---------------------------------------------------------------------------
%   PÁGINA DE TÍTULO
%---------------------------------------------------------------------------

% Título del documento.
\newcommand{\subject}{Análisis Matemático I}

% Autor del documento.
\newcommand{\docauthor}{Doble Grado de Informática y Matemáticas}

% Título
\title{
  \normalfont \normalsize 
  \textsc{Universidad de Granada} \\ [25pt]    % Texto por encima.
  \horrule{0.5pt} \\[0.4cm] % Línea horizontal fina.
  \huge \subject\\ % Título.
  \horrule{2pt} \\[0.5cm] % Línea horizontal gruesa.
}

% Autor.
\author{\Large{\docauthor}}

% Fecha.
\date{\vspace{-1.5em} \normalsize Curso 2016/17}

\begin{document}

\maketitle  % Título.
\tableofcontents    % Índice
\newpage

\section{Teorema de Bolzano-Weierstrass.}
Sea $\{x_n\}$ una sucesión de $\mathbb{R}^N$ acotada. Entonces existe una sucesión parcial suya $\{x_{\sigma(n)}\}$ convergente.

\subsection{Demostración.}
Notaremos $x_n = (x_n^1, \dots, x_n^N)$. Como $\{x_n^1\}$ es acotada en $\mathbb{R}$, existe $\sigma_1 : \mathbb{N} \rightarrow \mathbb{N}$ estrictamente creciente tal que $\{x_{\sigma_1(n)}^1\}$ es convergente.

Ahora, como $\{x_n^2\}$ es acotada, $\{x_{\sigma_1(n)}^2\}$ también es acotada, y existe $\sigma_2 : \mathbb{N} \rightarrow \mathbb{N}$ estrictamente creciente tal que $\{x_{(\sigma_2\circ\sigma_1)(n)}^1\}$ es convergente.

Procediendo de esta forma con cada componente de $x_n$, obtenemos $\sigma_1, \dots, \sigma_N$, y\\ $\{x_{\sigma_1(n)}^1\}, \{x_{(\sigma_2\circ\sigma_1)(n)}^2\}, \dots, \{x_{(\sigma_N\circ\dots\circ\sigma_2\circ\sigma_1)(n)}^N\}$ sucesiones convergentes en $\mathbb{R}$. Al ser $\sigma_i$ estrictamente creciente $\forall i=1,\dots,N$, $\{x_{(\sigma_N(n)\circ\dots\circ\sigma_{i+1}\sigma_i\circ\dots\circ\sigma_1)(n)}^i\}$ también es convergente (toda sucesión parcial de una sucesión convergente es convergente).

Así, tomando $\sigma = \sigma_1\circ\dots\circ\sigma_N$, $\{x_{\sigma(n)}\}$ es convergente.


\section{$\mathbb{R}^N$ es completo.}

Sea $\{x_n\} \subseteq \mathbb{R}^N$. Entonces:
\[
	\{x_n\} \text{ es de Cauchy} \iff \{x_n\}\rightarrow x \text{ es convergente}
\]

\subsection{Demostración.}
\boxed{\Leftarrow}\\
Dado $\varepsilon > 0$, existe $m\in \mathbb{N}$ tal que si $n \ge m$ entonces $d(x_n, x) < \displaystyle\frac{\varepsilon}{2}$, y si $p,q \ge m$ entonces $d(x_p, x_q) \le d(x_p, x) + d(x, x_q) < \varepsilon $

\boxed{\Rightarrow}\\
Como $\{x_n\}$ es de Cauchy, $\{x_n^i\}$ es de Cauchy $\forall i = 1,\dots,N$ (porque $|x_n^i-x_m^i| \le |x_n-x_m|$). $\implies$ $\{x_n^i\}\rightarrow x^i$ es convergente, por ser $\mathbb{R}$ completo. Luego $\{x_n\}$ es convergente.

\section{Teorema de Weierstrass generalizado.}
Sean $(X, d)$, $(Y, d)$ espacios métricos, $\emptyset \ne A \subseteq X$ compacto, y $f : A \rightarrow Y$ continua. Entonces, $f(A)$ es compacto.

\subsection{Demostración.}
Sea $\{x_n\}\subseteq A$ cualquiera. Entonces $\{f(x_n)\} \subseteq f(A)$. Como $A$ es compacto, existe $\sigma: \mathbb{N} \rightarrow \mathbb{N}$ estrictamente creciente tal que $\{x_{\sigma(n)}\} \rightarrow a\in A$. Luego, $\{f(x_{\sigma(n)})\}\rightarrow f(a)\in A$ y queda probado que $f(A)$ es compacto.

\section{Teorema del valor intermedio.}
Sea $\emptyset \ne A \subseteq \mathbb{R}^N$ arco conexo, y $f: A \longrightarrow \mathbb{R}^M$ continua. Entonces, $f(A)$ es arco-conexo en $\mathbb{R}^M$.


\subsection{Demostración.}
Sean $X,Y\in f(A)$. Entonces, $\exists x,y \in A : X=f(x), \ Y=f(y)$. Como $A$ es arco-conexo, $\exists\varphi : [a,b]\longrightarrow \mathbb{R}^N \text{ continua tal que}\ \varphi(a) = x,\; \varphi(b)=y,\; \varphi([a,b]) \subseteq A$.


Ahora, definimos $\psi := f\circ \varphi : [a,b] \longrightarrow \mathbb{R}^M$, que es continua por ser composición de funciones continuas. Entonces, se verifica que: $$\psi(a) = f(\varphi(a)) = f(x) = X;\quad \psi(b)= f(\varphi(b)) = f(y) = Y;\quad \psi([a,b]) = f(\varphi([a,b])) \subseteq f(A).$$
Por tanto, queda probado que $f(A)$ es arco-conexo en $\mathbb{R}^M$.

\section{Teorema de Heine-Cantor.}
Sea $\emptyset\ne A \subseteq \mathbb{R}^N$ compacto, y $f : A \longrightarrow \mathbb{R}^N$ continua. Entonces $f$ es uniformemente continua en $A$.



\subsection{Demostración.}
	$f$ es continua en A $\implies$ $f$ es continua en $a\;\; \forall a \in A$. Ahora, sea $\epsilon>0$ fijo.
	\[
	    \forall a \in A \quad \exists \delta = \delta_a > 0\;\; \forall x\in A\;\; d(x,a) < \delta_a \implies d(f(x),f(a))<\epsilon
	\]

	Tomamos un recubrimiento abierto de $A$, y como $A$ es compacto, encontramos un subrecubrimiento finito.
	
	\[
		A \subseteq \bigcup_{a\in A} B(a, \frac{\delta_a}{2}) \implies \exists a_1,\dots,a_n \in A: A \subseteq \bigcup_{i=1}^n B\left(a_i, \frac{\delta_{a_i}}{2}\right)
	\]
	
	Por esta última inclusión:
	\[
		\forall x\in A\quad \exists i \in \left \{ 1,\dots,n \right \} : x\in B\left(a_i,\frac{\delta_{a_i}}{2}\right)\cap A \implies f(x)\in B(f(a_i),\epsilon)
	\]
	
	Sean $\delta = \min\left\{\ddfrac{\delta_{a_i}}{2} : i \in \left \{ 1,\dots,n \right \}\right\} > 0$ y $y\in A : d(x,y) < \delta < \delta_{a_i}$ para un $x\in A$ fijo. Tomamos el $a_i$ proporcionado por la proposición anterior para $x$.
	\[
		d(y,a_i) \le d(y,x)+d(x,a_i) < \delta_{a_i} \implies y\in B(a_i,\delta_{a_i}) \implies f(y) \in B(f(a_i), \epsilon)
	\]
	
	Finalmente,
	\[
		d(f(x), f(y)) \le d(f(x),f(a_i)) + d(f(a_i), f(y)) < \epsilon
	\]
	
	Para cualquier $\epsilon$ para el que se desee que se verifique la condición de la continuidad uniforme, basta tomar $\ddfrac{\epsilon}{2}$ en la continuidad.

\subsection{Demostración alternativa.}
	La condición para la continuidad uniforme es la siguiente:
	\[
		\forall \varepsilon > 0 \  \exists \delta > 0 \ \forall x, y \in A : d(x,y) < \delta \implies d (f(x) , f(y) ) < \varepsilon
	\]
	
	Vamos a proceder por reducción al absurdo, para lo cual negamos esta condición:
	\[
		\exists \varepsilon_0 > 0 \ \forall \delta > 0 \ \exists x, y \in A : d (x,y) < \delta \wedge d (f(x) , f(y) ) \ge \varepsilon_0
	\]
	
	Tomamos este $\epsilon_0$, lo que nos da, para cada $\delta>0$, un par de puntos $x$ e $y$ que cumplen la propiedad expresada arriba. Tomamos $\delta = \frac{1}{n} \ \forall n\in \mathbb{N}$. Esto nos da dos sucesiones $\{x_n\}$ e $\{y_n\}$ tales que
	\[
		d(x_n,y_n) < \frac{1}{n} \wedge d(f(x_n),f(y_n)) \ge \epsilon_0
	\]
	
	Por ser A compacto, el teorema de Bolzano-Weierstrass nos da dos sucesiones parciales $\{x_{n_k}\}$ a $x_0$ e $\{y_{n_k}\}$ a $y_0$. Por tanto:
	\[
		d(x_{n_k},y_{n_k}) < \frac{1}{n_k} \wedge d(f(x_{n_k}),f(y_{n_k})) \ge \epsilon_0
	\]
	
	Sin embargo, $\{x_{n_k}\}$ e $\{y_{n_k}\}$ convergen al mismo punto (por converger su distancia a cero), y como $f$ es continua, esta proposición no puede ser verdadera. Hemos llegado por tanto a una contradicción, luego $f$ debe ser uniformemente continua.

\end{document}
