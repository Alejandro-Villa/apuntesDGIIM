\section{Ecuaciones lineales: teoremas de existencia y unicidad.}


\begin{ndef}[Ecuación diferencial lineal]
Sea $(\alpha,\beta)\subseteq \R$ un intervalo abierto y sean $A:(\alpha,\beta)\to \M_d(\R)$ y $b:(\alpha,\beta)\to \R^d$ funciones continuas. Entonces una ecuación diferencial lineal es de la forma:

\begin{equation}
x'=A(t)x+b(t) \tag{C} \label{lin:completa}
,\end{equation}

y se dice \emph{completa}, o bien

\begin{equation}
x'=A(t)x \tag{H} \label{lin:homogenea}
,\end{equation}

y en este caso se dice \emph{homogénea}.
\end{ndef}

\begin{nth}[Teorema de existencia y unicidad de la solución]
Dados $t_0\in(\alpha,\beta)$ y $x_0 \in \R^d$ y consideramos el \emph{problema de valores intermedios} (PVI):

\begin{equation}
\left\{\begin{array}{rl}
x' &=A(t)x+b(t), \\
x(t_0)&=x_0.
\end{array}\right.
\tag{PVI}\label{lin:PVI}
\end{equation}

Entonces, existe una única solución $\varphi : (\alpha,\beta)\to\R^d$, con $\varphi \in \C ^ 1(\R)$, que verifica

\[
\varphi'(t) = A(t)\varphi(t) + b(t) \qquad \forall t \in (\alpha,\beta)
\]

y que además cumple $\varphi(t_0)=x_0$, es decir, una única solución de \ref{lin:PVI}.
\end{nth}

Aplicando el teorema al PVI

\[
\begin{cases}
  x' = Ax, \\
  x(t_0) = e_j.
\end{cases}
\tag{PVI$_j$} \label{lin:PVIj}
\]

para cada $j=1,\dots,d$, obtenemos soluciones $\varphi_j$. Sea $x_0\in \R^d$, y consideramos $\varphi = \sum_{j=1}^d \langle x_0,e_j \rangle\varphi_j$.
Es inmediato comprobar que $\varphi$ es solución de \eqref{lin:PVI} con $b=0$. Por tanto, hemos probado:

\begin{ncor}
  Las soluciones de \eqref{lin:homogenea} son un espacio vectorial de dimensión $d$.
\end{ncor}

Ahora, consideremos $\varphi$ la solución de \eqref{lin:PVI}, para $x_0\in\R^d$ arbitrario, es decir, una solución cualquiera de \eqref{lin:completa}.
Sean $\varphi_c$ la solución de

\begin{equation}
\left\{\begin{array}{rl}
x' &=A(t)x+b(t), \\
x(t_0)&=x_0',
\end{array}\right.
\end{equation}

con $x_0'\in\R^d$ arbitrario, y $\varphi_h$ la de

\begin{equation}
\left\{\begin{array}{rl}
x' &=A(t)x, \\
x(t_0)&=x_0 - x_0',
\end{array}\right.
\end{equation}

y ahora es rutinario comprobar que $\varphi = \varphi_c + \varphi_h$. Además, para cualquier otra solución $\psi_{h}$ de \eqref{lin:homogenea}, $\varphi_c + \psi_h$
es solución de \eqref{lin:completa}.
En esta ocasión, hemos probado:

\begin{ncor}
  El conjunto $S_c$ de soluciones de \eqref{lin:completa}, dada cualquier solución $\varphi$ de la misma, es el espacio afín

  \[
    S_c = \varphi + S_h,
    \]

    con $S_h$ el conjunto de soluciones de \eqref{lin:homogenea}.
\end{ncor}

Otra consecuencia inmediata de la aplicación del teorema a \ref{lin:PVIj} es la siguiente:

\begin{ncor}[Existencia y unicidad de m.f.p.] \label{lin:cormfp}
  Existe una única función de clase 1 $\Phi: (\alpha, \beta) \to \M_d(\R)$ tal que $\Phi' = A\Phi$, y
  además $\Phi(t_0) = I_d$.
\end{ncor}

\subsection{Matriz fundamental principal en un punto}
Dada una EDO lineal y homogénea con coeficientes constantes:

\begin{equation}
\left\{\begin{array}{rl}
x' &=A(t)x, \\
x(t_0)&=x_0,
\end{array}\right.
\tag{PVIH} \label{lin:PVIH}
\end{equation}

La matriz cuya existencia garantiza \ref{lin:cormfp} (la \emph{matriz fundamental principal (m.f.p.) en $t_0$}) viene dada por:

\[ \Phi(t)=e^{A(t-t_0)} \qquad \forall t \in (\alpha,\beta) \]

Ahora nos preguntamos cómo se calcula la m.f.p. cuando la matriz $A$ es diagonalizable. En este caso, $\exists P, D \in \M_d(\C)$ tales que $det(P)\neq 0, \quad D=diag(\lambda_1,...\lambda_d)$ y $A=P\cdot D \cdot P^{-1}$. En cuyo caso, la matriz fundamental principal será:
$$e^{At}=P\cdot e^{Dt}\cdot P^{-1}=P \cdot  \begin{pmatrix}e^{\lambda_1 t} & \cdots & 0 \\ \vdots & \ddots & \vdots \\0 & \cdots & e^{\lambda_d t}\end{pmatrix} \cdot P^{-1}$$


Cuando $A$ no sea diagonalizable tendremos que utilizar la forma canónica de Jordan. En este caso, $\exists P,J\in \M_d(\C)$ tales que $det(P) \neq 0 $ y $J$ diagonal por bloques:
$$\begin{pmatrix}J_1 & \cdots & 0\\
 \vdots & \ddots & \vdots \\
 0& \cdots & J_r \end{pmatrix}$$
 donde $orden(J_k)=1 \quad \forall =1,...,r \implies J_k=(\lambda_k)$ con $\lambda_k$ valor propio de $A$ o si $orden(J_k)>1$,
 $$J_k=
 \begin{pmatrix}
 \lambda_k & 1  &\cdots & 0 \\
  0 & \lambda_k & \ddots & \vdots \\
  \vdots & \ddots & \ddots & 1 \\
  0 & \cdots & 0 & \lambda_k
\end{pmatrix}$$

%% FIXME : POR TERMINAR

\section{Estabilidad en el sentido de Lyapunov.}

En esta sección, con arreglo a definir los conceptos básicos de la teoría de estabilidad
de Lyapunov, consideramos un problema de valores iniciales más general:

\begin{equation}
\left\{\begin{array}{rl}
x' &=f(t,x), \\
x(t_0)&=x_0,
\end{array}\right.
\tag{P} \label{est:PVI}
\end{equation}

con $D \subseteq \R \times \R^d$, $f: D \to \R^d$ continua y tal que haya unicidad de soluciones en \ref{est:PVI}.
Notamos, para cada $x_0\in \R^d$ tal que $(t_0,x_0) \in D$, $\varphi_{x_0} : (\alpha, +\infty) \to \R^d$ a la solución maximal de \ref{est:PVI}.

\begin{ndef}
  $\varphi_{x_0}$ es \emph{estable} si, para cada $\varepsilon > 0$, existe $\delta > 0$ tal que

  \[
    (t_0, \bar{x_0}) \in D \text{ y } \|x_0 - \bar{x_0}\| < \delta \implies
  \begin{cases}
    \text{ $\varphi_{\bar{x_0}}$ está definida en $(t_0, +\infty)$}, \\
    \|\varphi_{x_0}(t) - \varphi_{\bar{x_0}}(t)\| < \varepsilon \text{ para cada $t \ge t_0$}.
  \end{cases}
  \]

  $\varphi_{x_0}[t_0, +\infty)+(-\varepsilon, \varepsilon)$ se llama \emph{entorno tubular} de $\varphi_{x_0}$.
\end{ndef}

\begin{ndef}
  Una solución se dice \emph{inestable} si no es estable.
\end{ndef}

%%% TODO: poner algunos ejemplos de PVI con soluciones estables e inestables.

Negando la definición de estabilidad, obtenemos:

\begin{nprop}[Caracterización de soluciones inestables]
  $\varphi_{x_0}$ es inestable si, y solo si, existen $\varepsilon_0 > 0, \xn \subseteq\R^d$ con
  $(t_0,x_n)\in D$, $x_n \to x_0$ y existe, para cada $n\in\N$, $t_n \ge t_0$ tal que

  \[
    \| \varphi_{x_n}(t_n) - \varphi_{x_0}(t_n) \| \ge \varepsilon_0
  .\]
\end{nprop}

\begin{ndef}[Atractor]
  $\varphi_{x_0}$ es un \emph{atractor} (local) si existe $\mu > 0$ tal que, si $(t_0, \bar{x_0})\in D$
  y $\| x_0 - \bar{x_0} \| < \mu$, entonces $\varphi_{\bar{x_0}}$ está definida en $(t_0, +\infty)$ y

  \[
     \lim_{t\to+\infty} \| \varphi_{\bar{x_0}}(t) - \varphi_{x_0}(t) \| = 0.
  \]
\end{ndef}

\begin{ndef}[Estabilidad asintótica]
 $\varphi_{x_0}$ es \emph{asintóticamente estable (a.e.)} si es estable y atractor.
\end{ndef}


%% CLASE LUNES 17

\begin{nprop}
Las siguientes afirmaciones son equivalentes:
\begin{nlist}
\item Todas las soluciones de \eqref{lin:completa} son atractores.
\item Existe una solución de \eqref{lin:completa} que es un atractor.
\item La solución trivial $y = 0$ de \eqref{lin:homogenea} es un atractor.
\item Todas las soluciones de \eqref{lin:homogenea} convergen hacia el vector 0 cuando $t \to +\infty$.
\item La matriz fundamental de \eqref{lin:homogenea} principal en $t_0$ converge hacia la matriz 0 cuando $t \to +\infty$.
\end{nlist}
\end{nprop}

\begin{ncor}
Los atractores de la ecuación \eqref{lin:homogenea} son asintóticamente estables.
\end{ncor}

\begin{ndef}
\begin{itemize}
\item Se dice que la ecuación \eqref{lin:completa} es estable si todas su soluciones son estables
\item Se dice que la ecuación \eqref{lin:completa} es asintóticamente estable si todas sus soluciones son asintóticamente estables.
\end{itemize}
\end{ndef}

No siempre podemos calcular la matriz fundamental, pero nosotros nos centraremos en el caso escalar donde sí podemos calcularla.

\subsection{Estabilidad de ecuaciones lineales escalares}
Sean $a,b:(\alpha,+\infty)\to \R$ continuas. Se considera la ecuación diferencial lineal escalar \eqref{completa} y sea $t_0\in(\alpha,+\infty)$.
La matriz fundamental principal en $t_0$ en $\phi(t)=exp(\int^t_{t_0} a(s)ds)$ y por tanto podemos caracterizar la estabilidad de \eqref{completa} controlando una primitiva del coeficiente $a(t)$.
\begin{nprop}
\begin{itemize}
\item La ecuación de \ eqref{completa} es estable sii la función $a(t)$ tiene una primitiva acotada superiormente en $[t_0,+\infty]$.
\item La ecuación \ eqref{completa} es a.e. sii la función $a(t)$ tiene una primitiva que converge hacia $-\infty$ [....]
\end{itemize}
\end{nprop}
Ejemplo 1:
- Inestable
- A.E.
- estable
-
- Inestable
- estable
- estable
-


Estable $-> \lambda <= 0$
A.E. $-> \lambda < 0$
Inestable $-> \lambda > 0$

\subsection{Estabilidad de ecuaciones lineales con coeficientes constantes}
Sea $A \in \M_d(\R)$ una matriz cuadrada.  Usaremos la siguiente notación:
NOTACIÓN:
\begin{ndef}{Espectro de $A$}
El espectro de $A$ es el conjunto de valores propios de $A$, tanto reales como complejos, $\sigma(A)={\lambda_1,...,\lambda_d}$, contados con su multiplicidad en nuestro caso.
\end{ndef}
\begin{ndef}{Multiplicidad del valor propio $\lambda_j$}
$m(\lambda_j) \forall \lambda_j \in \sigma{A}$
\end{ndef}
\begin{ndef}{Dimensión de cada subespacio propio $E_{\lambda_j}$}
$dim E_{\lambda_j} = dim ker(A-\lambda_{jI})= d-rango(A-\lambda I)$
\end{ndef}
\begin{ndef}
Los valores propios de A cuya parte real es 0 son:
$\sigma_0(A)={\lambda \in \sigma(A) \ Re(\lambda)=0}$
\end{ndef}

Consideramos la EDO lineal homogénea y autónoma $x'=Ax x\in\R^d$. El principal indicador para determinar la estabilidad de * es el máximo de las partes reales de los valores propios de A:
$\mu(A)=\max{Re(\lambda : \lambda\in\sigma(A))}$

\begin{nth}
\begin{enumerate}
\item Si $\mu(A)<0$ entonces la EDO lineal * es A.E.
\item Si $\mu(A)=0$ y $m(\lambda)=dimE_\lambda \forall \lambda \in \sigma_0(A)$, entonces la EDO lineal * es estable (pero no A.E.)
\item Si $\mu(A)=0$ y $\exists \lambda \in \sigma_0(A)$ tal que $m(\lambda) != dim E_lambda$, entonces la EDO lineal * es inestable.
\item Si $\mu(A)>0$ entonces la EDO lineal * es inestable.
\end{enumerate}
\end{nth}

A continuación, haremos una serie de ejemplos y proseguiremos con un poquito de teoría para poder producir la demostración de este enunciado.\\
\begin{ejemplo}
\begin{equation}
\left\{\begin{array}{lcl}
x_1'=2x_1+3x_2\\
x_2'=5x_1-x_2
\end{array}\right.
\end{equation}

\begin{equation}
\begin{pmatrix}
x_1 \\
x_2
\end{pmatrix}'=
\begin{pmatrix}
-2 & 3\\
5 & -1
\end{pmatrix}
\begin{pmatrix}
x_1 \\
x_2
\end{pmatrix}
\end{equation}


EC. $\lambda^2-traza(A)\lambda + det(A=0)$
$\lambda+3\lambda-13=0$
$\sigma(A)={\frac{-3+\sqrt{61}}{2},\frac{-3-\sqrt{61}}{2}}={2.4...,5.4...}$
$\mu(A)>2.4 >0$
La ecuación es inestable.
\end{ejemplo}

%% MARTES 18

Recordemos la forma canónica de Jordan.

Sea $A \in \M_d(\R) \exists P, J \in \M_d(\C) tales que:$
\begin{itemize}
\item $P$ es regular (invertible)
\item $J$ es diagonal por bloques:
\begin{equation}
J=diag(J_1,...,J_k) J_j matriz diagonal \forall j=1,...,k
\end{equation}
\item $A=P\cdot J\cdot P^{-1}$
Para calcular $J$ se usan:
\item $\sigma(A)={\lambda_1,...\lambda_d}$
\item $m(\lambda_j)$
\item $dim E_{\lambda_j}$
\item $\nu(\lambda_j)=min{k\in\mathbb{Z}^+: BORRA}$
Los bloques de $J$ pueden ser:
\item de orden 1: $J_k=(\lambda_k) \lambda_k \in \sigma(A)$
\item de orden mayor que 1: $J_k=diag(\lambda_k,...\lambda_k) (1...1 encima de la diag) con \lambda_k \in \sigma(A)$
\item el orden $(J_k) \leq \nu(\lambda_k)$
\item la suma de los órdenes de todos los bloques asociados aun mismo valor propio es igual a la multiplicidad de ese valor propio
\end{itemize}

INSERTE EJEMPLO DE CAJA DE JORDAN

Con estos datos, podemos hacer el siguiente cálculo de la exponencial:

$$A=P\cdot J \cdot P^{-1} \implies A^n=P \cdot J^n \cdot P^{-1} \implies e^A=P \cdot e^J \cdot P^{-1}$$

Y de manera análoga:
\begin{itemize}
\item $e^{At}=P \cdot e^{Jt} \cdot P^{-1}$
\item $e^{Jt}=diag(e^{J_1t},...,e^{J_rt} )$
\end{itemize}

%% Inserte ejemplo

\begin{lema}
$$|||e^{Jt}|||_1=\max{|||e^{J_1t}|||_1,|||e^J_rt|||_1}$$
\end{lema}

\begin{lema}
$\exists T \geq 0 $ tal que :
$$|||e^{Jt}|||_1=\left(1+...+\frac{t^n}{n!}\right)\cdot e^{\mu t} \qquad \forall t \geq T$$
donde:
$$\mu=\max{Re(\lambda_j):\lambda_j\in \sigma(A)}$$
$$n=\max{\nu(\lambda_j)-1 : \lambda_j\in \sigma(A) y Re(\lambda_j)=\mu}$$
\end{lema}

Usando el lema anterior:
\begin{equation}
|||e^{Jt}|||_1=\max
\left\{ {\left(
1+...+\frac{t^{\nu(\lambda_j)-1}}{(\nu (\lambda_j)-1)!}
\right) }
\cdot e^{Re(\lambda_j)t} : \lambda_j \in \sigma(A) \right\}
\tag{cosarara}\label{cosarara}
\end{equation}
$\exists T \geq 0$ tal que:
\begin{equation}
(1+...+\frac{t^{\nu(\lambda_j)-1}}{\nu(\lambda_j - 1})\cdot e^{Re(\lambda_j)t} \leq (1+...1+\frac{t^n}{n!}e^{\mu t})
\end{equation}

Finalmente, podemos demostrar el teorema anterior:

\begin{proof}
$\exists P, J \in \M_d(\C)$ tales que $P$ es inestable, $J$ es diagonal por bloques y $A=P\cdot J \cdot P^{-1}$. Por tanto $$e^{At}=P\cdot e^{Jt} \cdot P^{-1}$$.
Defino en $\M_d(\C)$ la norma matricial:
$$|||B|||=|||P\cdot J \cdot P^{-1}|||_1$$.
Así, tenemos que:
\begin{enumerate}
\item Si $\mu < 0 \implies \lim_{t\to\infty}|||e^{At}|||=0\implies x'=Ax$ es A.E.
\item Si $\mu = 0 y dimE_j=m(\lambda)' \forall \lambda \in \sigma_0(A)\implies n=0 \implies |||e^{At}|||=1\implies|||e^{At}||| es estable pero no A.E.$
\item Si $\mu = 0 y \exists \lambda \in \sigma_0(A) \nu(A)>1 \implies n\geq 1\implies |||e^{At}|||\geq 1+t$ no es acotada en $[0,+\infty)$
\item Si $\mu > 0 \qquad |||e^{At}|||\geq e^{\mu t}\to \infty$ no es acotada en $[0,+\infty)$
\end{enumerate}
\end{proof}


%% MIÉRCOLES 19
\section{Órbitas}
%% Lunes 24
\subsection{Diagramas de fase planos de EDO lineales}

Consideramos la EDO lineal homogénea autónoma en $\R^2$:

\begin{equation}
\left\{\begin{array}{rl}
x_1'=a_{11}x_1+a_{12}x_2 \\
x_2'=a_{21}x_1+a_{22}x_2
\end{array}\right.\Leftrightarrow
\begin{pmatrix}
x_1 \\
x_2 \\
\end{pmatrix}
'=
\begin{pmatrix}
a_{11} & a_{12} \\
a_{21} & a_{22} \\
\end{pmatrix}
\begin{pmatrix}
x_1 \\
x_2 \\
\end{pmatrix}
\end{equation}

%%%%%%%% Definiciones para los diagramas de fase. http://www.texample.net/tikz/examples/poincare/

\newcommand\newtemplate[4][0.4]%
 {\newsavebox#2%
   \savebox#2%
   {\begin{tabular}{@{}c@{}}
      \\[2#1ex]
      \begin{tikzpicture}[scale=#1]
      #4
      \end{tikzpicture}\\[4#1ex]
      %% \templatecaption{#3}\\[-1ex]
     \end{tabular}%
   }%
 }
\newcommand\template[1]{\usebox{#1}}             % use the Code stored in box #1
\newcommand\templatecaption[1]{{\sffamily\scriptsize#1}}       % typeset caption
\newcommand\Tr{\mathop{\mathrm{tr}}}

\newtemplate\sink{sink}%
 {\foreach \sx in {+,-}                   % for right/left half do:
   {\draw[flow] (\sx4,0) -- (0,0);        %   draw half of horizontal axis
    \draw[flow] (0,\sx4) -- (0,0);        %   draw half of vertical axis
    \foreach \sy in {+,-}                 %   for upper/lower quadrant do:
      \foreach \a/\b in {2/1,3/0.44}      %     draw two half-parabolas
        \draw[flow,domain=\sx\a:0] plot (\x, {\sy\b*\x*\x});
   }
 }

\newtemplate\source{source}%
 {\foreach \sx in {+,-}                   % for right/left half do:
   {\draw[flow] (0,0) -- (\sx4,0);        %   draw half of horizontal axis
    \draw[flow] (0,0) -- (0,\sx4);        %   draw half of vertical axis
    \foreach \sy in {+,-}                 %   for upper/lower quadrant do:
      \foreach \a/\b in {2/1,3/0.44}      %     draw two half-parabolas
        \draw[flow,domain=0:\sx\a] plot (\x, {\sy\b*\x*\x});
   }
 }

\newtemplate\stablefp{line of stable fixed points}%
 {\draw (-4,0) -- (4,0);                  % draw horizontal axis
  \foreach \sy in {+,-}                   % for upper/lower half do:
   {\draw[flow] (0,\sy4) -- (0,0);        %   draw half of vertical axis
    \foreach \x in {-3,-2,-1,1,2,3}       %   draw six vertical half-lines
      \draw[flow] (\x,\sy3) -- (\x,0);
   }
 }

\newtemplate\unstablefp{line of unstable fixed points}%
 {\draw (-4,0) -- (4,0);                  % draw horizontal axis
  \foreach \sy in {+,-}                   % for upper/lower half do:
   {\draw[flow] (0,0) -- (0,\sy4);        %   draw half of vertical axis
    \foreach \x in {-3,-2,-1,1,2,3}       %   draw six vertical half-lines
      \draw[flow] (\x,0) -- (\x,\sy3);
   }
 }

\newtemplate\spiralsink{spiral sink}%
 {\draw (-4,0) -- (4,0);                  % draw horizontal axis
  \draw (0,-4) -- (0,4);                  % draw vertical axis
  \draw [samples=100,smooth,domain=27:7]  % draw spiral
       plot ({\x r}:{0.005*\x*\x});       % Using "flow" here gives "Dimension
  \def\x{26}                              %        too large", so we draw a tiny
  \draw[->] ({\x r}:{0.005*\x*\x}) -- +(0.01,-0.01);%     tangent for the arrow.
 }

\newtemplate\spiralsource{spiral source}%
 {\draw (-4,0) -- (4,0);                  % draw horizontal axis
  \draw (0,-4) -- (0,4);                  % draw vertical axis
  \draw [samples=100,smooth,domain=10:28] % draw spiral
       plot ({-\x r}:{0.005*\x*\x});      % Using "flow" here gives "Dimension
  \def\x{27.5}                            %        too large", so we draw a tiny
  \draw[<-] ({-\x r}:{0.005*\x*\x}) -- +(0.01,-0.008);%   tangent for the arrow.
 }

\newtemplate[0.15]\centre{center}% British spelling since \center is in use
 {\draw (-4,0) -- (4,0);                  % draw horizontal axis
  \draw (0,-4) -- (0,4);                  % draw vertical axis
  \foreach \r in {1,2,3}                  % draw three circles
    \draw[flow=0.63] (\r,0) arc (0:-360:\r cm);
 }

\newtemplate\saddle{saddle}%
 {\foreach \sx in {+,-}                   % for right/left half do:
   {\draw[flow] (\sx4,0) -- (0,0);        %   draw half of horizontal axis
    \draw[flow] (0,0) -- (0,\sx4);        %   draw half of vertical axis
    \foreach \sy in {+,-}                 %   for upper/lower quadrant do:
      \foreach \a/\b/\c/\d in {2.8/0.3/0.7/0.6, 3.9/0.4/1.3/1.1}
        \draw[flow] (\sx\a,\sy\b)         %     draw two bent lines
          .. controls (\sx\c,\sy\d) and (\sx\d,\sy\c)
          .. (\sx\b,\sy\a);
   }
 }

\newtemplate\degensink{degenerate sink}%
 {\draw (0,-4) -- (0,4);                  % draw vertical axis
  \foreach \s in {+,-}                    % for upper/lower half do:
   {\draw[flow] (\s4,0) -- (0,0);         %   draw half of horizontal axis
    \foreach \a/\b/\c/\d in {3.5/4/1.5/1, 2.5/2/1/0.8}
      \draw[flow] (\s-3.5,\s\a)           %   draw two bent lines
        .. controls (\s\b,\s\c) and (\s\b,\s\d)
        .. (0,0);
   }
 }

\newtemplate\degensource{degenerate source}%
 {\draw (0,-4) -- (0,4);                  % draw vertical axis
  \foreach \s in {+,-}                    % for upper/lower half do:
   {\draw[flow] (0,0) -- (\s4,0);         %   draw half of horizontal axis
    \foreach \a/\b/\c/\d in {3.5/4/1.5/1, 2.5/2/1/0.8}
      \draw[flow] (0,0)                   %   draw two bent lines
        .. controls (\s\b,\s\d) and (\s\b,\s\c)
        .. (\s-3.5,\s\a);
   }
 }


%%%%%%%%


\subsubsection{Valores propios reales}

Consideramos primero el caso en que $\sigma(A)={\lambda_1,\lambda_2}\subset \R$, y a su vez analizamos el resto de casos según los signos
de los valores propios.

\paragraph{Reales, positivos y distintos:} $0<\lambda_1<\lambda_2$
$\Leftrightarrow \left\{ \begin{array}{ll}\Delta > 0, \\ \tr(A)>0, \\ \det(A) > 0. \end{array}\right.$

%%% FIXME: poner el texto y la figura en dos columnas con multicol

  \begin{figure}[h]
    \centering
    \begin{tikzpicture}[line cap=round,line join=round, scale=0.5]
   \foreach \sx in {+,-}                   % for right/left half do:
   {\draw[flow] (0,0) -- (\sx4,0);        %   draw half of horizontal axis
    \draw[flow] (0,0) -- (0,\sx4);        %   draw half of vertical axis
    \foreach \sy in {+,-}                 %   for upper/lower quadrant do:
      \foreach \a/\b in {2/1,3/0.44}      %     draw two half-parabolas
        \draw[flow,domain=0:\sx\a] plot (\x, {\sy\b*\x*\x});
        }
    \end{tikzpicture}
    \caption{Una fuente.}

  \end{figure}

$x(t)=c_1 e^{\lambda_1t}v_1+c_2 e^{\lambda_2t}v_2 \qquad v_j \in E_{\lambda_j}$

\begin{nlist}
  
\item Sonreales, negativos y distintos: $\lambda_2 < \lambda_1 \Leftrightarrow \left\{ \begin{array}{ll}\Delta > 0 \\ \tr(A) < 0 \\ \det(A)>0\end{array}\right.$
Inserte dibujito de sumidero (flechas hacia dentro)
\item Son reales, no nulos y con distinto signo : $\lambda_1 < 0 < \lambda_2 \Leftrightarrow \det(A)<0$
Inserte dibujito de punto hiperbólico o punto silla
\item Son reales, positivos e iguales : $\lambda_1=\lambda_2>0 \Leftrightarrow \left\{ \begin{array}{ll} \Delta = 0 \\ \tr(A)>0\end{array}\right.$.
  
\begin{enumerate}
\item Si $A$ es diagonalizable : En este caso tendremos todo el espacio y volveremos a tener una fuente.
\item Si $A$ no es diagonalizable : Nos encontramos con una fuente degenerada.
\end{enumerate}

\item Son reales, negativos e iguales : $\lambda_1=\lambda_2<0 \Leftrightarrow \left\{ \begin{array}{ll} \Delta = 0 \\ \tr(A)<0\end{array}\right.$.
  
\begin{enumerate}
\item Si $A$ es diagonalizable : En este caso tendremos todo el espacio y volveremos a tener un sumidero.
\item Si $A$ no es diagonalizable : Nos encontramos con un sumidero degenerado.
\end{enumerate}

\item Un valor propio es 0 y el otro positivo: $\lambda>0$ \\
Inserte dibujito rectas paralelas y perpendiculares a una única recta generada por el 0(Hacia dentro). Rectas  de puntos de equilibrio estables
\item Un valor propio es 0 y el otro negativo: $\lambda<0 $ \\
Inserte dibujito rectas paralelas y perpendiculares a una única recta generada por el 0(Hacia fuera). Rectas  de puntos de equilibrio inestables
\item Dos valores propios iguales a 0: $\lambda_1=\lambda_2=0 \Leftrightarrow \left\{ \begin{array}{ll}\Delta=0 \\ \tr(A)=0 \\ \det(A)=0\end{array}\right.$
  
\begin{enumerate}
\item Si $A$ es diagonalizable $A=0$
\item Si $A$ no es diagonalizable.
\end{enumerate}


\item Caso 2: Cuando los valores propios de la matriz A no son reales : $\sigma(A)={\lambda_1,\lambda_2}\subset \mathbb{C} - \R, \; \lambda_1=a+bi$ y $ \lambda_2=a-bi$.
  
\begin{enumerate}
\item Tiene parte real positiva $\left\{\begin{array}{ll} \Delta < 0 \\ tr(A) > 0 \end{array}\right.$ Fuente en espiral
\item Tiene parte real negativa $\left\{\begin{array}{ll} \Delta < 0 \\ tr(A) < 0 \end{array}\right.$ Sumidero en espiral
\item Tiene parte real igual a 0. Diagrama de Poincaré
\end{enumerate}

\end{nlist}

%% MARTES 25

\begin{ejemplo}
\end{ejemplo}

PREGUNTA EXAMEN PARCIAL: Representar el diagrama de fases

\begin{nota}
Recuerda que si $\lambda \in \sigma(A)$ y $v \in E_\lambda$ tal que $A\cdot v =\lambda\cdot v$ entonces $\varphi(t)=k\cdot e^{\lambda t}\cdot v \quad \forall t \in \R$ es solución de $x'=Ax$ y si $\varphi(t)\in E_\lambda \quad t \in \R \implies E_\lambda$ es invariante. De manera análoga $E_{\lambda_1}\oplus E_{\lambda_2} \quad k_1 e^{\lambda_1 t}v_1+k_2 e^{\lambda_2 t}v_2$ es también invariante.\\
El subespacio propio generalizado sería $\lambda\in\sigma(A) \quad \hat{E}_\lambda = \cup_{k\in\N}Ker(A-\lambda I)^k=Ker(A-\lambda I)^{\nu(\lambda) }$ con $\dim \hat{E}_\lambda=m(\lambda)$.\\
Se puede hacer la siguiente descomposición de $\R^d$.\\
Se definen:
$$E=\bigoplus_{ \begin{array}{c}\lambda\in\sigma(A)\\ Re(\lambda)<0\end{array}}\hat{E}_\lambda$$
$$\mho=\bigoplus_{\begin{array}{c}\lambda\in\sigma(A) \\ Re(\lambda)>0\end{array}}\hat{E}_\lambda$$
$$\mathcal{Z}=\bigoplus_{\begin{array}{c}\lambda\in\sigma_0(A)\end{array}}\hat{E}_\lambda$$
\end{nota}


entonces para $x'=Ax$:

$$\R^d=E\oplus\mathcal{Z}\oplus\mho$$

\begin{ejer}
Estudia la estabilidad del sistema lineal:
\[
\left\{\begin{array}{ll}
x'=-2tx \\
y'=e^tx-2ty
\end{array}\right.
\]
\end{ejer}
\begin{ejer}
Estudia la estabilidad del sistema lineal:
\[
\left\{\begin{array}{ll}
x'=-2x+5y\\
y'=-3x+4y
\end{array}\right.
\]
Comenzamos calculando la traza y el determinante de la matriz $A=\begin{pmatrix}-2 & 5 \\ -3 & 4 \end{pmatrix}$. Tenemos entonces que:
$$tr(A)=2=\lambda_1+\lambda_2=2 Re \lambda_1 >0$$
$$det(A)=7=\lambda_1 \cdot \lambda_2$$
$Re\lambda_1, Re\lambda_2>0$ luego si son reales será inestable y como $2 Re \lambda_1 >0$ entonces si son complejos también será inestable.
\end{ejer}

\begin{ejer}
\[
\left\{\begin{array}{ll}
x'=-2x+y-z\\
y'=2x-3y+z\\
z'=4x-4y+2z
\end{array}\right.
\]

Tras unos cálculos, tenemos que: $\sigma(A)={-2,-1,0}$ luego es estable pero no asintóticamente estable.
\end{ejer}

\begin{nota}
 El polinomio característico de $A\in\M(\mathbb{C})$ es $-\lambda^3+tr(A)\lambda^2-tr(adj(A))\lambda+det(A)$.
\end{nota}

\begin{ejer}
Estudia la estabilidad (FALTA)
\end{ejer}

\begin{ejer}
Representa el diagrama de fases de la EDO lineal:
\[
\left\{\begin{array}{ll}
x'=-x+3y\\
y'=2x-4y
\end{array}\right.
\]
Sus valores propios son:
$$\lambda_1=\frac{1}{2}(-5-\sqrt{33})\approx -5.37$$
$$\lambda_2=\frac{1}{2}(-5+\sqrt{33})\approx 0.37$$
Y sus vectores propios serían:
$$v_1=\begin{pmatrix}-0.68 \\ 1 \end{pmatrix} $$
$$v_2=\begin{pmatrix}2.19 \\ 1 \end{pmatrix} $$

%%\includegraphics{streamplotwolframalpha.gif}
\end{ejer}
