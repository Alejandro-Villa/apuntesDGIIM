\section{Fundamentación probabilística de vectores aleatorios}

En esta sección definimos algunos conceptos bien conocidos de la teoría de la probabilidad,
a modo de breve repaso y para establecer la notación que utilizaremos en todo el libro.

% TODO: insertar referencia a apuntes de Análisis Matemático II y Probabilidad.

\begin{ndef}[Vector y variable aleatorios]
    Sea $(\Omega, \mathscr{A}, P)$ un espacio de probabilidad. Un \emph{vector aleatorio}, $\boldsymbol X = (X_1, \dots, X_p)$ es una función medible
    \begin{align*}
      \boldsymbol X:(\Omega, \mathscr{A}) &\rightarrow (\mathbb{R}^p, \mathscr{B}^p)\,.
    \end{align*}
     Cuando $p=1$, lo notamos $X$ y decimos que es una \emph{variable aleatoria}.
    La condición de medibilidad es: \(\boldsymbol X^{-1}(B) \in \mathscr{A}\) para cada \( B \in \mathscr{B}^p\).
\end{ndef}



\begin{ndef}[Probabilidad inducida]
    La \emph{probabilidad inducida por} o \emph{distribución de} un vector aleatorio $\boldsymbol X = (X_1, \dots, X_p)$ se define como la función
    \[
    P_{\boldsymbol X}[B] := P[\boldsymbol X^{-1}(B)]\,.\] para todo \(B \in \mathscr{B}^p\). Es decir, $P_X = P \circ \restr{\boldsymbol X^{-1}}{\mathscr{B}}\,.$
\end{ndef}

\begin{notacion}
  Cuando escribamos probabilidades de sucesos asociados a un vector aleatorio $\boldsymbol X$, usaremos la siguiente notación:
  \begin{itemize}
  \item $P[\boldsymbol X \in A] = P_X[A]$,
  \item $P[X_1 \le x_1, \dots, X_p \le x_p] = P_X[(-\infty, x_1]\times \dots \times (-\infty, x_p]]$,
  \end{itemize}
  y algunas formas análogas.
\end{notacion}

\begin{ndef}[Función de distribución]
    Se define la \emph{función de distribución} asociada a $P_{\boldsymbol X}$ como
    \begin{align*}
    F_{\boldsymbol X}:\mathbb{R}^p &\rightarrow [0,1] \\
    (x_1,\dots,x_p) &\mapsto P[X_1 \leq x_1, \dots, X_p \leq x_p]\,.
    \end{align*}
\end{ndef}

\begin{ndef}[Función de densidad]
    Si existe una función $f_{\boldsymbol X} : \R^p \to \R$ que sea integrable en el sentido de Lebesgue y tal que
    \[
    F_{\boldsymbol X}(\boldsymbol x) = \int^{x_1}_{-\infty} \dots \int^{x_p}_{-\infty} f_{\boldsymbol X}(u_1, \dots,  u_p) \mathop{}\!\mathrm{d}u_1 \dots \mathop{}\!\mathrm{d}u_p\quad \text{para todo } \boldsymbol  x \in \mathbb{R}^p
    ,\]
    diremos que $f_{\boldsymbol X}$ es una \emph{función de densidad} asociada a  $F_{\boldsymbol X}$. En los puntos de continuidad de $f_{\boldsymbol X}$, la podemos escribir como:
    \[
    f_{\boldsymbol X}(\boldsymbol x) = \frac{\partial^p}{\partial x_1 \dots \partial x_p} F_{\boldsymbol X}(\boldsymbol x)
    .\]
\end{ndef}

\subsection{Independencia}

Veamos ahora algunas nociones relativas a la independencia de variables aleatorias.

\begin{ndef}[Conjunto rectangular]
  Se dice que $B\in \mathscr{B}^p$ es un \emph{conjunto rectangular} si es un producto de $p$ conjuntos de $\mathscr{B}$. Es decir, si
\[
    B = B_1 \times \dots \times B_p,\quad\text{con }B_j \in \mathscr{B} \quad\text{para } j = 1, \dots, p
.\]
\end{ndef}

Observamos que en general, dado un conjunto rectangular $B \in \mathscr{B}^p$ y un vector aleatorio $\boldsymbol X$ se tiene que
\[
P_{\boldsymbol X}[B] \neq P_{X_1}[B_1] \cdots  P_{X_p}[B_p],
\]
lo que motiva la siguiente definición.

\begin{ndef}[Independencia]
    Sea $\boldsymbol X = (X_1, \dots, X_p)$ un vector aleatorio. Si para todo conjunto rectangular $B \in \mathscr{B}^p$ se verifica que \[
      P_{\boldsymbol X}[B] = P_{X_1}[B_1] \cdots  P_{X_p}[B_p],
      \] se dice que las variables aleatorias \(X_1, \dots, X_p\) son \emph{independientes}.
\end{ndef}

\begin{nprop}[Caracterizaciones de independencia]
  Dado un vector aleatorio $\boldsymbol X = (X_1, \dots, X_p)$ sus componentes son independientes si, y solo si se da alguna de las siguientes condiciones:
  \begin{nlist}
  \item La función de distribución verifica $F_{\boldsymbol X} (\boldsymbol x) = F_{X_1} (x_1) \cdots F_{X_p}(x_p)\,$ para todo \(\boldsymbol x \in \mathbb{R}^p\).
   \item $\boldsymbol X$ tiene una función de densidad y \(f_{\boldsymbol X} (\boldsymbol x) = f_{X_1}(x_1) \cdots f_{X_p}(x_p)\,\) para todo \(\boldsymbol x \in \mathbb{R}^p\), salvo, a lo sumo, en un conjunto de medida de Lebesgue nula.
  \end{nlist}
\end{nprop}

\subsection{Función característica}

\begin{ndef}[Función característica] \label{funcioncaracteristica}
    La \emph{función característica} de un vector aleatorio \(\boldsymbol X = (X_1,\dots,X_p)\) se define como la función \[\psi_{\boldsymbol X}(\boldsymbol t)=E\left[e^{i\boldsymbol t^T\boldsymbol X}\right], \quad\text{para } \boldsymbol t\in \mathbb{R}^p.\]
\end{ndef}

\begin{nth}[Unicidad]
  La función característica de un vector aleatorio determina de forma única su distribución.
\end{nth}

El siguiente resultado nos proporciona otra caracterización de la independencia de las componentes de un vector aleatorio, análoga a las descritas anteriormente pero ahora utilizando la función característica recién definida.

\begin{nprop}
  Las componentes de \(\boldsymbol X=(X_1,\dots,X_p)^T\) son independientes si, y solo si

  \[
    \psi_{\boldsymbol X}(\boldsymbol t) = \prod_{k=1}^p\psi_{X_k}(t_k)\,
  .\]
\end{nprop}

\begin{nprop}
  Si las componentes de \(\boldsymbol X=(X_1,\dots, X_p)\) son independientes, entonces la función característica de la variable $Y=\sum_{k=1}^p X_k$ es

  \[
    \psi_{Y}(t) = \prod_{k=1}^p\psi_{X_k}(t)
  .\]
\end{nprop}

\subsubsection{Transformaciones lineales}

A continuación estudiaremos qué forma tiene la función característica de un vector aleatorio cuando éste es el resultado de someter otro vector aleatorio a transformaciones lineales.

Sea $\boldsymbol X = (X_1, \dots X_p)^T$ un vector aleatorio $p-$dimensional y sea $\boldsymbol Y = (Y_1,\dots,Y_q)^T$ otro vector aleatorio $q-$dimensional definido como $\boldsymbol Y = B\boldsymbol X + \boldsymbol b$, con $B \in \mathscr{M}_{q\times p}(\R)$ una matriz constante y $\boldsymbol b\in \mathbb R ^n$ un vector también constante.

\begin{nprop}
  La función característica de $\boldsymbol Y$ se obtiene como:
  
  \[
  \psi_{\boldsymbol Y}(\boldsymbol t) = e^{i\boldsymbol t^T \boldsymbol b} \psi_{\boldsymbol X}(B^T \boldsymbol t)\,, \quad \text{para }\boldsymbol t = (t_1,\dots,t_q)^T \in \mathbb R^q\,.
  \]
\end{nprop}

  En particular, si \[ \vectX= \left( X_{(1)} | X_{(2)}\right)^T := \left(X_{(1),1}, \dots, X_{(1),k}, X_{(2),1}, \dots, X_{(2),p-k}\right),\] con $X_{(1)}$ de dimensión $(k\times1)$ y $X_{(2)}$ de dimensión $(p-k) \times 1$, se tendría que
  \[
X_{(1)} = \left( I_k | 0_{p-k} \right)X + 0_{k\times 1}\,,
\]
por lo que, para $\boldsymbol t_{(1)} = (t_1,\dots,t_k)^T \in \mathbb R^k$ la función característica viene determinada por
\[
\psi_{X_{(1)}}\left(\boldsymbol t_{(1)}\right) = \underbrace{e^{i\boldsymbol t_{(1)}^T 0}}_{=1} \psi_X\left(\left( I_k | 0_{p-k} \right)\boldsymbol t_{(1)}\right) = \psi_X\left(\frac{\boldsymbol t_{(1)}}{0_{(p-k)}}\right)\,.
\]

\subsubsection{Relación con la función de densidad}

La función de densidad (en el caso continuo) de un vector aleatorio $\boldsymbol X = (X_1,\dots,X_p)^T$ y la correspondiente función característica constituyen un par de \emph{transformadas de Fourier}, de forma que
\[
\psi_{\boldsymbol X}(\boldsymbol t) = \int_{\mathbb R ^p} e^{i \boldsymbol t^T \boldsymbol x} f_X(\boldsymbol x) \mathop{}\!\mathrm{d}\boldsymbol x\,,
\]
y por otro lado
\[
f_{\boldsymbol X}(\boldsymbol x) =  \dfrac{1}{(2\pi)^p} \int_{\mathbb R^p} e^{- i \boldsymbol t^T x} \psi_{\boldsymbol X}(t) \mathop{}\!\mathrm{d}\boldsymbol t\,.
\]


\subsection{Aspectos generales sobre vectores aleatorios}

En este apartado definiremos y estudiaremos algunas de las propiedades de los vectores aleatorios con las que trabajaremos habitualmente.

\subsubsection{Esperanza y covarianza}

Para definir la esperanza de un vector aleatorio partimos de la definición de esperanza de una variable aleatoria y la extendemos de forma natural al caso multivariante.

\begin{ndef}
  Sea $\boldsymbol X=(X_1,\dots,X_n)^T$ un vector aleatorio. Se define el \emph{vector de medias} de $\boldsymbol X$ como:
  \[
  \boldsymbol \mu_{\boldsymbol X} = E[\boldsymbol X] = \begin{pmatrix}  E[X_1] \\ \dots \\ E[X_p] \end{pmatrix} = \begin{bmatrix} \mu_1 \\ \dots \\ \mu_p \end{bmatrix}
  \]
  siempre que existan \emph{todas} las esperanzas unidimensionales.
  \end{ndef}

  Observamos a continuación que la propiedad de linealidad de la esperanza en las variables aleatorias se traslada trivialmente al caso multivariante.

\begin{nprop}[Propiedad de linealidad]
  Sea $\boldsymbol Y = B\boldsymbol X + \boldsymbol b$ con $B \in M_{q\times p}$ constante y $\boldsymbol b \in \mathbb R^q$ constante. Entonces,
  \[
    \boldsymbol \mu_{\boldsymbol Y} = E[\boldsymbol Y] = BE[\boldsymbol X] + \boldsymbol b = B\boldsymbol \mu_{\boldsymbol X} + \boldsymbol b\,.
\]
\end{nprop}
%#TODO: Demostración(ejercicio)

Podemos incluso seguir iterando este proceso y extender esta propiedad al caso de matrices aleatorias. Sea $X\in \mathcal M_{p\times q}$ es una matriz aleatoria, y $B\in \mathcal M_{m\times p}$ , $C \in \mathcal M_{q\times n}$ , $D \in \mathcal M_{m \times n}$ matrices constantes. Entonces, para $W = BXC + D$, una matriz aleatoria de dimensión $m\times n$, la matriz de medias viene dada por
\[
E[W] = \left(E[W_{kl}]\right)_{(kl)} = B \cdot E[X] \cdot C + D\,, \quad \text{con } E[X] = \left(E[X_{ij}]\right)_{(ij)}\,.
\]

Como comprobaremos a continuación, la definición de la matriz de covarianzas no es más que una extensión al caso multivariante de la definición de covarianza de dos variables aleatorias.

\begin{ndef}
  Se define la \emph{matriz de covarianzas} del vector $\boldsymbol X = (X_1,\dots,X_p)^T$ como:
  \[
\Sigma = \operatorname{Cov}(\boldsymbol X) = E\left[(\boldsymbol X-\boldsymbol \mu)(\boldsymbol X-\boldsymbol \mu)^T\right] = \begin{bmatrix} \sigma_{11} & \dots & \sigma_{1p} \\ \vdots& \ddots & \vdots \\ \sigma_{p1} &  \dots & \sigma_{pp}\end{bmatrix}\,,
\]
donde $\sigma_{ij} = E\left[(X_i - \mu_i)(X_j - \mu_j)\right] = \sigma_{ji}$ es la covarianza de las variables aleatorias $X_i$ y $X_j$. Solo puede definirse si existen todas las covarianzas. 
\end{ndef}

\begin{nota}
  Los elementos de la diagonal de la matriz de covarianzas se pueden calcular como \[\sigma_{ii}=E\left[(X_i - \mu_i)^2\right] = \operatorname{Var}(X_i) = \sigma_i^2\,,\]siendo $\sigma_i$ la \emph{desviación típica}.
\end{nota}

\begin{nota}
  La desigualdad de Cauchy-Schwarz nos dice que existen todas las esperanzas si, y solo si existen las varianzas de cada componente.
\end{nota}

\begin{nprop}[Propiedades elementales]
  La matriz de covarianzas verifica las siguientes propiedades elementales:
  \begin{enumerate}
    \item Es simétrica.
    \item Los elementos de su diagonal son no negativos.
    \item La clase de matrices de covarianzas en dimensión $p\times p$ coincide con la clase de matrices simétricas definidas no negativas.
    \end{enumerate}
  \end{nprop}

\begin{proof}
  La comprobación de las dos primeras propiedades es inmediata, por lo que vamos a ver únicamente la demostración de la propiedad 3.
  \begin{nlist}
    \item[3.] $\boxed{\implies}\,$ Supongamos que $\Sigma$ es la matriz de covarianza de un vector aleatorio $\boldsymbol X$, y $\boldsymbol \mu_{\boldsymbol X}$ es su vector de medias. Sea $\boldsymbol \alpha \in \mathbb R^p$. Tenemos que probar que $\boldsymbol \alpha^T \Sigma \boldsymbol \alpha^T \geq 0$. Consideramos la variable aleatoria $\boldsymbol \alpha^T \boldsymbol X$. \begin{align*}
      0 \leq \operatorname{Var}(\alpha^T X) = E\left[\|\alpha^T X - E[\alpha^T X]\|^2\right] &= E\left[\|\alpha^T X - \alpha^T \mu_X\|^2\right] = E\left[\|\alpha^T(X-\mu)\|^2\right]\\ 
        &=^{(1)} E\left[\alpha^T(X-\mu)(X-\mu)^T \alpha\right] \\
        &= \alpha^T E\left[(X-\mu)(X-\mu)^T\right] \alpha \\
        &= \alpha^T \Sigma \alpha
    \end{align*} Donde en $(1)$ hemos separado el cuadrado y hemos traspuesto uno de los términos (el resultado es el mismo, un escalar).

    $\boxed{\impliedby}\,$ Sea $A$ una matriz simétrica, definida no negativa. Entonces, se puede encontrar una factorización $A = B B^T$. donde $C \in \mathscr{M}_{p}(\R)$.
    Consideramos un vector aleatorio $\vectX$ cuyas entradas son $p$ variables independientes e idénticamente distribuidas con distribución normal estándar. Entonces, $\Sigma_{\vectX} = I_p$. Esto se puede comprobar usando que $E[XY] = E[X]E[Y]$ si $X$ e $Y$ son variables aleatorias independientes. Consideramos también $\vectY = B\vectX$, entonces:

    \[
       \Sigma_{\vectY} = B\Sigma_{\vectX} B^T = B B^T = A
    .\]

  \end{nlist}
\end{proof}

%#TODO: revisar el estilo de esta parte, quizá añadir una subsección?
Vamos a ver ahora diferentes \textbf{casos} que se pueden dar respecto a la matriz $\Sigma$.

\paragraph{Si $\Sigma$ es definida positiva} Se nota como $\Sigma > 0$. En este caso $\Sigma$ es \emph{no singular} pues $(|\Sigma| > 0)$, y por tanto existe su inversa, $\Sigma^{-1}$. Esto es interesante pues nos permite normalizar el vector aleatorio de manera que tenga un vector de medias cero y una matriz de covarianzas que sea la identidad.

\begin{nota}
  Usaremos la notación $X\sim(\mu,\Sigma)$ para denotar que la media y la matriz de covarianzas están bien definidas.
\end{nota}

\begin{ndef}[Normalización]
    Dado un vector aleatorio $\boldsymbol X\sim(\mu,\Sigma)$, con $\Sigma>0$, se define su \emph{normalización} en «origen» y «escala» como
    \[
  Z = C^{-1}(X - \mu)\,,
\]
para cualquier matriz $C\in M_{p\times p}$ tal que $E = CC^T$. Esta matriz $C$ no es única, pues existen infinitas descomposiciones de $\Sigma$ de esta forma.

  \end{ndef}

Veamos a continuación algunas de las propiedades de esta variable $Z$.
\begin{nprop}
  Una normalización de un vector aleatorio tiene media $0$ y matriz de covarianzas identidad.
\end{nprop}

\begin{proof} \hfill\\
    \[
    E[Z] = E[C^{-1}(X-\mu)] = C^{-1}E[(X-\mu)] = C^{-1}(E[X] - \mu) = C^{-1}(\mu - \mu ) = 0,
    \]

    \[
    \operatorname{Cov}(Z) = E[ZZ ^T] = E[C^{-1}(X-\mu)(X-\mu)^T(C^{-1})^T]
    \]

    y, por la propiedad de las matrices que nos dice que $(C^{-1})^T = (C^T)^{-1}$, se tiene en la igualdad anterior:

    \[
      \operatorname{Cov}(Z) = C^{-1}E[(X-\mu)(X-\mu)^T](C^T)^{-1} = C^{-1} \Sigma (C^T)^{-1} = C^{-1}CC^T(C^T)^{-1} = I \cdot I = I
    .\]

\end{proof}
  \begin{ndef}
    Se define la \emph{distancia de Mahalanobis} de $\vectX\sim(\mu,\Sigma)$ con $\Sigma > 0$ con respecto a su vector de medias como
    \[
    \Delta(\vectX,\mu) = \left\{ (\vectX-\mu)^T \Sigma (\vectX-\mu) \right\} ^{\frac{1}{2}},
    \]
    que es una variable aleatoria.
  \end{ndef}

  Vamos a interpretar esta definición. Sea $\vectZ$ una normalización de $\vectX$:
  \[
  \Delta(\vectX,\mu) = \left\{ (\vectX-\mu)^T (CC^T)^{-1}(\vectX-\mu)\right\}^{\frac{1}{2}} = \left\{ (\vectX-\mu)^T (C^T)^{-1}C^{-1}(\vectX-\mu)\right\}^{\frac{1}{2}} = \left\{\vectZ^T \vectZ\right\}^{\frac{1}{2}} = \Vert \vectZ\Vert\,,
  \]
  con norma la euclídea en $\mathbb R^p$. Así, esta distancia es la norma de cualquier normalización de $\vectX$. Recordemos que $\Vert \vectZ\Vert$ es una variable aleatoria, pues es una aplicación de una función medible a un vector aleatorio.

  \begin{nprop} Sea $\vectX = (X_1, \dots, X_p)^T \sim (\boldsymbol \mu, \Sigma)$ un vector aleatorio con $\Sigma > 0$. Entonces,
    \begin{nlist}
    \item $E[\Delta^2 (X,\mu)] = p$,
    \item la ecuación $\Delta(X,\mu) = k$ —con $k\geq 0$ constante— define la \textit{hipervariedad de contorno} correspondiente a aquellos puntos $x \in \mathbb R^p$ tales que, en el espacio transformado $\mathbb R^p$ por
      \[
      x \mapsto z = C^{-1}(x-\mu)
      \]
      se sitúan en la esfera euclídea $p-$dimensional de radio $k$ y centro en el origen.
    \end{nlist}

  \end{nprop}

\paragraph{$\Sigma$ es \textbf{semidefinida positiva}} (esto es, $\Sigma \geq 0$ con $|\Sigma| = 0$).  En este caso $\Sigma$ es singular, por lo que no existe $\Sigma ^{-1}$. Por tanto, se tiene que $\operatorname{rango}(\Sigma) = r < p$ y $\Sigma = C C^T$ con $C \in \mathscr{M}_{p\times r}(\R)$ y cuyo rango es $r$.\\

  \begin{ncor}
    Con probabilidad $1$, las componentes del vector $\vectX = (X_1, \dots, X_n)$ cumplirán una relación de dependencia lineal del tipo:
    \[
    \alpha^T \vectX = k, \quad \quad \alpha \in \mathbb R^p, \quad \alpha \ne 0 , \quad k \in \R
    \]
    
  Por tanto, toda la variabilidad de $\vectX$ se sitúa en un hiperplano afín (casi seguramente con respecto a la medida $P$).
\end{ncor}

  \begin{proof}
    Como $\det(\Sigma) = 0$, $\exists \alpha \ne 0$ tal que $\alpha^T \Sigma \alpha = 0$. Definimos la variable aleatoria $\vectY = \alpha^T \vectX$, y se tiene que:
    
    \[
      E[\vectY] = E[\alpha^T \vectX] = \alpha^T E[\vectX] = \alpha^T \mu =: k
    ,\] y
    
     \[
        \begin{split}
          \operatorname{Var}(\vectY) & = E\left[\|\vectY-k\|^2\right]  = E\left[\|\alpha^T\vectX - \alpha^T \mu\|^2\right] \\
          & = E\left[\|\alpha^T(\vectX-\mu)\|^2\right] = E\left[\alpha^T(\vectX-\mu)(\vectX-\mu)^T\alpha\right] \\
          & = \alpha^T E\left[(\vectX-\mu)(\vectX-\mu)^T\right]\alpha = \alpha^T \Sigma \alpha = 0
        \end{split}  
     \]
      con lo que $\vectY$ es una variable aleatoria degenerada: $P[\vectY = k] = 1 \iff \alpha^T \vectX = k$ c.s.
  \end{proof}

Dicho esto, existen varias \textbf{medidas de variación global}.Podemos distinguir, dado $\Sigma$, y $\lambda_j$ con $j = 1,\dots,p$ los autovalores de $\Sigma$:
\begin{nlist}
\item $\det(\Sigma) =  \prod_{j=1}^p \lambda_j$,
  \item $\traza(\Sigma) = \sum_{j = 1}^p \sigma_j^2 = \sum_{j=1}^p \lambda_j$
\end{nlist}


\subsection{Caracterización de la distribución de un v.a. en términos de las distribuciones univariantes de c.l. de sus componentes}

\begin{nth}
  Sea $\vectX = (X_1,\dots, X_n)^T$ un vector aleatorio. Se tiene que la distribución de $\vectX$ queda unívocamente determinada por las distribuciones univariantes de la forma
  \[
\alpha ^T \vectX, \quad \alpha \in \mathbb R ^p\,.
  \]
\end{nth}
\begin{proof}
  Sea $Y_\alpha = \alpha^T \vectX$ (variable aleatoria unidimensional), para cada $\alpha \in \mathbb R^p$. La función característica de $Y_\alpha$ viene dada por
  \[
  \psi_{Y_\alpha}(t) = E\left[e^{itY_\alpha}\right] = E \left[e^{it(\alpha^T \vectX)}\right] \quad \text{para todo } t \in \mathbb R\,.
  \]
  En particular, si $t=1$
  \[
  \psi_{Y_\alpha}(1) = E\left[e^{i\alpha^T \vectX}\right] = \psi_{\vectX} (\alpha)
  \]
  Es decir:
  \[
  \psi_{\vectX}(t) = \psi_{Y_t}(1), \quad \forall t \in \mathbb R ^p
  \]
\end{proof}

\subsection{Momentos y cumulantes}

Sea $\vectX= (X_1,\dots,X_p)^T$ un vector aleatorio con función característica $\phi_{\vectX}$.
\begin{ndef}
  Se define el \textbf{momento} (no centrado) $p-$dimensional de orden $(r_1, \dots, r_p)$ de $X$ como:
  \[
\mu_{r_1,\dots,r_p}^{1,\dots,p} = E \left[ X_1^{r_1} \dots X_p^{r_p}\right]\,.
  \]
\end{ndef}

Los momentos se pueden obtener a partir de la función característica derivándose de su expansión de Taylor(respecto al origen).

\[
\phi_X(t) = E\left[e^{it^T X}\right] = E\left[\sum_{r=0}^\infty \frac{1}{r!}(i t^T X)^r\right] = \sum_{r= 0}^\infty \sum _{r_1+\dots + r_p = r} \mu_{r_1 \dots r_p}^{1 \dots p} \frac{(it_1)^{r_1} \dots (it_p)^{r_p}}{r_1 ! \dots r_p !}
\]
En particular, podemos enunciar este teorema:
\begin{nth}
  Si $E=\left[|X_1|^{m_1} \dots |X_p|^{m_p}\right] < \infty$, entonces la función característica de $X$ es $(m_1, \dots, m_p)$ veces diferenciable y , si $m = m_1 + \dots + m_p$,
  \[
\frac{\partial ^m}{\partial t_1^{m_1} \dots \partial t_p^{m_p}} \phi_X(t)| _{t = 0} = i^m \mu_{m_1}^1 \dots \mu_{m_p}^p
  .\]
\end{nth}

Consideremos el logaritmo de la función característica:
\[
\log \phi(t).
\]
\begin{ndef}
  Se definen los \emph{cumulantes} ($p-$dimensionales) de orden $(r_1 , \dots, r_p)$ como los coeficientes de la correspondiente expansión:
  \[
  \log \phi_X(t) =  \sum_{r= 0} \sum_{r_1+\dots + r_p = r} \mathcal K_{r_1 \dots r_P}^{1\dots p} \frac{(it_1)^{r_1} \dots (it_p)^{r_p}}{r_1 ! \dots r_p !}
  .\]
\end{ndef}


\subsubsection{Cambio de variables}

\begin{nth} \label{cambiovariablealeatoria}
  Sea $\vectX = (X_1,\dots,X_n)^T$ un vector aleatorio con función de densidad $f_X(x)$ positiva sobre $S \subseteq \mathbb R^d$ y continua. Sea $Y = (Y_1,\dots,Y_p)^T$ un vector aleatorio con:
  \[
  Y = g(X) = (g_1(X),\dots ,g_p(X))^T
  \]
  con $g = (g_1,\dots,g_p): S \to \mathcal T$ biyectiva, por lo que $\exists g^{-1}$, y denotamos $g^{-1} = h = (h_1,\dots, h_p)^T : \mathcal T \to S$.

  Supongamos que existen las derivadas parciales:
  \[
    \frac{\partial h_i(y)}{\partial y_j} \quad (i,j = 1,\dots,p)
  \]
  y que son continuas sobre $\mathcal T$.

  Entonces, la función de densidad $f_Y(y)$ del vector $Y=(Y_1,\dots,Y_p)^T = y(X)$ viene dada por:
  \[
  f_Y(y) = f_X(g^{-1}(y))\left|\det(J_{g^{-1}}(y))\right|
  .\]
\end{nth}

\begin{ncor}[Caso lineal] \label{cambiocasolineal}
  Sea $Y = BX + b$ con $B\in \mathscr{M}_{p}(\R)$ y no singular. En este caso,
  \[
  J_{g^{-1}}(g(x)) = \left[ J_{g}(x) \right]^{-1} = \det(B)^{-1} = \det\left(B^{-1}\right)
  \]
  y se tiene:
  \[
  f_Y(y) = f_X\left(B^{-1}(y-b)\right)\left|\det(B)^{-1}\right|
  .\]

\end{ncor}
\begin{nota}
  Al factorizar una matriz definida positiva, siempre existe una factorización $CC^T$ con $\det(C) > 0$.
  \end{nota}

\subsection{Distribución normal multivariante (en el caso $\Sigma > 0$)}
  En esta sección estudiamos la distribución normal multivariante, refiriéndonos siempre al caso en el que tiene matriz de covarianzas definida positiva.
Algunos aspectos que justifican el estudio de esta distribución en particular:
\begin{itemize}
\item es la distribución multivariante más estudiada, y existen infinidad de resultados acerca de la misma,
\item DNM es la base de muchas técnicas del análisis multivariante confirmatorio o inferencial,
\item DNM es una extensión \emph{natural} de la DNM univariante al caso multivariante,
\item las distribuciones marginales de cualquier orden en una DNM son también normales,
\item las distribuciones condicionadas (internamente) en una DNM también son normales y se obtienen de forma sencilla,
\item la familia de DNM es cerrada bajo transformaciones lineales,
\item la familia de DNM es cerrada bajo combinaciones lineales de vectores (mutuamente) independientes,
\item la DNM viene determinada por los momentos de primer y segundo orden (media y matriz de covarianzas),
  \item si dos subvectores de un vector aleatorio con DNM tienen correlaciones cruzadas nulas, entonces dichos subvectores son (mutuamente) independientes.

\end{itemize}


\begin{ndef}
  Sea $\vectX = (X_1,\dots, X_p)^T$ un vector aleatorio. Se dice que $\vectX$ tiene una distribución normal $p-$variante si su densidad es de la forma:
  \[
f_{\vectX}(x) = \frac{1}{(2\pi)^{p/2}\det(\Sigma)^{\frac{1}{2}}} \exp\left\{- \dfrac{1}{2}(x-\mu)^T \Sigma^{-1}(x-\mu)\right\}
\]
Siendo $\mu = (\mu_1, \dots, \mu_p)^T \in \mathbb R^p$ y $\Sigma$ una matriz escalar $p\times p$ simétrica y definida positiva.
\end{ndef}

Veamos que, en efecto, $f_{\vectX}$ es una función de densidad
\begin{enumerate}
\item $f_{\vectX}(x) \geq 0, \ \forall x \in \mathbb R$,
\item $\int_{\mathbb R^p} f_{\vectX}(x) \mathrm{d}x = 1$. En efecto, como $\Sigma$ es una matriz simétrica y definida positiva, podemos descomponerla de la forma $\Sigma = C C^T$ con $C \in \mathcal \mathscr{M}_{p}(\R), \det(C) > 0$. La forma de la densidad sugiere el cambio de variable $Z = C^{-1}(x-\mu)$.  El determinante del jacobiano de la transformación es $\det\left(C^{-1}\right)$. Por otro lado:

\[
  (x-\mu)^T \Sigma^{-1}(x-\mu) = zz^T
\]

Por tanto, podemos escribir:

\begin{DispWithArrows*}[fleqn, mathindent = 0cm, wrap-lines]
      \int_{\R^p} f_{\vectX}(x) \mathrm{d}x & = \int_{\mathbb R^p} \frac{1}{(2\pi)^{p/2}\det(\Sigma)^{\frac{1}{2}}} \exp\left\{- \dfrac{1}{2}(x-\mu)^T \Sigma^{-1}(x-\mu)\right\} \mathrm{d}x \Arrow{cambio de variable} \\
      & = \int _{\mathbb R^p} \frac{1}{(2\pi)^{\frac{p}{2}}\det(C)^{\frac{1}{2}} \det(C^T)^{\frac{1}{2}}} \exp\left\{ -\frac{1}{2}zz^T\right\} \det(C) \mathrm{d}z\\
      & = \int_{\mathbb R^p} \frac{1}{\left(\sqrt{2\pi}\right)^p}\exp\left\{-\frac{1}{2}\sum_{j=1}^p z_j^2\right\} \mathrm{d}z_1 \dots \mathrm{d}z_p\\
      & = \prod_{j=1}^p \int_{\mathbb R} \underbrace{ \frac{1}{\sqrt{2\pi}} \exp\left\{- \frac{1}{2} z^2\right\} }_{\text{densidad de una $N(0,1)$}} \mathrm{d}z_j = \prod_{j=1}^p 1 = 1.
\end{DispWithArrows*}
\end{enumerate}


\subsubsection{Vector de medias y matriz de covarianzas}
Del cálculo anterior se desprende que el vector aleatorio \[ \vectZ = C^{-1}(\vectX-\mu) \] se distribuye con función de densidad
\[
f_{\vectZ}(z) = \prod_{j = 1}^p \frac{1}{\sqrt{2\pi}}\exp\left\{-\frac{1}{2}z_j^2\right\} = \prod_{j=1}^p f_{Z_j}(z_j)
\]
Es decir, para cada $j = 1,\dots, p$
\[
Z_j \sim N(0,1)
\]
siendo $Z_1,\dots,Z_p$ independientes (en particular, incorreladas), luego $\vectZ \sim N_p(0,I_p)$.

A partir de esto, usando \ref{cambiovariablealeatoria}, se obtiene que $\mu$ y $\Sigma$ son, respectivamente, el vector de medias y la matriz de covarianzas del vector aleatorio $\vectX \sim N_p(\mu,\Sigma)$.


\begin{nprop}[Transformación afín sobre una normal] \label{afinnormal}
  Sea $\vectX \sim N_p(\mu,\Sigma)$, con $\Sigma > 0$ e $\vectY = B\vectX+b$ con $B$ matriz escalar $p\times p$ no singular y $b$ un vector aleatorio escalar $p\times 1$. Entonces, se tiene que:
  \[
     \vectY \sim N_p(B\mu+b, B \Sigma B^T)
  .\]
\end{nprop}

\begin{proof}
  De \ref{cambiocasolineal} se sigue que

  \[
  \begin{split}
    f_{\vectY}(y) & = f_{\vectX}(B^{-1}(y-b))\left|\det(B)^{-1}\right| \\
    & = \frac{1}{(2\pi)^{p/2} \det(\Sigma)^{1/2} \left|\det(B)\right|}
    \exp\left\{ -\frac{1}{2} \left(B^{-1}(y-c) - \mu\right)^T \Sigma^{-1} \left(B^{-1}(y-c) - \mu \right) \right\} \\
    & = \frac{1}{(2\pi)^{p/2} \det(B^T\Sigma B)}
    \exp\left\{ -\frac{1}{2} \left(y-c - B\mu\right)^T (B^T\Sigma B)^{-1} \left(y - c - B\mu \right) \right\}.
  \end{split}
  \]
\end{proof}

\begin{nth}[Caracterización de DNM]
  Sea $\vectX$ un vector aleatorio $p-$dimensional, entonces $\vectX$ tiene distribución normal multivariante, de vector de medias $\mu$ y matriz de covarianzas $\Sigma > 0$, si y solo si
  \[
\vectX = A\vectZ + \mu
\]
con $A\in \mathscr{M}_{p}(\R)$ no singular, $AA^T = \Sigma$, $\vectZ \sim N_p(0,I_p)$.
\end{nth}

%#TODO: Demostración (clase de prácticas)

\subsubsection{Sobre Independencia y Condicionamiento}

  \begin{nprop}
    Sea $\vectX = (X_1,\dots,X_p)^T$ un vector aleatorio con DNM $\vectX \sim N_p(\mu,\Sigma)$, $\Sigma > 0$. Si la matriz $\Sigma$ es diagonal,
    entonces las variables aleatorias componentes del vector son independientes y tienen distribución normal univariante $X_i \sim N\left(\mu_i, \sigma_i^2\right)$.
  \end{nprop}
  \begin{proof}
    Por hipótesis, sabemos que:
    \[
    f_X(x) = \frac{1}{(2\pi)^{\frac{p}{2}} \det(\Sigma)^{\frac{1}{2}}}\exp\left\{- \frac{1}{2}(x-\mu)^T \Sigma^{-1}(x-\mu)\right\}
    \]
    Y además:
    \begin{itemize}
    \item $\det(\Sigma)^{\frac{1}{2}} = \left(\prod_{i = 1}^p \sigma_i^2\right)^{\frac{1}{2}} = \prod_{i=1}^p \sigma_i$,
    \item $\Sigma^{-1} = \diag\left(\left(\sigma_1^2\right)^{-1}, \dots , \left(\sigma_p^2\right)^{-1}\right)$,
      \item $(x-\mu)^T \diag\left(\left(\sigma_1^2\right)^{-1}, \dots , \left(\sigma_p^2\right)^{-1}\right) (x-\mu) = \sum_{i=1}^p (x_i-\mu_i)^2 \sigma_i^{-2} = \sum_{i = 1}^p \left(\frac{x_i-\mu_i}{\sigma_i}\right)^2$.
    \end{itemize}
    Por tanto, reescribiendo:
    \[
      \begin{split}
        f_X(x) = \frac{1}{(2\pi)^{\frac{p}{2}} \prod \sigma_i} \exp \left\{ -\frac{1}{2}(x-\mu)^T \diag\left(\left(\sigma_1^2\right)^{-1}, \dots , \left(\sigma_p^2\right)^{-1}\right) (x-\mu)  \right\} =\\
        \frac{1}{(2\pi)^{\frac{p}{2}} \prod \sigma_i} \exp \left\{ -\frac{1}{2} \sum_{i = 1}^p \left(\frac{x_i-\mu_i}{\sigma_i}\right)^2  \right\} = \prod_{i = 1}^p  \frac{1}{(2\pi)^{\frac{p}{2}} \prod \sigma_i} \exp \left\{ -\frac{1}{2} \left(\frac{x_i-\mu_i}{\sigma_i}\right)^2  \right\} = \prod_{i = 1}^p f_{X_i}(x_i)
    \end{split}
    \]

  \end{proof}


  El recíproco de este último resultado dado es cierto:
  \begin{nprop}
    Si $\vectX = (X_1,\dots,X_p)^T$ es un vector aleatorio con componentes \emph{mutuamente} independientes que tienen DNM, $X_i \sim N(\mu_i, \sigma_i)$, con $\sigma > 0$. Entonces, $\vectX$ tiene DNM
    \[
    \vectX \sim N_p(\mu, \Sigma)
    \]
    con $\mu = (\mu_1,\dots,\mu_p)^T$ y $\Sigma = \diag(\sigma_1^2 , \dots , \sigma_p^2) > 0$.
  \end{nprop}

  \begin{nota}
Exigimos $\sigma_i > 0, \ \forall i = 1,\dots, p$, y entonces $\Sigma > 0$.
  \end{nota}

  \begin{nprop}
    Sea $\vectX$ una v.a. $N(0,1)$ y sea $W$ una v.a. con distribución $U(\{-1,1\})$, independiente de $\vectX$. Consideramos $\vectY = W\vectX$. Entonces, se verifica:
    \begin{itemize}
    \item $\vectY \sim N(0,1)$,
    \item $\vectX$ e $\vectY$ son incorreladas,
      \item $\vectX$ e $\vectY$ no son independientes. 
    \end{itemize}

  \end{nprop}
  Esta proposición es un contraejemplo para ver que , en general, incorrelación no implica independencia (aunque alguna de las marginales sea una normal).

  \begin{nth} \label{independenciacovbloques}
    Sea $\vectX = (X_1, \dots ,X_p)^T \sim N_p(\mu,\Sigma)$ con $\Sigma > 0$. Supongamos que las componentes de $\vectX$ están ordenadas de tal modo que, por la partición del vector $\vectX = \left(\vectX_{(1)}^T | \vectX_{(2)}^T\right)^T$ con $\vectX_{(1)} = (X_1,\dots,X_q)^T$ y $\vectX_{(2)} = (X_{q+1},\dots,X_p)^T$, se tiene
    
    \[
    \mu = \left(\mu_{(1)}^T | \mu_{(2)}^T\right)^T, \quad \Sigma = \begin{pmatrix} \Sigma_{(11)} & 0 \\ 0 & \Sigma_{(22)} \end{pmatrix}
    .\]
    
    Entonces, $\vectX_{(1)}$ y $\vectX_{(2)}$ son mutuamente independientes y además, $\vectX_{(1)} \sim N_q(\mu_{(1)}, \Sigma_{(11)})$ y $\vectX_{(2)} \sim N_{p-q}(\mu_{(2)}, \Sigma_{(22)})$.
  \end{nth}

  \begin{proof}
    Comenzamos notando que, como $\Sigma = \diag(\Sigma_{(11)},\Sigma_{(22)})$, entonces:
    \begin{itemize}
    \item $\det(\Sigma)^{\frac{1}{2}} = \det(\Sigma_{(11)})^{\frac{1}{2}} \det(\Sigma_{(22)})^{\frac{1}{2}}$
      \item $ \Sigma^{-1} = \diag(\Sigma_{(11)}^{-1}, \Sigma_{(22)}^{-1})$
    \end{itemize}

    De esta forma, tenemos:
    
    \[
    \begin{split}
      (x-\mu)^T \Sigma^{-1}(x-\mu) & = \left( (x_{(1)}- \mu_{(1)})^T | (x_{(2)}-\mu_{(2)})^T \right) \begin{bmatrix} \Sigma_{(11)} & 0 \\ 0 & \Sigma_{(22)} \end{bmatrix}
      \left(\begin{array}{c} x_{(1)} - \mu_{(1)} \\ \hline
        x_{(2)} - \mu_{(2)} \end{array}\right) \\
       & = (x_{(1)}- \mu_{(1)})^T \Sigma_{(11)}^{-1}  (x_{(1)}- \mu_{(1)}) +  (x_{(2)}- \mu_{(2)})^T \Sigma_{(22)}^{-1}  (x_{(2)}- \mu_{(2)})
      \end{split}
    \]
    
    Podemos factorizar entonces la función de densidad de $\vectX$:
    
    \[
    \begin{split}
    f_{\vectX}(x) =  \frac{1}{(2\Pi)^{\frac{q}{2}} \det(\Sigma_{(11)})^{\frac{1}{2}} } \exp \left\{ - \frac{1}{2} (x_{(1)}- \mu_{(1)})^T \Sigma_{(11)}^{-1}  (x_{(1)}- \mu_{(1)})\right\} \\
    \cdot \frac{1}{(2\Pi)^{\frac{p-q}{2}} \det(\Sigma_{(22)})^{\frac{1}{2}} } \exp \left\{ - \frac{1}{2} (x_{(2)}- \mu_{(2)})^T \Sigma_{(22)}^{-1}  (x_{(2)}- \mu_{(2)})\right\}
    \end{split}
    \]
    
  \end{proof}

  El recíproco también es cierto, en el sentido en el que lo era el recíproco del resultado anterior.

  \begin{nth}
    Sea $\vectX = (X_1, \dots, X_p)^T \sim N_p(\mu,\Sigma)$ con $\Sigma > 0$. Supongamos el particionamiento de $\vectX$ como en el resultado anterior, pero ahora con
    
    \[
\Sigma = \begin{pmatrix}  \Sigma_{(11)} & \Sigma_{(12)} \\ \Sigma_{(21)} & \Sigma_{(22)} \end{pmatrix}
\]

Entonces, se tiene:
\begin{enumerate}
\item $\vectX_{(1)}$ y $\vectX_{(2)} - \Sigma_{(21)} \Sigma_{(11)}^{-1}\vectX_{(1)}$ son independientes, \label{indepitem}
\item \[
  \begin{cases}
    \vectX_{(1)} \sim N_q(\mu_{(1)}, \Sigma_{(11)}) \\
    \vectX_{(2)} - \Sigma_{(21)} \Sigma_{(11)}^{-1}X_{(1)} \sim N_{p-q}\left(\mu_{(2)} - \Sigma_{(21)}\Sigma_{(11)}^{-1}\mu_{(1)} \ , \ \Sigma_{(22)} - \Sigma_{(21)} \Sigma_{(11)}^{-1} \Sigma_{(12)}\right)
  ,\end{cases}
  \] \label{distitem}
  
\item La distribución condicionada de $\vectX_{(2)}$ dado $\vectX_{(1)} = x_{(1)}$ es una DNM
  
  \[
    N_{p-q}\left(\mu_{(2)} + \Sigma_{(21)}\Sigma_{(11)}^{-1}(x_{(1)} - \mu_{(1)}) \ , \ \Sigma_{(22)} - \Sigma_{(21)} \Sigma_{(11)}^{-1} \Sigma_{(12)}\right)
  .\] \label{conditem}
  
\end{enumerate}
  \end{nth}
  \begin{proof}
    Sea $C = \begin{pmatrix} I_1 & 0 \\ -\Sigma_{(21)}\Sigma_{(11)}^{-1} & I_2\end{pmatrix}$. $C$ es no singular, pues $\det(C) = 1$. Consideramos el cambio lineal $\vectY = C \vectX$:
      
   \[
      \vectY = \begin{pmatrix} \vectX_{(1)} \\ \vectX_{(2)} - \Sigma_{(21)}\Sigma_{(11)}^{-1}\vectX_{(1)}\end{pmatrix} := \begin{pmatrix} \vectY_{(1)} \\ \vectY_{(2)} \end{pmatrix}.
   \]

   Por \ref{afinnormal}, $\vectY \sim N_p\left( C\mu, C\Sigma C^T \right)$. Se tiene:

   \[
      C\mu = \left(
      \begin{array}{c}
        \mu_{(1)} \\ \hline
        \mu_{(2)} - \Sigma_{(21)}\Sigma_{(11)}^{-1}\mu_{(1)}
      \end{array}
      \right),
      \quad
      C\Sigma C^T = \left(
      \begin{array}{c | c}
        \Sigma_{(11)} & 0 \\ \hline
        0            & \Sigma_{(22\cdot 1)}
      \end{array}
      \right),
  \]

  donde $\Sigma_{(22\cdot 1)} := \Sigma_{(22)} - \Sigma_{(21)}\Sigma_{(11)}^{-1}\Sigma_{(12)}$. Así, \ref{indepitem} y \ref{distitem} se siguen de \ref{independenciacovbloques}.

  Para probar \ref{conditem} escribimos, usando el teorema de Bayes, la expresión de la de función de densidad de la distribución condicionada:

  \begin{DispWithArrows*}[fleqn, wrap-lines]
    f_{\vectX_{(2)} | \vectX_{(1)} = x_{(1)}}(x_{(2)}) & = \frac{f_{\vectX_{(1)}, \vectX_{(2)}} (x_{(1)}, x_{(2)})} {f_{\vectX_{(1)}}(x_{(1)})} \Arrow{usando el teorema de cambio de variable \ref{cambiocasolineal} con $\vectX = C^{-1}\vectY$, $\det(C) = 1$} \\
    & = \frac{f_{\vectY_{(1)}, \vectY_{(2)}}\left(x_{(1)}, x_{(2)} - \Sigma_{(21)}\Sigma_{(11)}^{-1}x_{(1)}\right)} {f_{\vectX_{(1)}}(x_{(1)})}
  \end{DispWithArrows*}


  Veamos el valor de la forma cuadrática que aparecerá en $f_{\vectY_{(1)}, \vectY_{(2)}}$:

  \[
     \left(
     \begin{array}{c}
       x_{(1)} \\ \hline
       x_{(2)} - \Sigma_{(21)}\Sigma_{(11)}^{-1}x_{(1)}
     \end{array}
     \right) - C\mu = \left(
     \begin{array}{c}
       x_{(1)} - \mu_{(1)} \\ \hline
       x_{(2)} - \mu_{(2)} - \Sigma_{(21)}\Sigma_{(11)}^{-1}(x_{(1)} - \mu_{(1)})
     \end{array}
     \right) := D(x_{(2)}),
   \]

   \[
   \begin{split}
     D(x_{(2)})^T (C\Sigma C^T) D(x_{(2)}) = \overbrace{(x_{(1)} - \mu_{(1)})^T \Sigma_{(11)}^{-1} (x_{(1)} - \mu_{(1)})}^{:= A} \\
     + \underbrace{(x_{(2)} - \mu_{(2)} - \Sigma_{(21)}\Sigma_{(11)}^{-1}(x_{(1)} - \mu_{(1)}))^T \Sigma_{(22\cdot 1)}^{-1}
       (x_{(2)} - \mu_{(2)} - \Sigma_{(21)}\Sigma_{(11)}^{-1}(x_{(1)} - \mu_{(1)}))}_{:= B(x_{(2)})}
   \end{split}
   \]  

   Resultando:

   \[
   \begin{split}
     f_{X_{(2)} | X_{(1)} = x_{(1)}}(x_{(2)}) & = \frac{ \frac{1}{(2\pi)^{p/2} (\det(\Sigma_{(11)})\det(\Sigma_{(22\cdot 1)}))^{1/2}} \exp\left\{ -\frac{1}{2} (A + B(x_{(2)}))  \right\} }
     { \frac{1}{(2\pi)^{q/2} \det(\Sigma_{(11)})^{1/2}} \exp\left\{ -\frac{1}{2} A \right\} } \\
     & = \frac{1}{(2\pi)^{\frac{p-1}{2}} \det(\Sigma_{(22\cdot 1)})^{1/2}} \exp\left\{ -\frac{1}{2} B(x_{(2)}) \right\}.
   \end{split}
   \]
  \end{proof}



  \subsection{Función característica de la DNM}

  Cabe recordar en esta sección la definición \ref{funcioncaracteristica}.

  \begin{nprop}[Función característica de la DNM]
    Dado un vector aleatorio $\vectX \sim N_p(\mu, \Sigma)$, su función característica viene dada por
    \[
      \psi_{\vectX}(t) = \exp\left\{ it^T\mu - \frac{1}{2} t^T \Sigma t \right\}
    .\]
  \end{nprop}

  
  \begin{proof}
    Tomamos una descomposición $\Sigma = CC^T$, con $C\in \mathscr{M}_{p}(\R)$ no singular, $\det(C) > 0$, y aplicamos el cambio de variable $\vectY = C^{-1}(\vectX - \mu)$.
    \[
    \begin{split}
      \Psi_{\vectX}(t) & = \int_{\mathbb R^p} e^{it^T x} f_{\vectX} (x) \mathrm{d}x \\
      & = \int_{R^p} \frac{1}{(2\Pi)^{\frac{p}{2}}\det(\Sigma)^{\frac{1}{2}}} \exp \left\{ it^Tx - \frac{1}{2}(x-\mu)^T\Sigma^{-1}(x-\mu) \right\}\mathrm{d}x \\
      & =  \int_{R^p} \frac{1}{(2\Pi)^{\frac{p}{2}}\det(\Sigma)^{\frac{1}{2}}} \exp \left\{ it^T(Cy + \mu)- \frac{1}{2}y^Ty \right\} \det(C)\mathrm{d}y \\
      & = e^{it^T\mu} \int_{\mathbb R^p} \frac{1}{(2\Pi)^{\frac{p}{2}}} \exp \left\{ \sum_{j=1}^p i\alpha_j y_j -  \frac{1}{2}y_j^2 \right\} \\
      & = \prod_{j=1}^p e^{it_j\mu_j}\Psi_{Y_j}(\alpha_j) \\
      & =^{(1)} \exp\left\{it^T \mu- \frac{1}{2}t^T \Sigma t\right\},
    \end{split}
    \]
    con $\alpha_j = t^T c_{.j}$, y $c_{.j}$ la columna $j-$ésima de la matriz $C$, recordando que:
    \[
    \Psi_{N(\mu,\sigma^2)}(t) = e^{it\mu - \frac{1}{2}\sigma^2 t^2}
    \]
    y en $(1)$ teniendo en cuenta:
    \begin{itemize}
    \item $ \sum_{j=1}^p t_j \mu_j = t^T \mu$,
    \item $\sum_{j = 1}^p \alpha_j^2 = \alpha^T \alpha = t^T CC^T t = t^T \Sigma t$.
    \end{itemize}

  \end{proof}

 
    \begin{nprop}
      Sea $\vectX = (X_1,\dots,X_p)^T \sim N_p(\mu,\Sigma)$ con $\Sigma > 0$. Sea $\vectY = (Y_1,\dots,Y_q)^T$ definido como $\vectY = B\vectX + b$, donde $B\in \mathscr{M}_{q\times p}(\R)$ de rango $q \leq p$, y $b \in \mathbb R^q$. Entonces, se tiene que
      \[
        \vectY \sim N_q\left(B\mu+b,\, B\Sigma B^T\right),
      \]
      con $B\Sigma B^T > 0$.
    \end{nprop}
    \begin{proof}
      Para $t\in \R^q$, se tiene:

      \[
      \begin{split}
        \psi_{\vectY}(t) & = E\left[ e^{it^T(B\vectX + b)} \right] \\
        & = e^{it^Tb} E\left[ i\left(B^Tt\right)^T\vectX \right] \\
        & = e^{it^Tb} \psi_{\vectX}\left(B^Tt\right) \\
        & = e^{it^Tb} \exp\left\{i\left(B^Tt\right)^T \mu- \frac{1}{2}\left(B^Tt\right)^T \Sigma B^Tt\right\} \\
        & = \exp\left\{it^T(B\mu + b) - \frac{1}{2}t^T B\Sigma B^Tt\right\},
      \end{split}
      \]
      y la correspondencia biunívoca entre funciones características y distribuciones concluye la prueba de la primera afirmación.
      Además, se verifica que $B\Sigma B^T > 0$ por ser tanto $\Sigma > 0$ como $B$ de rango máximo.
    \end{proof}

    \begin{ncor}[Marginalización]
      Sea $\vectX = (X_1,\dots,X_p)^T \sim N_p(\mu,\Sigma)$ un vector aleatorio con $\Sigma > 0$. Entonces para todo subvector $\vectX_r = (X_{r_1},\dots, X_{r_q})^T$ —donde $r = (r_1,\dots,r_q)^T$\, con $r_1,\dots,r_q \in \{1,\dots,p\}$ y $q \leq p$— se tiene que $\vectX_r \sim N_q(\mu_r, \Sigma_r)$, siendo
      \begin{itemize}
      \item $\mu_r$ el subvector de $\mu$ correspondiente a $r$,\, y
        \item $\Sigma_r$ la submatriz de $\Sigma$ definida por las filas y columnas correspondientes a $r$.
      \end{itemize}
    \end{ncor}

    \begin{ndef}
      Se dice que $A_{p\times p}$ es diagonalizable si existe una matriz $Q_{p \times p}$ no singular tal que:
      \[
      Q^{-1}AQ = D,
      \]
      con $D$ una matriz diagonal.
    \end{ndef}

    \begin{nth}
      Sea $A$ una matriz $p\times p$. Supongamos que $A$ es diagonalizable por otra matriz $Q_{p \times p}$ no singular, esto es
      \[
      Q^{-1}AQ = D,
      \]
      con $D$ una matriz diagonal. Denotemos $Q = (q_1,\dots,q_p)$, y $D= \operatorname{diag}(d_1,\dots,d_p)$. Entonces:
      \begin{enumerate}
      \item $\operatorname{rango}(A) = $ número de elementos de la diagonal de $D$ que son distintos de 0.
      \item $|A| = \prod_{i = 1}^p d_i$
      \item $\operatorname{tr}(A) = \sum_{i = 1}^p d_i$
      \item El polinomio característico de $A$ es:
        \[
        p(\lambda) = (-1)^p (\lambda-d_1)\dots(\lambda-d_p).
        \]
      \item El espectro de $A$ comprende los escalares distintos incluidos en la diagonal de $D$.
      \item Las multiplicdades algebraicas y geométricas de un autovalor $\lambda$ de $A$ coinciden y son iguales al número de elementos diagonales de $D$ que igualan $\lambda$
      \item Las columnas de $Q$ son autovectores linealmente dependientes de $A$ (la columna $q_i$ es un autovector correspondiente al autovalor $\lambda_i$
      \item Se dice que $A$ es diagonalizable ortogonalmente si es diagonalizable por una matriz ortogonal, esto es, $A = QDQ^T$, con $Q^T = Q^{-1}$, $QQ^T = Q^TQ = I$.
    \end{enumerate}

\end{nth}
\begin{ncor}
              $A$ debe ser simétrica, pues $A^T = (QDQ^T)^T = (Q^T)^T D^T Q^T = QDQ^T = A$. El recíproco es cierto: A es diagonalizable ortogonalmente si y solo si $A$ es simétrica
\end{ncor}

Ahora, para matrices \textbf{definidas no negativas}:
\begin{itemize}
\item $A$ es definida no negativa $\iff$ los autovalores de $A$ son no negativos
\item $A$ es definida positiva $\iff $ los autovalores de $A$ son positivos
\end{itemize}
Y, si $A$ es \textbf{simétrica}:
\begin{itemize}
\item $A$ es definida no negativa $\iff$ los autovalores de $A$ son no negativos
\item $A$ es definida positiva $\iff$ los autovalores de $A$ son positivos
  \item $A$ es semidefinida positiva $\iff$ los autovalores de $A$ son no negativos y al menos uno es nulo

\end{itemize}

Por último vamos a definir lo siguiente:
\begin{ndef}

  Sea una matriz $A_{p\times p}$ simétrica y definida no negativa. Entonces, existe una matriz $R_{p\times p}$ simétrica y definida no negativa tal que
  \[
  A = R^2 = RR
  \]
  Además, $R$ es única y puede expresarse como:
  \[
  R = Q \operatorname{diag}(\sqrt{d_1}, \dots, \sqrt{d_p}) Q^T
  \]
  A $R$  se le llama raíz cuadrada de una matriz simétrica y definida no negativa
\end{ndef}


\subsubsection{Matrices de intercambio de filas/columnas}

Sea $I_{(r,s)}$ la matriz obtenida intercambiando en la identidad $I_p$ las filas $r$ y $s$

%#TODO: Completar esto... dep.

% #TODO: revisar el encaje de esto (clase 18-10-19)

\begin{ejer}
  Sea $\vectX = (X_1, \dots, X_p)^T \sim (\boldsymbol \mu, \Sigma)$ un vector aleatorio con $\Sigma > 0$. \begin{enumerate}
    \item Estudiar el lugar geométrico de los puntos $\Delta(\boldsymbol x, \boldsymbol \mu) = k$ e interpretarlo en funcón de los autovalores de $\Sigma$.
    \item Si además $\vectX \sim N_p(\boldsymbol \mu, \Sigma)$, estudiar el lugar geométrico de los puntos definidos por $f_{\boldsymbol X}(\boldsymbol x) = h$ —con $h \geq 0$— y dar una interpretación geométrica espectral.
  \end{enumerate}
\end{ejer}

\begin{sol}
  Comenzamos dando la solución del apartado (1). Para $k \geq 0$ consideramos \[
      \Delta(\boldsymbol x, \boldsymbol \mu) = k \iff (\boldsymbol x - \boldsymbol \mu)^T\Sigma^{-1}(\boldsymbol x - \boldsymbol \mu) = k^2 \iff (\boldsymbol x - \boldsymbol \mu)^T(k^2\Sigma)^{-1}(\boldsymbol x - \boldsymbol \mu) = 1\,.
      \] Visto así se tiene que los puntos $\boldsymbol x$ solución de esta ecuación definen el \textit{elipsoide} de centro $\boldsymbol \mu$ con ejes principales definidos por los autovectores de $(k^2\Sigma)^{-1}$, siendo la longitud al cuadrado de los semiejes igual a los inversos de los autovalores.

      Descomponiendo $\Sigma = H\Lambda H^T$, podemos escribir \begin{align*}
        (k^2\Sigma)^{-1} 
        &= (kH\Lambda H^T k)^{-1} 
        = k^{-1}(H\Lambda H^T)^{-1} 
        = k^{-1}(H^T)^{-1}\Lambda^{-1}H^{-1}k^{-1} 
        = (k^{-1}H)\Lambda^{-1}(k^{-1}H)^T\\
        % #TODO: añadir anotación HH^T = 1
        &= H(k^2\Lambda)^{-1}H^T\,.
      \end{align*}

      Los autovectores vienen definidos por la matriz $H$ —columnas—, dando las direcciones de los ejes del elipsoide. Los autovalores de $(k^2\Sigma)^{-1}$ vienen dados por los elementos diagonales de la matriz diagonal $(k^2\Lambda)^{-1}$, es decir, los inversos de la forma $(k^2\lambda_j)^{-1}$ de los autovalores de la matriz $k^2\Sigma$.

      Las longitudes al cuadrado de los semiejes coinciden con los propios autovalores $k^2\lambda_j$, por lo que las longitudes de los semiejes son los valores $k\lambda_j^{\sfrac{1}{2}}$.

      Procedemos ahora a ver la solución del apartado (2). La ecuación es, por definición, \[
        f_{\vectX}(\boldsymbol X) 
        = h \iff \frac{1}{(2\pi)^{\sfrac{p}{2}} |\Sigma|^{\sfrac{1}{2}}}\exp\left\{-\frac{1}{2}(\boldsymbol x - \boldsymbol \mu)^T\Sigma^{-1}(\boldsymbol x - \boldsymbol \mu)\right\} 
        = h\,.\] 
      Escribimos entonces \[
        (\boldsymbol x - \boldsymbol \mu)^T \Sigma^{-1}(\boldsymbol x - \boldsymbol \mu) 
        = - 2\ln\left(2\pi^{\sfrac{p}{2}}|\Sigma|^{\sfrac{1}{2}}h\right)\,.
          \]
      El máximo de $f_{\vectX}(\boldsymbol x)$ se alcanza en $\boldsymbol x = \boldsymbol \mu$, que llamaremos \[
        M 
        \vcentcolon= f_{\vectX}(\boldsymbol \mu) 
        = \frac{1}{(2\pi)^{\sfrac{p}{2}} |\Sigma|^{\sfrac{1}{2}}}\,.
      \] Así, la ecuación tendrá solución si $0 < h \leq M$. En ese caso, \[(2\pi)^{\sfrac{p}{2}}|\Sigma|^{\sfrac{1}{2}}\,h \leq 1\,,\] luego \[\log\left((2\pi)^{\sfrac{p}{2}}|\Sigma|^{\sfrac{1}{2}}\,h\right) \leq 0\,.\]
\end{sol}

\subsection{Caracterización de la DNM $(\Sigma > 0)$ en términos de la normalidad de las combinaciones lineales de las componentes}

Dado un vector aleatorio $\vectX = (X_1,\dots,X_p)^T$, decimos que tiene DNM si y solo si toda combinación lineal de las componentes $X_1,\dots,X_P$ de la forma
\[
\alpha^T \vectX
\]
con $\alpha\in \mathbb R^P -\{0\}$, tiene DN univariante no degenerada.

Hasta ahora hemos descrito tenemos las siguientes formulaciones equivalentes de DNM:
\begin{itemize}
\item $[D]$: Definición en términos de la densidad $f_{\vectX}$,
\item $[C-I]$: $\vectX = AZ + \mu$ con $A$ no singular, $\Sigma = AA^T$,
\item $[C-II]$: $\Psi_{\vectX} = e^{it^T\mu - \frac{1}{2}t^T \Sigma t}$, y
\item $[C-III]$: $\alpha^T \vectX \sim N(\alpha^T \mu, \alpha^T \Sigma \alpha), \forall \alpha \in \mathbb R^p - \{0\}$.
\end{itemize}


\begin{nota}[Descomposición espectral de $\Sigma \geq 0$ simétrica]
Como $\Sigma$ es simétrica, podemos escribirla como
\[
\Sigma = H \Lambda H^T,
\]
con $H_{p\times p}$ ortogonal, y $\Lambda_{p \times p}$ diagonal. Sea $r = \operatorname{rango}(\Sigma)$ con $r\leq p$.\\

Supongamos que las matrices se eligen de modo que:
\[
\Lambda = \left(\begin{array}{@{}c|c@{}}
  D & 0 \\\hline
  0 & 0
  \end{array}\right)
\]
con $D_{r\times r}$ no singular, escribiendo entonces:
\[
H = \begin{pmatrix} H_1 & H_2 \end{pmatrix}, \quad \quad H_{1_{p \times r}}, H_{2_{p\times (p-r)}}
\]
Con esto, tenemos:
\[
\Sigma = \begin{pmatrix} H_1 & H_2 \end{pmatrix}\left(\begin{array}{@{}c|c@{}}
  D & 0 \\\hline
  0 & 0
  \end{array}\right)\begin{pmatrix} H_1^T  \\ H_2^T \end{pmatrix} = H_1 D H_1^T
\]

\end{nota}

Utilizando lo que acabamos de ver, podemos escribir $[C-II]$ de la siguiente manera:
\begin{nprop}
  Dado $\vectX = (X_1,\dots,X_p)^T \sim N_p(\mu,\Sigma)$ un vector aleatorio con $\Sigma \geq 0$, se dice que tiene una DNM si su función característica viene dada por:
  \[
\Phi_{\vectX}(t) = E\left[e^{it^T \vectX}\right] = e^{it^T \mu - \frac{1}{2}t^T \Sigma t} \qquad \text{para todo } t \in \mathbb R^p.
  \]

\end{nprop}

\begin{nprop}[Transformaciones lineales de rango no necesariamente máximo]
  Dado un vector aleatorio $\vectX  = (X_1,\dots,X_p)^T \sim N_p(\mu,\Sigma)$ con $\Sigma \geq 0$. Sea $\vectY = (Y_1,\dots,Y_p)^T$ definido como $\vectY = B\vectX + b$, con $B_{q\times p}$ matriz de rango $r \leq \min \{q,p\}$, y $b_{q \times 1}$ un vector.

  Entonces, se tiene que
  \[
  \vectY \sim N_q(B\mu+b, B\Sigma B^T).
  \]

\end{nprop}
\begin{nota}
  La única novedad que nos aporta esta nueva proposición respecto a lo que ya teníamos, es que ahora $B\Sigma B^T \geq 0$ (antes teníamos que tenía que se estrictamente mayor).\\

  La demostración es análoga a la del caso $\Sigma > 0$
\end{nota}


Recordemos la $[C-II]$: Sea $\vectX = (X_1,\dots,X_p)^T\sim N_p(\mu,\Sigma)$, con $\Sigma > 0$, sii
\[
\Psi_{\vectX}(t) = e^{it^T\mu - \frac{1}{2}t^T\Sigma t} \quad \forall t\in \mathbb R^p
\]
Supongamos $\vectX$ un vector aleatorio. Entonces, podemos utilizar una de las caracterizaciones anteriores y escribir:
\[
\Psi_{\vectX}(t) = e^{it^T\mu - \frac{1}{2}t^T\Sigma t} = e^{i^T HH^T \mu - \frac{1}{2}t^T H_1 D H_1^T t} = e^{it^T H_1 H_1^T\mu - \frac{1}{2}t^T H_1DH_1^Tt}* e^{itH_2H_2^T \mu}
\]
La forma de la densidad obtenida, sugiere considerar el siguiente cambio de variable: $\vectX = H\vectY$:
\[
Y = \begin{pmatrix} H_1^T \\ H_2^T \end{pmatrix} \vectX=\begin{pmatrix} H_1^T \vectX \\ H_2^T\vectX \end{pmatrix} = \begin{pmatrix} Y_1 \\ Y_2 \end{pmatrix}
\]

Con esto, se tiene:
\begin{itemize}
\item $\mu_{\vectY} = H^T \mu = \begin{pmatrix} H_1^T \mu \\ H_2^T\mu \end{pmatrix} = \begin{pmatrix} \mu_{\vectY_1} \\ \mu_{\vectY_2} \end{pmatrix}$
  \item $\Sigma_{\vectY} =  H^T \Sigma H = H^T H \Delta H^T H = \Delta = \begin{pmatrix} D & 0 \\ 0 & 0 \end{pmatrix} \implies \begin{cases} \Sigma_{\vectY_1} = D \\ \Sigma_{\vectY_2} = 0\end{cases}$,
\end{itemize}
Y, por tanto:
\[
\begin{cases} \vectY_1 \sim N_r(H_1\mu,D) \quad D> 0 \\ \vectY_2 \equiv H_2\mu \quad p.c.s \end{cases}
\]
Vamos a dar una \textbf{intrepretación} de $\Psi_{\vectX} (t)$:
\[
\Psi_{\vectX}(t) = E[e^{it^T\vectX}] = E[e^{it^TH\vectY}] = E[e^{i(H^Tt)^T \vectY}] = \Psi_{\vectY}(H^T t)
\]
%#TODO: revisar la siguiente letra: \nu
Denotamos
\[
\nu = H^T t = \begin{pmatrix} H_1^T \\ H_2^T\end{pmatrix}t = \begin{pmatrix} H_1^Tt \\ H_2^Tt\end{pmatrix} = \begin{pmatrix} \nu_1 \\ \nu_2\end{pmatrix}
\]
Con esto:
\[
\Psi_{\vectY}(\nu) = \Psi_{\vectX}(t) = e^{i\nu_1\mu_{\vectY_1} - \frac{1}{2} \nu_1^T \Sigma_{\vectY_1}\nu_1}* e^{i\nu_2^T \mu_{\vectY_2}}
\]
y en la última igualdad, el primer término es $\Psi_{\vectY_1}(\nu_1)$ y la segunda es $\Psi_{\vectY_2}(\nu_2)$, con $\vectY_1,\vectY_2$ independientes.\\

Aplicamos a $\vectY_1$ la $[C-I]$: tenemos que $\vectY_1$ puede representarse en distribución como
\[
Y_1 \equiv^d D^{\frac{1}{2}}Z + H_1^T \mu \quad \quad  Z \sim N_r(0,I) 
\]
Para $\vectY_2 \equiv^{p.c.s.} H_2^T \mu \equiv 0*Z + H_2^T \mu$, tenemos, multiplicando por $H$:
\[
\vectY = \begin{pmatrix} \vectY_1 \\ \vectY_2 \end{pmatrix} \equiv \begin{pmatrix} D^{\frac{1}{2}} Z + H_1\mu \\ 0Z+ H_2^T \mu\end{pmatrix} = \begin{pmatrix} D^{\frac{1}{2}} \\ 0 \end{pmatrix}Z + H^T \mu
\]
Vemos que:
\[
\vectX = H\vectY = H\begin{pmatrix} D^{\frac{1}{2}} \\ 0\end{pmatrix} Z + HH^T\mu = H_1D^{\frac{1}{2}}Z + \mu
\]

Denotando $A= H_1D^{\frac{1}{2}}$ se tiene finalmente:
\[
X \equiv^{d} AZ + \mu
\]
con $A$ una matriz $p\times r$, con $r\leq p$ de rango $r$ y $Z\sim N_r(0,I)$.\\

Recíprocamente, supongamos que se cumple lo anterior. Tenemos que, por el teorema de inversión de P. Lévy,
\[
\Psi_{\vectX} = \Psi_{AZ + \mu}  = E[e^{it^T(AZ +  \mu)}] = E[e^{it^T(AZ)}]e^{it^T\mu} = E[e^{i(A^t t)^T Z}]e^{it^T \mu} = \Psi_Z(A^Tt)*e^{it^T \mu} = (*)
\]
como $\Psi_Z(t) = e^{it^T 0 - \frac{1}{2}t^tIt} = t^{-\frac{1}{2}t^t t}$, entonces:
\[
(*) = e^{-\frac{1}{2}(A^T t)^T(A^T t)} e^{it^T \mu} = e^{-\frac{1}{2}t^T A A^T t} e^{it^T \mu} = (**)
\]
y, como tenemos:
\begin{itemize}
  \item $\mu_{\vectX} = E[\vectX]=A*0 +\mu = \mu$
  \item $\Sigma_{\vectX} = Cov(\vectX) = AIA^T = AA^T$
\end{itemize}
Nos queda:
\[
(**) = e^{it\mu - \frac{1}{2}t^T(AA^T)t} \implies \vectX \sim N_p(\mu, AA^T) \quad AA^T \geq 0
\]

Con todo este procedimiento, podemos enunciar la siguiente caracterización, que es una extensión de $[C-I]$.
\begin{nprop}[$\bar{[C-I]}$]
  Un vector aleatorio $\vectX = (X_1,\dots,X_p)^T$ tiene DNM de vector de medias $\mu$ y matriz de covarianzas $\Sigma$ (con $\Sigma \geq 0$) , $N_P(\mu,\Sigma)$ si y solo si se puede representar:
  \[
\vectX \equiv^{d} AZ + \mu \quad con \quad \begin{cases} A_{p\times r}, \ \ rango(A) = r, \ \ AA^T = \Sigma \\ Z\sim N_r(0,I)\end{cases}
  \]
\end{nprop}
Enunciamos también una nueva versión de $[C-III]$:
\begin{nprop}[$\bar{[C-III]}$]
  Un vector aleatorio $\vectX = (X_1,\dots,X_p)^T$ tiene DNM (con $\Sigma \geq 0$) si y solo si toda combinación lineal de las componentes $X_1,\dots,X_p$ de la forama:
  \[
\alpha^T X \quad \quad \alpha^T \in \mathbb R^p
\]
tiene distribución normal (univariante, posiblemente degenerada)
\end{nprop}
\begin{proof}
Es análoga al caso anterior
\end{proof}

\begin{ndef}[$\bar{[D]}$]
  En el subespacio (afín) generado por los autovalores (y correspondientes autovectores) no nulos, se tiene que se puede formular la función de densidad normal multivariante.

\end{ndef}

\begin{nprop}[Normalidad de C.L. de vectores con DNM independientes]
  Sean $\vectX_k$ con $k = 1,\dots,m$ v.a. $p-$dimensionales independientes con distribución $N_p(\mu_k, \Sigma_k)$ respectivamente. Entonces, para cualquier conjunto de matrices constantes $A_k$ con $k = 1,\dots, m$ de dimensión $q\times p$, se ferifica que:
  \[
Y:= \sum_{k=1}^{m} A_k X_k \sim N_q(\sum_{k=1}^m(A_k \mu_k), \sum_{k=1}^m(A_k \Sigma_k A_k^T))
  \]
\end{nprop}
%#TODO: Demostración, es seguir de la definición

\begin{nth}[Cramer]
  Sean $\vectX_1$ y $\vectX_2$ vectores aleatorios y $p$-dimensionales independientes, con $\vectX_1 \sim N_p(\mu_1, \Sigma_1)$ y $\vectX_2 \sim N_p(\mu_2, \Sigma_2)$. Entonces, se tiene que $\vectX_1 + \vectX_2 \sim N_p(\mu_1 + \mu_2, \Sigma_1 + \Sigma_2)$.
\end{nth}

La conjetura de Paul Lévy preguntaba si el recíproco de esta afirmación es cierta. Cramer dio una solución para $p=1$ en el año 1936, de la que es sencillo inducir la solución para cualquier $p$. 

\begin{nth}[Recíproco del teorema de Cramer]
  Sean $\vectX_1$ y $\vectX_2$ vectores aleatorios y $p$-dimensionales independientes, tales que $\vectX_1 + \vectX_2$ tiene DNM. Entonces, $\vectX_1$ y $\vectX_2$ también tienen DNM.
\end{nth}

\begin{ejer}
  Suponiendo que este último resultado es cierto para $p=1$, probar que es cierto para cualquier $p$.
\end{ejer}

% ¿Es esto una sección nueva?
\subsection{Distribuciones esféricas y elípticas}

\subsubsection{Caracterización en términos de la densidad esférica estándar (C-IV)}

Vamos a comenzar razonando cómo podemos trabajar con coordenadas polares. Sea $\vectX \sim N_p(\boldsymbol \mu, \Sigma)$, con $\Sigma > 0$. Para cada $k > 0$ la ecuación \[(\boldsymbol x - \boldsymbol \mu)^T \Sigma^{-1}(\boldsymbol x - \boldsymbol \mu) = k^2\] define un elipsoide en $\mathbb R^p$ sobre el que la densidad $f_{\vectX}$ es constante. 

Sea $\Sigma = CC^T$, con $C_{p\times p}$ no singular. Definimos $\boldsymbol Z = C^{-1}(\boldsymbol X - \boldsymbol \mu)$, con $\boldsymbol Z \sim N_p(\boldsymbol 0, \mathcal I_p)$, a partir del cual vamos a definir el vector aleatorio —excepto para el caso $\boldsymbol Z = 0$— \[\boldsymbol U = \frac{\boldsymbol Z}{\Vert \boldsymbol Z \Vert} = \frac{\boldsymbol Z}{(\boldsymbol Z^T \boldsymbol Z)^{1/2}}\,,\] que tiene norma 1. Observamos que $\boldsymbol U$ se distribuye sobre la esfera unidad en $\mathbb R^p$.

Definimos a continuación $H \boldsymbol U$, con $H$ matriz ortogonal. Entonces, \[
  H \boldsymbol U = H \frac{\boldsymbol Z}{\left(\boldsymbol Z^T \boldsymbol Z\right)^{1/2}} = \frac{H\boldsymbol Z}{\left(\boldsymbol Z^T H^T H\boldsymbol Z\right)^{1/2}} = \frac{H\boldsymbol Z}{\Big((H\boldsymbol Z)^T (H\boldsymbol Z)\Big)^{1/2}}\,.
\] Observamos que $H \boldsymbol Z \sim N_p(H \boldsymbol 0,\, H \mathcal I_pH^T) = N_p(\boldsymbol 0,\, \mathcal I_p)$. Por tanto tenemos que $\boldsymbol U$ y $H \boldsymbol U$ se distribuyen de igual forma. Ahora, como la distribución de $\boldsymbol U$ no se ve afectada bajo ninguna isometría —multiplicando por una matriz ortogonal—, concluimos que $\boldsymbol U$ se distribuye uniformemente sobre la esfera unidad en $\mathbb R^p$, lo que llamamos \textit{densidad esférica estándar}. Para recuperar la información de $\boldsymbol Z$ a partir de $\boldsymbol U$ necesitamos conocer el «radio» $R = (\boldsymbol Z^T \boldsymbol Z)^{1/2} = \Vert Z \Vert$.

\begin{nprop}
  El vector aleatorio $\boldsymbol U = \frac{\boldsymbol Z}{(\boldsymbol Z^T \boldsymbol Z)^{1/2}}$ y la variable aleatoria $R = (\boldsymbol Z^T \boldsymbol Z)^{1/2}$ son independientes.
\end{nprop}

% 23-10-19

\begin{nprop}
  Un vector aleatorio $\vectX = (X_1, \dots, X_n)$ tiene DNM $N_p(\boldsymbol \mu, \Sigma)$ con $\Sigma > 0$ si, y solo si, $\vectX$ es un vector aleatorio correspondiente al siguiente «experimento»:

  \begin{enumerate}
    \item Obtener el valor de una variable aleatoria $R^2 \vcentcolon= V \sim \chi_p^2$. Se identifica $V = v$.
    \item Elegir un punto aleatoriamente —es decir, según la distribución uniforme— en la esfera $p$-dimensional de radio $r = \sqrt{v}$. \[\vectZ = (Z_1, \dots, Z_p)^T = (z_1, \dots, z_p)^T = \boldsymbol z\,.\]
    \item Dado el vector de medias $\boldsymbol \mu$ y la matriz de covarianzas $\Sigma > 0$, elegir $C$ (no singular) tal que $\Sigma = CC^T$ y obtener el valor de $\vectX$ mediante la expresión $X\vcentcolon= C\vectZ + \boldsymbol \mu$. De nuevo se identifica $\vectX = \boldsymbol x = C\boldsymbol z + \boldsymbol \mu$.
  \end{enumerate}
\end{nprop}

\begin{proof}
  Veamos la justificación de la implicación ($\implies$). Por la caracterización (C-I) sabemos que podemos escribir $\vectX = C\vectZ + \boldsymbol \mu$, con $\vectZ \sim N_p(\boldsymbol 0, I_p)$.

  Escribimos, para $\vectZ \neq 0$, \[\vectZ = \left(\vectZ^T\vectZ\right)^{\sfrac{1}{2}} \cdot \frac{\vectZ}{\left(\vectZ^T\vectZ\right)^{\sfrac{1}{2}}} = R \cdot \boldsymbol U\,,\] siendo $R$ y $\boldsymbol U$ independientes.

  Ahora tenemos que: \begin{itemize}
    \item \[R^2 = \vectZ^T\vectZ = Z_1^2 + \cdots + Z_p^2 \sim \chi_p^2\]al ser $Z_1, \dots, Z_p$ independientes con $Z_i \sim N_p(0, 1)$ para $i = 1, \dots, p$,
    \item $\boldsymbol U$ tiene distribución uniforme sobre la esfera unidad.
  \end{itemize}

  A partir de la expresión $\vectX = C \cdot R \cdot \boldsymbol U + \boldsymbol \mu$ se tiene que para generar los valores de $\vectX$ se puede seguir el procedimiento del enunciado.
\end{proof}

\section{Distribuciones esféricas y elípticas}

\subsection{Distribuciones esféricas}

\begin{ndef}
  Se dice que un vector aleatorio $\vectX = (X_1, \dots, X_p)^T$ tiene una distribución \textit{esférica} si $\vectX$ y $H\vectX$ tienen la misma distribución para toda matriz ortogonal $H$.
\end{ndef}

Veamos a continuación dos caracterizaciones de las distribuciones esféricas.

\begin{nprop}[Caracterización en términos de la densidad]
  Dado un $\vectX$ un vector aleatorio $p$-dimensional con función de densidad $f_{\vectX}$ tiene distribución esférica si y solo si $f_{\vectX}$ puede expresarse como $g(\boldsymbol x^T \boldsymbol x)$ para alguna función escalar $g : \mathbb R_0^+ \to \mathbb R_0^+$.
\end{nprop}

\begin{nprop}[Caracterización en términos de la función característica]
  Dado un $\vectX$ un vector aleatorio $p$-dimensional con función de densidad $f_{\vectX}$ tiene distribución esférica si y solo si su función característica tiene la forma $\xi(\boldsymbol t^T \boldsymbol t)$ para alguna función escalar $\xi: \mathbb R_o^+ \to \mathbb R_0^+$.
\end{nprop}

\subsection{Distribuciones elípticas}

\begin{ndef}[Distribución elíptica]
  Se dice que un vector aleatorio $\vectX = (X_1, \dots, X_p)^T$ tiene \textit{distribución elíptica} de parámetros $\boldsymbol \mu (p \times 1)$ y $V (p \times p)$ —simétrica y definida positiva— si se puede expresar de la forma $\vectX = A\vectZ + \boldsymbol \mu$, con $V = AA^T$ y $\vectZ$ un vector aleatorio con distribución esférica. Se denota como $\vectX \sim E_p(\boldsymbol \mu, V)$.
\end{ndef}

\begin{nprop}[Caracterización en función de la densidad]
  Dado un $\vectX$ un vector aleatorio $p$-dimensional con función de densidad $f_{\vectX}$ tiene distribución elítica si y solo si $f_{\vectX}$ puede expresarse como \[|V|^{-\sfrac{1}{2}}g\left((\boldsymbol x - \boldsymbol \mu)V^{-1}(\boldsymbol x - \boldsymbol \mu)\right)\] para alguna función escalar $g : \mathbb R_0^+ \to \mathbb R_0^+$.
\end{nprop}

\begin{nprop}[Caracterización en términos de la función característica]
  Dado un $\vectX$ un vector aleatorio $p$-dimensional con función de densidad $f_{\vectX}$ tiene distribución esférica si y solo si su función característica tiene la forma \[e^{i \boldsymbol t^T \boldsymbol \mu}\xi(\boldsymbol t^T V \boldsymbol t)\] para alguna función escalar $\xi: \mathbb R_o^+ \to \mathbb R_0^+$.
\end{nprop}

% TODO: Falta observaciones sobre la distribución elíptica
% TODO: Falta lo siguiente a esto
